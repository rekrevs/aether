\section{Predictions and Numerical Targets}

We summarize the qualitative and quantitative predictions of the framework in a form suitable for comparison with existing and future experiments.

\textbf{Negative predictions (should not be seen):}

\begin{itemize}
\item No deviations from tested gravitational laws, no observable vacuum dispersion, and no anomalous signals in torsion-balance or gravitational-wave experiments. The metric sector remains governed by standard GR with $\alpha\equiv 1$; aether resonance affects only the distribution of stress--energy, not the gravitational coupling.
\item No robust effects in homogeneous crystals or fluids: in the limit $N\gg 1$ equivalent placements of a pattern, degeneracy dilution drives the pattern quality factor $\mathcal Q\to 0$ so that $\Psig\to 0$ (Sec.~5--6).
\item No signals in accelerator experiments: beams and final states are approximately translationally invariant and thermal on microscopic scales, implying $\langle O_S\rangle_{\rm in}\approx\langle O_S\rangle_{\rm out}\approx 0$ even though the seed operator is built from irrelevant operators with $\Delta>4$ (Sec.~5.3). Collider cross-sections therefore receive negligible contributions from the $S$-channel.
\item No everyday signalling without shared structure and pump: in the absence of deliberately engineered, high-$\mathcal Q$ patterns and nontrivial pump power $P_{\rm pump}$, Eq.~(6.1) forces $\Psig\approx 0$.
\end{itemize}

\textbf{Positive predictions with numerical targets:}

\subsection{Pred.\ 1: Twin-reservoir correlations (E1)}

\textbf{Goal.} A statistically significant reduction in bit error rate (BER) between two spacelike-separated reservoirs that share structure in $S$ and are both driven by high-quality pumps, compared to carefully matched control configurations that differ only in the pattern features entering $O_S$. Such reservoirs can be implemented using physical reservoir computing architectures, including photonic and electronic reservoirs, which are known to provide high-dimensional dynamical feature maps for complex input patterns~\cite{tanaka2019_physical_reservoir,vandersande2017_photonic_reservoir}.

Specializing the thermodynamic resource inequality (8.3) to the E1 setup gives an upper bound on the achievable FTL bit rate per edge $e$,
\[
  R_{\rm bit}(e)
  \;\le\;
  \frac{\beta}{k_B T \ln 2}\,
  \varepsilon\,\Ksig(e)\,\mathcal Q(e)\,\tilde{\Delta\Phi}(\tilde e)\,P_{\rm pump}(e),
\]
and a corresponding lower bound on the BER that can be achieved for a fixed $P_{\rm pump}$. A null result at sensitivity $R_{\rm bit}^{({\rm null})}$ then constrains the combination
\[
  \varepsilon\,\Ksig(e)\,\mathcal Q(e)\,\tilde{\Delta\Phi}(\tilde e)
  \;<\;
  \frac{k_B T \ln 2}{\beta\,P_{\rm pump}(e)}\,R_{\rm bit}^{({\rm null})},
\]
for edges whose $d_\sigma$ is calibrated via the distance ladder (Sec.~7.3). Conversely, a positive result in which the BER improvement scales with the ladder level $\ell$ as $\propto \exp[-d_\sigma(\ell)/\lambda_\sigma]$ would provide direct evidence for a nonzero $\lambda_\sigma$.

\subsection{Pred.\ 2: Energy tunnel and sidereal modulation (E2)}

\textbf{Goal.} Detect or bound a small, persistent asymmetry in the energy balance between two macroscopic reservoirs $A$ and $B$ that are structurally matched in $S$ and rotated with respect to the substrate rest frame. The basic energy balance is
\begin{equation}
\Delta E_A + \Delta E_B = \Psig \cdot \Delta t,
\tag{12.2}
\end{equation}
where $\Delta E_{A,B}$ are the energy changes in the reservoirs over an interval $\Delta t$ and $\Psig$ is the net power into the $S$-sector along the active edges, given phenomenologically by Eq.~(6.1).

For representative experimental parameters and integration time $\Delta t=10^3\,$s, three illustrative scenarios are:

\begin{table}[h]
\centering
\begin{tabular}{llll}
\toprule
Scenario & $Q$ & $\Delta E$ (over $10^3$ s) & Detectability \\
\midrule
\textbf{Baseline} & $10^{-5}$ & $\sim 10^{-40}$ J & Not detectable ($\delta E \sim 10^{-26}$ J) \\
\textbf{Target}   & $10^{-3}$ & $\sim 10^{-27}$ J & Below current limit but approaching \\
\textbf{Ambitious} & $10^{-2}$ & $\sim 10^{-25}$ J & Marginally detectable at limit \\
\bottomrule
\end{tabular}
\end{table}

Here $\delta E$ denotes a plausible sensitivity of state-of-the-art cryogenic calorimetry.
The ``Baseline'' row is intended to correspond to demonstrated performance
of existing devices; the ``Target'' and ``Ambitious'' scenarios require roughly one
and two orders of magnitude improvement in energy sensitivity, respectively,
which we regard as realistic on a decadal time scale rather than as
far-future technology.  These numbers are indicative only; the key point is
that $\Delta E$ scales linearly with $\varepsilon Q$ and with integration time, and that
null results at the ambitious level would bound $\varepsilon Q$ to be far below unity
for the probed platform.

Because the Earth rotates relative to the substrate rest frame, any preferred-frame effect should also give a sidereal modulation in $\Psig$ with frequency $\Omega_\oplus$ (sidereal day):
\begin{equation}
  \Psig(t) \;=\; \bar \Psig\left[1+ \mathcal A\cos\!\big(\Omega_\oplus t + \phi\big)\right],
  \tag{12.3}
\end{equation}
where $\mathcal A$ is a dimensionless amplitude and $\phi$ a phase set by the orientation at some reference time. In the matter sector an effective parametrization is
\begin{equation}
  A_{\rm sid}^{(\mathrm{mat})} \;\simeq\;
  \frac{\epsmat}{\epsgam} \cdot \frac{\lsig}{L_{\rm exp}} \cdot \frac{\Qmat}{\Qgam} \cdot \Xi \,,
  \tag{12.4}
\end{equation}
where $\epsmat,\Qmat$ are the matter-sector coupling and quality factor, $\epsgam,\Qgam$ their photon-sector counterparts, $L_{\rm exp}$ is the apparatus size, and $\Xi=\mathcal{O}(1)$ is a geometry factor.

\textbf{Important:} Eq.~(12.4) emphasizes that the E2 sidereal signal is primarily sensitive to the \emph{ratio} $(\epsmat,\Qmat)$ to $(\epsgam,\Qgam)$. Bounds in Sec.~11.4 constrain $\epsgam\Qgam$ in the photon sector, so a positive E2 signal would effectively determine or bound $\epsmat\Qmat$ for the matter sector. A null result at target sensitivity $A_{\rm sid}\gtrsim 10^{-20}$ (achievable with integration over $\sim 10^7$\,s) would translate into a corresponding upper bound on $\epsmat\Qmat(\lsig/L_{\rm exp})$.

\subsection{Pred.\ 3: Complexity-dependent hop rate (E3)}

\textbf{Goal.} Demonstrate that the rate of apparent FTL influence (e.g.\ correlation strength or effective $\Psig$) as a function of configurational entropy $\Sigma$ exhibits a nontrivial maximum at some optimal complexity $\Sigma_{\rm opt}$, as expected if $\mathcal Q$ peaks near critical or edge-of-chaos regimes.

In E3-type experiments one scans over a family of configurations of a given platform (e.g.\ electronic networks, photonic lattices, or spin systems), varying a control parameter that tunes complexity (e.g.\ connectivity, disorder, or pump strength). For each configuration one infers an effective hop rate $r_{\rm hop}(\Sigma)$ or a related measure of FTL strength. The model predicts:
\begin{itemize}
  \item $r_{\rm hop}(\Sigma)\approx 0$ for very low complexity (ordered, crystalline) and very high complexity (fully random) states, where degeneracy dilution drives $\mathcal Q\to 0$.
  \item A peak $r_{\rm hop}(\Sigma_{\rm opt})>0$ at intermediate complexity, where patterns selected by $O_S$ are rare enough to avoid dilution but common enough to be accessible.
\end{itemize}
A null result in which $r_{\rm hop}(\Sigma)$ is consistent with zero across the scanned range would bound $\lambda_\sigma$ and $\mathcal Q$ for that class of configurations.

\subsection{Summary of parameter combinations}

For clarity, Table~\ref{tab:observable-constraints} summarizes which experimental observables primarily constrain which products of model parameters at leading order.

\begin{table}[h]
\centering
\small
\begin{tabular}{lll}
\toprule
Observable & Primary constraint & Secondary / notes \\
\midrule
E1 ($\Delta$BER, coherence) & $\varepsilon\,\lambda_\sigma\,\mathcal Q$ & via Eqs.~(8.3)--(8.4) and ladder (Sec.~7.3) \\
E2 ($\Delta E$ over time) & $\varepsilon\,\omega_0\,\mathcal Q$ & Power form (6.1) and Eq.~(12.2) \\
E2 (sidereal amplitude) & $\varepsilon\,\mathcal Q\cdot(\lambda_\sigma/L_{\rm exp})$ & Geometry factor $\Xi$ (Sec.~11.3) and Eq.~(12.3) \\
E3 (hop rate vs.\ complexity) & $\lambda_\sigma$, $\mathcal Q$ & Peak location near $\Sigma_{\rm opt}$ \\
\bottomrule
\end{tabular}
\caption{Leading-order parameter combinations constrained by the proposed experiments.}
\label{tab:observable-constraints}
\end{table}

Table~\ref{tab:predictions-bounds} complements this by explicitly listing the target signals for a positive result and the null bounds that would be placed on parameter combinations if no signal is observed.

\begin{table}[h]
\centering
\small
\begin{tabular}{lll}
\toprule
Experiment & Positive signal & Null bound \\
\midrule
E1 (ansible) & $\Delta{\rm BER} \sim 10^{-3}$ (match vs.\ mismatch) & $\varepsilon \lambda_\sigma \mathcal Q < 10^{-12}$ m \\
E2 (energy) & $\Delta E > 10^{-25}$ J ($\mathcal Q \sim 10^{-2}$) & $\varepsilon \omega_0 \mathcal Q < 10^{-8}$ Hz \\
E2 (rotation, matter) & $A_{\rm sid} \geq 10^{-20}$ (3$\sigma$, $10^7$ s) & $\varepsilon_{\rm mat} \mathcal Q_{\rm mat} < 10^{-20}$ \\
E3 (chaos) & $r_{\rm sync}$ peak at $\Sigma_{\rm opt}$ & $\varepsilon \mathcal Q < 10^{-15}$ \\
\bottomrule
\end{tabular}
\caption{Summary of target signals and null bounds for the three proposed experiments. These are illustrative targets; actual limits depend on integration time, systematic uncertainties, and achieved experimental sensitivity.}
\label{tab:predictions-bounds}
\end{table}
