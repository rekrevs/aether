\section{Discussion and Implications}

The hypothesis of aether resonance places superluminal transfer at the substrate level, where it is local with respect to a structural distance $d_\sigma$ on pattern space $S$. In the emergent spacetime $(M,g)$ this shows up as apparent FTL redistribution of energy and information, but only under strict conditions:
\begin{itemize}
  \item sufficiently small structural distance $d_\sigma$ (shared pattern structure);
  \item high pattern quality factor $Q$ (near-critical, low-entropy configurations);
  \item non-negligible pump power and entropy production (thermodynamic cost).
\end{itemize}
Outside this regime—e.g.\ in homogeneous crystals, thermal baths, collider beams, or everyday macroscopic systems—degeneracy dilution drives the effective $\langle O_S\rangle$ to zero and the substrate channel is effectively closed.

By deriving the framework from a variational principle, we ensure
consistency with general relativity and a well-defined $T^S_{\mu\nu}$.
The Einstein equations follow from the total action and the split
conservation laws, and the Bianchi identity forces a universal
gravitational coupling $\alpha\equiv 1$ in the presence of the exchange
current $J^\nu_\sigma$ (Sec.~\ref{subsec:alpha-constraint}). There is thus:
\begin{itemize}
  \item no modification of the light-speed causal structure of $(M,g)$;
  \item no freedom to weaken the gravitational response to the $S$-sector.
\end{itemize}
All apparent FTL behaviour is pushed into substrate locality and the pattern-dependent excitation of $T^S_{\mu\nu}$.

Momentum-neutrality (Sec.~3.4) guarantees that there are no reactionless drives: integrated spatial momentum transfer between sectors vanishes, and any momentum gained in one region of $M$ must be balanced elsewhere. The localization of $S$-flows as sources in $M$ via a compactly supported pushforward (Sec.~3.3) ensures that the effective sources entering the Einstein equations are well-defined and diffeomorphism-covariant.

The existence of a local $S$-mediator (Sec.~4) shows how structural proximity can be implemented using only local rules on $S$, without a global search: the mediator propagates at finite speed $c_S$ and generates an exponentially decaying kernel $K_\sigma=\exp[-d_\sigma/\lambda_\sigma]$. The modified Lieb–Robinson bound (Sec.~9) quantifies how this alters microcausality: it introduces a soft tail outside the usual light cone, but always exponentially suppressed in $d_\sigma$ and retarded in substrate time. The category-theoretic causality proof with the anti-telephone rule (Sec.~10) then guarantees that no closed causal loops or antitelephone protocols are possible once $\tau$-monotonicity is enforced locally.

Thermodynamic resource bounds (Sec.~8) connect the substrate power
$\Psig$ to an information-theoretic cost, yielding a Landauer-like
limit on FTL bitrates.  Combined with the $d_\sigma$-metric and distance
ladder (Sec.~7), this makes the framework strongly falsifiable.  Each
class of experiment (E1--E3) constrains specific products of parameters
$(\varepsilon,\lambda_\sigma,Q,\omega_0)$, and null results can be
reported as explicit inequalities.

Finally, the experimental design in Sec.~13 situates the proposal firmly in the ``no-loophole'' tradition: pre-registered protocols, commit–reveal schemes, spacelike separation, environmental vetoes, and multiple-test corrections are all built in from the beginning. Whether or not a signal is seen, the outcome is informative: either we obtain sharp upper bounds on a clearly defined region of parameter space, or we open a new empirical domain where substrate-local effects can be quantified and further scrutinized.
