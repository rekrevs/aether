\section{Operationalized $d_\sigma$-Metric}

Algorithmic similarity between two microscopic configurations is in general non-computable. For phenomenology and experiment design we therefore introduce a \textbf{practical proxy} for the structural distance $d_\sigma$ and the associated similarity factor $\Ksig$.

\subsection{Feature embedding and surrogate distance}

We associate to each substrate configuration $\sigma\in S$ a finite-dimensional feature vector
\[
\mathcal{I}(\sigma)
=
\big(\mathcal{I}_1(\sigma),\ldots,\mathcal{I}_{N_{\mathrm{feat}}}(\sigma)\big),
\]
where the components $\mathcal{I}_I$ are chosen to be:
\begin{itemize}
  \item invariant under gauge transformations and diffeomorphisms in $M$;
  \item local or quasi-local in the emergent spacetime (e.g.\ power spectra, correlation functions, topological descriptors);
  \item experimentally accessible in principle (e.g.\ from probe measurements, imaging, or electrical response).
\end{itemize}
Given positive weights $w_I>0$ and an embedding scale $\ell_0$, we define
\begin{equation}
\begin{split}
  d_\sigma(\sigma,\sigma')
  &:= \ell_0\,
  \left(
    \sum_{I=1}^{N_{\mathrm{feat}}}
      w_I\,
      \big\|
        \mathcal{I}_I(\sigma)
        -\mathcal{I}_I(\sigma')
      \big\|^2
  \right)^{1/2},
  \\
  \Ksig(\sigma,\sigma')
  &\equiv
  \exp\!\big[-d_\sigma(\sigma,\sigma')/\lambda_\sigma\big].
\end{split}
\tag{7.1}
\end{equation}
This construction makes $d_\sigma$ a bona fide metric on $S$ (up to equivalence of configurations with identical feature vectors) and implements the exponential falloff assumed in the mediator kernel.

\subsection{Choice of features}

The specific choice of features is not unique and can be adapted to the experimental platform. Examples include:
\begin{itemize}
  \item local Fourier spectra or wavelet coefficients (capturing periodicity and band structure);
  \item two-point and higher-order correlators (capturing spatial organization and long-range order);
  \item topological descriptors (e.g.\ Betti numbers from persistent homology of level sets of a field);
  \item response functions (e.g.\ impedance spectra, transfer functions of neural or electronic circuits).
\end{itemize}
The scale $\ell_0$ sets the typical size of $d_\sigma$ for a ``unit'' change in features. In practice, we will choose $\ell_0$ and $\lambda_\sigma$ such that a qualitatively clear change in the pattern (e.g.\ block-scrambling or strong noise) corresponds to $d_\sigma \sim \lambda_\sigma$, while minor modifications (e.g.\ global phase rotations) yield $d_\sigma\ll\lambda_\sigma$.

\subsubsection{Example feature maps with explicit invariants}

For concreteness, one can build the feature map from explicit local gauge/diffeomorphism-invariant scalars constructed from visible-sector fields and curvature. Examples include:

\begin{center}
\begin{tabular}{ll}
\toprule
Feature & Definition \\
\midrule
$\mathcal{I}_1$ & $F_{\mu\nu}F^{\mu\nu}$ (EM field strength squared) \\
$\mathcal{I}_2$ & $\bar\psi\gamma^\mu D_\mu\psi$ (fermion current density) \\
$\mathcal{I}_3$ & $R_{\mu\nu\rho\sigma}R^{\mu\nu\rho\sigma}$ (curvature invariant) \\
$\mathcal{I}_4$ & $\nabla_\mu\phi\nabla^\mu\phi$ (scalar kinetic term) \\
\bottomrule
\end{tabular}
\end{center}

Each $\mathcal{I}_I$ is built from fields and a \emph{finite} number of derivatives, ensuring locality in $M$. The number of features $N_{\mathrm{feat}}$ is kept finite for EFT control. For the electromagnetic sector, $F_{\mu\nu}=\partial_\mu A_\nu - \partial_\nu A_\mu$ is manifestly gauge-invariant under $A_\mu \to A_\mu + \partial_\mu\chi$; similarly, curvature scalars like $R_{\mu\nu\rho\sigma}R^{\mu\nu\rho\sigma}$ are diffeomorphism-invariant, ensuring the framework respects all relevant symmetries.

\subsubsection{Alternative: signature-based extraction pipeline}

For time-evolving or spatially extended configurations (e.g.\ in E1 or E3 experiments), one can extract features via a signature-based pipeline:

\begin{itemize}
\item \textbf{Spectral:} FFT $\to$ power spectrum $\to$ dominant frequencies $(f_1,\ldots,f_m)$, spectral entropy $H_{\mathrm{spec}}$;
\item \textbf{Topological:} persistent homology $\to$ Betti curves $(\beta_0(r),\beta_1(r))$;
\item \textbf{Statistical:} autocorrelation time $\tau_c$, Lyapunov exponents $\lambda_i$ (if applicable), moments $(\mu_k)$.
\end{itemize}

The combined signature is then a vector
\[
\sigma(s) = (f_i,\,H_{\mathrm{spec}},\,\beta_j(r_k),\,\tau_c,\,\lambda_i,\,\mu_k) \in \mathbb{R}^d,
\]
and the structural distance can be augmented with a Gromov--Wasserstein term to account for distributional differences:
\begin{equation}
d_\sigma(s,s') := \ell_0 \left[ \|\sigma(s) - \sigma(s')\|_2 + \alpha_W \, W(\mu_s,\mu_{s'}) \right],
\tag{7.2}
\end{equation}
where $W$ is the Gromov--Wasserstein distance between probability distributions $\mu_s,\mu_{s'}$, and $\alpha_W\ge0$ is a weighting parameter. This extended metric is particularly useful for complex, high-dimensional dynamical systems where simple Euclidean distances between features may not fully capture structural similarity. \emph{This is an optional refinement for complex dynamical systems; all our bounds and experiments are already meaningful with the simpler Eq.~(7.1).}

\subsection{Distance ladder for calibration}

To connect the abstract $d_\sigma$ to observable quantities, we introduce a \textbf{distance ladder}: a set of calibrated transformations labelled by an integer level $\ell=0,\ldots,5$ applied to a reference pattern $s$, producing configurations $s_\ell$. For each level we measure or infer a typical structural distance $\varepsilon_\ell:=d_\sigma(s,s_\ell)$ and the corresponding similarity $K_{\sigma,\ell}:=\exp[-\varepsilon_\ell/\lambda_\sigma]$.

An illustrative ladder is:

\begin{table}[t]
\centering
\begin{tabular}{@{}c l c c@{}}
\toprule
Level $\ell$ & Transformation & $d_\sigma$ scale & $K_{\sigma,\ell}$ \\
\midrule
0 & Identical $(s' = s)$ & $0$ & $1$ \\
1 & Phase rotation (spectrum preserved) & $\varepsilon_1 \ll \lambda_\sigma$ & $\approx 0.9$ \\
2 & Permuted label / mild relabelling & $\varepsilon_2 \approx 0.3\,\lambda_\sigma$ & $\approx 0.7$ \\
3 & Block-scramble (temporal/spatial) & $\varepsilon_3 \approx 0.7\,\lambda_\sigma$ & $\approx 0.5$ \\
4 & Additive noise (SNR $=10$ dB) & $\varepsilon_4 \approx \lambda_\sigma$ & $\approx 0.37$ \\
5 & Independent realization & $\varepsilon_5 \gg \lambda_\sigma$ & $\ll 0.1$ \\
\bottomrule
\end{tabular}
\end{table}

In practice, the values in the last two columns are not assumed but \emph{measured} (or inferred) by running the full experimental protocol with controlled pattern distortions. The ladder then provides a direct, empirical link between:
\begin{itemize}
  \item observable changes in the experimental pattern;
  \item the corresponding change in $d_\sigma$;
  \item the decay of the effect through the factor $\Ksig=\exp[-d_\sigma/\lambda_\sigma]$.
\end{itemize}

\subsection{Pre-registration}

For the proposed experiments E1--E3 we envisage pre-registering the distance ladder as part of the analysis plan:
\begin{itemize}
  \item the transformations corresponding to each level $\ell$ are fixed in advance and implemented blindly (e.g.\ by a separate team or by automated code);
  \item the ladder is used to fit or constrain $\lambda_\sigma$ by comparing the observed effect size at each level with the expected $\exp[-d_\sigma/\lambda_\sigma]$ falloff;
  \item the mapping between level labels and transformation details is kept sealed until after the primary analysis to avoid experimenter bias.
\end{itemize}
This operationalizes the otherwise abstract notion of $d_\sigma$ in a way that can be compared directly to data.
