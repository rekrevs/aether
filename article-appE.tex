\section{Assumptions (Summary)}
\label{app:assumptions}

For clarity, we collect here the main structural assumptions used in the derivations of the modified Lieb–Robinson bound, the causality proof, and the phenomenological estimates.

\begin{itemize}
\item \textbf{(A1) Global ordering and strict retardation.} There exists a coarse-grained substrate time function $\tau$ that is strictly increasing along any physically allowed microscopic transition. All mediator kernels are retarded in $\tau$.

\item \textbf{(A2) Resource monotonicity.} The cost functor $\tilde{\mathcal K}:\mathcal{C}\to(\mathbb{R}_{\ge 0},+)$ is additive under composition, vanishes on identity morphisms, and assigns nonnegative cost to every nontrivial process. In particular, there are no negative-cost cycles that could compensate for time reversal.

\item \textbf{(A3) Sparse and weak $S$-links.} The substrate graph has bounded degree $g$ (each node participates in at most $g$ $S$-edges) and the total coupling strength attached to any node is bounded by a small parameter $\eta$:
\[
  \sum_{e\ni\sigma} |J_e| \le \eta \quad \forall \sigma \in S.
\]
This ensures convergence of the path sums leading to the function $\Phi$ in Eq.~(9.2) and prevents the substrate tail from becoming distance-independent.

\item \textbf{(A4) Structure of $O_S$ and Degeneracy Dilution.} The selection operator is normalized so that $[O_S]=4$. We define the quality factor $\mathcal{Q}[\rho] = |\mathrm{Tr}(\rho O_S)|/O_0$. For homogeneous, periodic, or thermal states with $N \gg 1$ equivalent pattern embeddings, we require:
\begin{equation}
  \mathcal{Q}[\rho_{hom}] \sim \frac{1}{N} \to 0
\end{equation}
This ensures the S-channel is suppressed in generic matter. In engineered platforms, the pattern is realized in $O(1)$ locations, allowing $\mathcal{Q} \sim 1$.

\item \textbf{(A5) Positivity and causality of the $S$-kernel.} The static kernel on $S$, $K_\sigma(\sigma,\sigma')$, is positive semidefinite and depends only on the structural distance $d_\sigma(\sigma,\sigma')$. The retarded mediator Green function in $S$ has finite speed $c_S$ and vanishes when $\tau'<\tau$.

\item \textbf{(A6) Minimal Lorentz breaking in gravity.} In the absence of $S$-mediated resonance ($J^\nu_\sigma=0$), gravitational waves propagate at light speed, $c_T=c$, and the aether/khronon sector parameters are chosen within the bounds summarized in Eq.~(3.1C).

\item \textbf{(A7) Momentum-neutrality.} The interaction action is translationally invariant in $M$, and the effective interaction stress--energy $T^{\mu\nu}_{\rm int}$ obeys
\[
  \partial_\mu T^{\mu i}_{\rm int} = -J^i_\sigma.
\]
Combined with the split conservation laws, this implies $\partial_\mu T^{\mu i}_{\rm tot}=0$ and, upon integration over space,
\[
  \frac{d}{dt}\,P^i_{\rm tot} = 0,
\]
so there is no reactionless drive even in the presence of time-dependent resonance.

\item \textbf{(A8) Universal gravitational coupling ($\alpha=1$).} The Einstein equations couple the metric to $T^{\mu\nu}_{\rm vis}+T^{\mu\nu}_S$ with a \emph{universal} strength. The Bianchi identity together with the split conservation laws enforce $\alpha\equiv 1$ when $J^\nu_\sigma\neq 0$, so there is no tunable reduced gravitational coupling for the $S$-sector. Apparent FTL effects arise entirely from substrate locality, not from modifications of the causal structure of $(M,g)$.

\item \textbf{(A9) Length dimensions for $d_\sigma$ and $\lambda_\sigma$.} The structural distance $d_\sigma$ and the range $\lambda_\sigma$ are assigned dimensions of length via an embedding scale $\ell_0$ set by the feature map $\mathcal{I}:S\to\mathbb{R}^{N_{\rm feat}}$. This allows direct comparison between $d_\sigma$, $\lambda_\sigma$, and experimental length scales $L_{\rm exp}$.

\item \textbf{(A10) Pattern Sequestering.} There exists a coarse-grained background field $Q_*(x)$ measuring pattern activation. The S-mediated interaction is always proportional to $Q_*^2(x)$. In the homogeneous/vacuum states defining SME bounds, $Q_* \equiv 0$. In the engineered configurations of E1-E3, $Q_* \sim \mathcal{Q}$. This ensures that S-induced Lorentz violation is absent from the vacuum EFT by construction.
\end{itemize}

\subsection{Method note}

For the proposed experimental tests (E1--E3), we assume:
\begin{itemize}
  \item full pre-registration of hypotheses, analysis pipelines, and stopping rules on a public repository (e.g.\ OSF);
  \item use of cryptographic commit--reveal schemes for codebooks and analysis choices that could otherwise be tuned post hoc;
  \item explicit multiple-test correction (e.g.\ Holm–Bonferroni or false discovery rate) across the family of pre-specified statistics.
\end{itemize}
A null result interpreted under assumptions (A1)--(A9) then yields clear, quantitative bounds on combinations of $(\varepsilon,\lambda_\sigma,Q,\omega_0)$; a non-null result would provide a well-defined target for replication and further scrutiny.
