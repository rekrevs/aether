\section{Thermodynamics and Measurable Cost}

We now connect the phenomenological power $\Psig$ to thermodynamic and information-theoretic constraints. The key idea is that any reliable transfer of information through the $S$-channel requires the production of entropy in the combined system, and that a finite pump power therefore bounds the achievable FTL bitrate.

\subsection{Pattern free energy}

We define an effective \textbf{pattern free energy} for the $S$-sector,
\begin{equation}
  \mathcal{F}_S
  =
  \langle E_S \rangle - \Temp\,\Sigma_S,
  \tag{8.1}
\end{equation}
where $\Sigma_S$ is an entropy-like functional of the microscopic $S$-state (e.g.\ approximated by minimum description length or compression ratio) and $\Temp$ is an effective temperature associated with the degrees of freedom that the pump can thermalize. In the spirit of Landauer's principle, we will require that each reliably distinguishable bit communicated through the $S$-channel carries a minimal entropy cost~\cite{landauer1961_irreversibility,berut2012_landauer_exp,jun2014_landauer_precision}.

\subsection{Minimal Markov model on S-edges}

We model each active $S$-edge $e$ as a two- or few-state Markov process driven by the pump. The edge switches between microstates $\{i\}$ with transition rates $k_{ij}(t)$ that depend on the external driving. Standard stochastic thermodynamics then associates to each realization:
\begin{itemize}
  \item a stochastic entropy production $\Delta S_{\mathrm{tot}}(e)$;
  \item a heat $Q_{\mathrm{diss}}(e)$ dumped into the environment;
  \item a Kullback--Leibler divergence $D_{\mathrm{KL}}$ between the forward and time-reversed trajectory ensembles.
\end{itemize}
Results due to Crooks, Seifert, and others imply a general inequality
\begin{equation}
  \langle \Delta S_{\mathrm{tot}} \rangle
  \;\ge\;
  k_B\,D_{\mathrm{KL}}(\mathrm{forward}\|\mathrm{backward}),
  \tag{8.2}
\end{equation}
with $\langle\cdot\rangle$ denoting an ensemble average. When the dynamics is used to transmit an average of $I_e$ bits of information along edge $e$, the KL divergence satisfies
\[
  D_{\mathrm{KL}} \;\ge\; I_e \ln 2,
\]
so we obtain a Landauer-type inequality
\begin{equation}
  \langle Q_{\mathrm{diss}}(e)\rangle
  \;\ge\;
  k_B \Temp \ln 2 \cdot I_e,
\end{equation}
i.e.\ at least $k_B \Temp \ln 2$ of dissipated heat per reliably transmitted bit.

\subsection{Resource inequality for FTL bitrate}

Let $R_{\mathrm{bit}}(e)$ be the average bit rate (bits per second) transmitted along the $S$-channel for edge $e$ in a stationary regime, and let $P_{\mathrm{diss}}(e)$ be the average rate of heat dissipation that can be used for signalling. From the inequality above we obtain
\begin{equation}
  R_{\mathrm{bit}}(e)
  \;\le\;
  \frac{P_{\mathrm{diss}}(e)}{k_B \Temp \ln 2}.
\end{equation}
Not all of the pump power is available for structured signalling. We therefore set
\begin{equation}
  P_{\mathrm{diss}}(e)
  =
  \beta\,
  \Psig(e),
\end{equation}
where $0<\beta\le 1$ collects inefficiencies in the conversion from pump work to low-entropy pattern changes that actually carry information (e.g.\ losses to uncontrolled degrees of freedom). Using Eq.~(6.1) for $\Psig(e)$, we obtain
\begin{equation}
  R_{\mathrm{bit}}(e)
  \;\le\;
  \frac{\beta}{k_B \Temp \ln 2}\,
  \varepsilon\,
  \Ksig(e)\,
  \mathcal{Q}(e,t)\,
  \Delta\tilde{\Phi}(e)\,
  P_{\mathrm{pump}}(e).
  \tag{8.3}
\end{equation}
\textbf{This is the central bound of the paper.} Equation~(8.3) is the thermodynamic \emph{resource inequality} that underpins all subsequent predictions and experimental design: for fixed pump power and temperature, the achievable FTL bit rate is bounded by the product of the microscopic parameters $(\varepsilon,\Ksig,\mathcal{Q},\tilde{\Delta\Phi})$. In particular, for a given experimental platform, a null result in a carefully designed protocol directly constrains combinations of these parameters.

\subsection{Bounding parameters from null results}

Suppose an experiment of type E1 yields a null result in the sense that no FTL signal is detected above a bit-rate threshold $R_{\mathrm{bit}}^{(\mathrm{null})}$ for a given edge and pump configuration. Taking $\tilde{\Delta\Phi}\le 1$ as a conservative assumption (i.e.\ the free-energy difference per operation does not exceed $k_B\Temp$), Eq.~(8.3) implies
\begin{equation}
  \varepsilon\,\mathcal{Q}\,e^{-d_\sigma/\lambda_\sigma}
  \;<\;
  \frac{k_B \Temp \ln 2}{\beta\,P_{\mathrm{pump}}}\,
  R_{\mathrm{bit}}^{(\mathrm{null})}.
  \tag{8.4}
\end{equation}
In a regime where $\varepsilon$ is independently constrained (for example by photon-sector tests or SME bounds), Eq.~(8.4) can be interpreted as a direct bound on the product $\mathcal{Q}\,e^{-d_\sigma/\lambda_\sigma}$, with $d_\sigma$ calibrated using the distance ladder of Sec.~7. Conversely, if an effect is observed and cross-checked against multiple protocols, Eq.~(8.3) provides a consistency relation between the inferred parameters.

Taken together with the energetic constraints from E2 (Sec.~12) and the anisotropy bounds from photon-sector tests, the thermodynamic inequality in Eq.~(8.3) helps carve out a small, sharply defined region of parameter space where aether resonance could reside.
