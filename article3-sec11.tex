\section{Compatibility with Established Tests and Anisotropy Budget}

\textbf{Relativity:} Lorentz symmetry emergent in low-energy sector; preferred frame gives weak anisotropy.

\subsection{Anisotropy Derivation}

From Lagrangian (5.1), a preferred frame $(\xi^\mu = (1,0,0,0)$ in substrate rest frame) gives modifications to dispersion:

\begin{equation}
    E^2 = p^2 c^2 + m^2 c^4\,\big[1+\Delta(E,\hat p\!\cdot\!\hat\xi)\big],
    \tag{11.1}
\end{equation}

where $\Delta$ is a dimensionless modifier:

\begin{equation}
    \Delta \;\sim\; \varepsilon\left(\frac{\lambda_\sigma}{\lambda_C}\right)\mathcal Q
    \left(\frac{E}{mc^2}\right)\,(\hat p\!\cdot\!\hat\xi)^2.
    \tag{11.2}
\end{equation}

For photons $(m=0)$, term rescaling gives effective velocity variation:

\begin{equation}
    \frac{\Delta c}{c} \;\sim\; \epsgam\left(\frac{\lsig}{\lambda_C}\right)\Qgam.
    \tag{11.3}
\end{equation}

where $\epsgam$ denotes the photon-sector coupling, $\lsig$ is the substrate coherence length, and $\Qgam$ is the photon-sector quality factor.

\textbf{Michelson–Morley/Hughes–Drever bounds:} $\Delta c/c \lesssim 10^{-18}$.

This requires:

\begin{equation}
    \epsgam\left(\frac{\lsig}{\lambda_C}\right)\Qgam \;\lesssim\; 10^{-18}.
    \tag{11.4}
\end{equation}

For $\lsig \sim 1\,\mu{\rm m}$, $\lambda_C \sim 10^{-12}\,{\rm m}$ (Compton) $\Rightarrow \lsig/\lambda_C \sim 10^6$, so $\epsgam\cdot \Qgam \lesssim 10^{-24}$.

\subsection{Daily/Annual Modulation and Quantitative Budget}

We tie sidereal/annual modulation to model parameters (see detailed phenomenology in §12, Eq. (12.4)). For the \textbf{matter sector}:

\[
A_{sid}^{(\rm mat)} \simeq \epsmat \left( \frac{\lsig}{L_{exp}} \right) \Qmat \, \Xi,
\]

where $\epsmat$ and $\Qmat$ denote matter-sector coupling and quality factor, $(\Xi)$ is a geometry/rig-dependent factor $(\sim 1)$, and $(L_{exp})$ is apparatus scale. This makes \textbf{anisotropy budget} measurable in own data.

\textbf{Amplitude:} $(\propto \epsmat\cdot \Qmat\cdot\cos \theta(t))$ and under $\Delta c/c$-bounds requires small magnitudes. Set \textbf{measurement goal} for E2-rotation as
\begin{equation}
A_{\rm sid}\gtrsim 10^{-20}\ \text{(3$\sigma$ on }10^7\text{ s)},
\end{equation}
which is numerically consistent with (11.4) and apparatus scale $L_{\rm exp}$.

A pre-registered null result sets a numerical upper bound on $(\epsmat \lsig \Qmat)$ for given apparatus, making ``anisotropy budget'' testable. Note that photon-sector bounds (11.4) and matter-sector measurements are distinct; see §12 and Appendix F for sector separation.

\begin{tcolorbox}[colback=yellow!5!white,colframe=orange!75!black,title=\textbf{SME Mapping: Dimensional Analysis Justification}]
\textbf{Setup:} After integrating out substrate degrees of freedom, the leading Lorentz-violating operator visible in spacetime $M$ has mass dimension $\Delta = 6$ (next-to-leading beyond Standard Model).

\textbf{Gauge and symmetry constraints:} For \emph{photon sector}, require:
\begin{itemize}[leftmargin=*,noitemsep,topsep=3pt]
\item Gauge invariance (electromagnetic $U(1)$)
\item Diffeomorphism invariance (general covariance)
\item Parity-even (CPT-preserving leading term)
\end{itemize}

\textbf{Operator structure:} Lowest-dimension CPT-even photon operator is dimension-6, schematically:
\[
\frac{1}{M_*^2} \, F_{\mu\nu}F^{\mu\rho}\, \xi^\nu \xi_\rho \quad \Rightarrow \quad \text{velocity shift} \propto (\hat p \cdot \hat \xi)^2.
\]

\textbf{Parameter identification:} Available dimensionless ratios are:
\begin{itemize}[leftmargin=*,noitemsep,topsep=3pt]
\item $\varepsilon$ (substrate coupling strength)
\item $\mathcal{Q}$ (resonance quality factor)
\item $\lambda_\sigma/\lambda_C$ (coherence length over Compton scale)
\end{itemize}

Dimensional consistency + symmetry $\Rightarrow$
\[
\Delta \sim \varepsilon \left(\frac{\lambda_\sigma}{\lambda_C}\right) \mathcal{Q} \, (\hat p \cdot \hat \xi)^2.
\]

\textbf{SME coefficient mapping:} In minimal SME frame (SCCEF), photon anisotropy parametrized by $\tilde{\kappa}_{e-}^{JK}$. Our model gives:
\[
\tilde{\kappa}_{e-}^{JK} \sim \varepsilon_\gamma \mathcal{Q}_\gamma \left(\frac{\lambda_\sigma}{L_{\rm exp}}\right) \Xi^{JK},
\]
where $\Xi^{JK} = \mathcal{O}(1)$ is geometry-dependent tensor encoding apparatus orientation.

\textbf{Key point:} This is \emph{not} a free fit parameter—it is determined by independently measurable quantities $(\varepsilon, \mathcal{Q}, \lambda_\sigma, L_{\rm exp})$ plus known geometry $\Xi^{JK}$.
\end{tcolorbox}

\subsection{SME Parametrization (For Reporting and Comparison)}

In the photon sector, anisotropies are conveniently expressed in minimal SME coefficients $\tilde\kappa_{e-}^{JK},\,\tilde\kappa_{o+}^{JK},\,\kappa_{\rm tr}$. Our model effectively gives
\begin{equation}
\tilde\kappa_{e-}^{JK}\ \sim\ \epsgam\,\Qgam\,
\Big(\frac{\lsig}{L_{\rm exp}}\Big)\,\Xi^{JK},
\end{equation}
with $\Xi^{JK}$ a geometry factor $(\mathcal O(1))$. Therefore report $\hat A_{\rm sid}$ in parallel with estimates of $|\tilde\kappa_{e-}^{JK}|$ to enable direct comparison with resonator and atomic clock studies.

