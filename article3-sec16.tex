\section{Conclusion}

We have presented a consistent, falsifiable hypothesis for \textbf{aether resonance} as structure-local FTL transfer in a discrete substrate. The framework unifies:

\begin{enumerate}
\item \textbf{Action-level} formulation (§3.1) with well-defined $(T^{\mu\nu}_S)$ via variation,
\item \textbf{Covariant energy-momentum accounting} (§3.2) with \textbf{$\alpha=1$} and light-speed causal metric response,
\item \textbf{$\alpha$-choice} (§3.5): $\alpha\equiv1$ for consistency with Bianchi identity,
\item \textbf{Localization of S-flows} (§3.3) as sources in M,
\item \textbf{Momentum-neutrality} (§3.4) preventing reactionless drive,
\item \textbf{Length units for $d_\sigma$ and $\lambda_\sigma$} (§2, §7) via embedding scale $(\ell_0)$,
\item \textbf{Local S-mediator implementation} (§4) of S-proximity,
\item \textbf{Selection operator} (§5) explaining absence in standard sector,
\item \textbf{Dimensionally correct coupling law} (§6) with consistent K/$\mathcal{K}$ notation,
\item \textbf{Operationalized $d_\sigma$-metric} (§7) and distance ladder,
\item \textbf{Thermodynamic resource bounds} (§8) with bitrate bound,
\item \textbf{Modified Lieb-Robinson bound with explicit conditions} (§9, Lemma 9.1) quantifying microcausality breaking,
\item \textbf{Formal causality proof} (§10) via category theory with consistent $\mathcal{K}$ notation,
\item \textbf{Anisotropy budget} (§11) with quantitative bounds,
\item \textbf{Numerical predictions} (§12) with three scenarios for E2 detectability,
\item \textbf{No-loophole experiments} (§13) with multiple-test correction and parameter inference.
\end{enumerate}

Either it leads to stringent upper bounds:
\begin{equation}
\varepsilon \lambda_\sigma \mathcal{Q} < 10^{-12} \, \text{m}, \quad
\varepsilon \omega_0 \mathcal{Q} < 10^{-8} \, \text{Hz}, \quad
\varepsilon \mathcal{Q} < 10^{-20}
\end{equation}

or it opens reproducible non-local effects. Both outcomes are scientifically informative and guide further work toward a more complete substrate theory.

