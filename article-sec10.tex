\section{Causality – Formal Proof}

The causality results in this section do not derive a global time function from more primitive ingredients; rather, they formalize the consequences of Postulate~P1 and Assumption~A1, which posit the existence of a substrate time function $\tau$ that is strictly increasing along every allowed microscopic transition. Our contribution is to show that, given this structure, the presence of $S$-mediated superluminal transfers in $M$ still forbids antitelephone-type paradoxes and closed causal loops.

In this section we make precise what is meant by ``causal'' in the presence of substrate-local FTL transfer and prove that the framework excludes closed causal loops and antitelephone constructions.

\subsection{Monotonic substrate time}

The substrate dynamics comes with a global discrete update parameter $T=0,1,2,\dots$ and a coarse-grained \emph{substrate time function} $\tau$ which is strictly monotonically increasing along any physically allowed microscopic transition. We summarize this as (Assumption~A1 in Appendix~\ref{app:assumptions}):

\textbf{Monotonic-$(T)$ axiom:} For every elementary substrate transition
\[
  (s,T)\longrightarrow (s',T')
\]
with nonzero amplitude, the associated substrate time increment satisfies
\begin{equation}
  \Delta \tau := \tau(s',T') - \tau(s,T) > 0,
  \qquad
  \text{all kernels are retarded in $T$}.
  \tag{10.1}
\end{equation}
In particular, the mediator kernel $G_{\rm ret}$ in $S$ (Sec.~4) vanishes for $T'<T$, and all effective kernels in $M$ obtained by integrating out $S$ inherit this retardation.

\subsection{Events as objects in a causal category}

We formalize the causal structure on $M\times S$ using the language of categories, following standard approaches in algebraic quantum field theory.

\begin{itemize}
\item \textbf{Objects:} Localized events $e$, i.e.~equivalence classes of microscopic configurations supported in small regions of $M\times S$ together with a finite interval of substrate time $\tau$.
\item \textbf{Morphisms:} A morphism $f:e\to e'$ is a physically allowed influence chain produced by composing:
  \begin{enumerate}
    \item local evolution in $M$ generated by the visible-sector Hamiltonian $H_M$;
    \item substrate-mediated hops through $S$ via the $S$-mediator and selection operator $O_S$;
    \item classical communication constrained to future light cones in $(M,g)$.
  \end{enumerate}
\end{itemize}

Composition of morphisms is concatenation of influence chains. By construction, every elementary piece of a morphism is retarded in substrate time and respects the Monotonic-$(T)$ axiom.

Define a functor
\[
  \mathcal{T}:\mathcal{C}\to(\mathbb{R},\le)
\]
from the causal category $\mathcal{C}$ to the totally ordered set of real numbers, by assigning to each event $e$ the minimal substrate time $\tau(e)$ at which it can occur and to each morphism $f:e\to e'$ the increment
\[
  \mathcal{T}(f)=\tau(e')-\tau(e).
\]
By the Monotonic-$(T)$ axiom, $\mathcal{T}(f)>0$ for every non-identity morphism $f$.

\begin{theorem}[Causal monotonicity]
Under the assumptions that (i) all microscopic rules are retarded in substrate time and satisfy $\Delta\tau>0$ as in Eq.~\textup{(10.1)}, (ii) the $S$-mediator has finite propagation speed $c_S$ and exponential damping in $d_\sigma$ (Sec.~9), and (iii) every macroscopic process is built from a finite concatenation of such microscopic transitions, there exist no closed causal loops in $(M\times S)$.
\end{theorem}

\begin{proof}[Sketch]
Suppose for contradiction that there exists a nontrivial closed loop in $\mathcal{C}$, i.e.~a morphism $f:e\to e$ which is not the identity. Functoriality of $\mathcal{T}$ implies
\[
  \mathcal{T}(f\circ f) = \mathcal{T}(f) + \mathcal{T}(f).
\]
Iterating, for $n$ copies we obtain $\mathcal{T}(f^{\circ n}) = n\,\mathcal{T}(f)$. On the other hand, by the Monotonic-$(T)$ axiom every non-identity morphism has strictly positive substrate time length, so $\mathcal{T}(f)>0$. This is incompatible with $f$ being a loop based at $e$, because by definition a causal loop must return to the same event at the same substrate time, which would require $\mathcal{T}(f)=0$. Hence no such nontrivial loop exists. \qedhere
\end{proof}

The technical assumptions (ii) and (iii) ensure that the effective evolution on $M$ inherits a generalized Lieb--Robinson bound with a soft cone (Sec.~9) and that the category $\mathcal{C}$ is well-defined.

\subsection{Anti-telephone rule}

We now formulate the operational rule that forbids antitelephone constructions.

\textbf{Rule (anti-telephone):} A substrate-mediated transfer between two laboratories $A$ and $B$ is only permitted if each microscopic resonance step in the chain satisfies
\begin{equation}
  \Delta\tau > 0
\end{equation}
in the substrate frame. Chains that would require $\Delta\tau<0$ for any microscopic step are assigned zero amplitude.

Importantly, requiring local $\Delta\tau>0$ for each microscopic resonance step is \emph{not} a fine-tuned prohibition imposed by hand; rather, it is simply a restatement of the fact that the microscopic substrate rules define an absolute substrate time ordering in which updates occur sequentially. This is a structural property of the substrate dynamics itself.

This rule is local in $(M\times S)$: it is enforced at the level of individual resonance events and does not depend on the global motion of the laboratories. Because $\tau$ is a scalar time function on the substrate, it is the same for all inertial observers in $M$.
This local rule is compatible with the cost and sparsity assumptions
(A2)--(A3) of Appendix~\ref{app:assumptions}, which ensure that any physically realized
influence chain carries a positive substrate time length and a
nonzero thermodynamic cost.

\begin{corollary}[Two-lab antitelephone exclusion]
Consider two laboratories $A$ and $B$ following arbitrary timelike worldlines in $(M,g)$ with four-velocities $U_A^\mu$ and $U_B^\mu$. Assume each laboratory can use a combination of subluminal signals in $M$ and substrate-local resonance steps in $S$. Under the Monotonic-$(T)$ axiom and the anti-telephone rule above, there exists no protocol by which $A$ and $B$ can arrange for a message to arrive at the sender's own past proper time.
\end{corollary}

\begin{proof}[Sketch]
Any protocol can be decomposed into a finite sequence of morphisms in $\mathcal{C}$ corresponding to local operations at $A$ or $B$, subluminal communications in $M$, and resonance steps in $S$. The functor $\mathcal{T}$ assigns to each such morphism a strictly positive time increment, because local operations and subluminal signals are future-directed in $(M,g)$ and resonance steps obey $\Delta\tau>0$. Therefore the total substrate time elapsed along the protocol is strictly positive as seen by both laboratories, and no closed causal loop in the sense of proper time is possible. \qedhere
\end{proof}

\subsection{Relation to the Lieb–Robinson bound}

Section~9 established a modified Lieb--Robinson inequality, Eq.~(9.1), which bounds the norm of commutators of local observables outside an effective light cone and shows that substrate-mediated effects produce only a soft tail exponentially suppressed in $d_\sigma$ and retarded by $c_S$ in $S$. Combining that bound with the categorical argument above yields:

\begin{itemize}
\item No exact, instantaneous commutators at spacelike separation in $M$; the commutator norm remains exponentially small outside the soft cone.
\item No possibility of closed causal curves in $(M\times S)$, because every influence chain has strictly positive $\Delta\tau$.
\end{itemize}

\textbf{Conclusion.} The model permits FTL transfer in the emergent spacetime $(M,g)$ while maintaining a globally well-defined causal order in substrate time. The combination of the Monotonic-$(T)$ axiom, the anti-telephone rule, and the modified Lieb--Robinson bound rules out antitelephone paradoxes.

\subsection{Relation to standard no-signalling theorems}
\label{subsec:no-signalling}

It is useful to compare the present framework with standard
no--signalling and microcausality results in relativistic quantum
field theory and quantum information (see, e.g., the review in Ref.~\cite{brunner2014_bell_nonlocality}).

First, relativistic QFT assumes that the fundamental fields are local
on $(M,g)$ and that the interaction Hamiltonian is an integral of
local densities.  Under these assumptions one can impose exact
microcausality,
\[
[\hat\phi(x),\hat\phi(y)] = 0 \qquad \text{for spacelike-separated } x,y,
\]
which in turn implies that no influence can propagate faster than the
speed of light in~$M$.  By contrast, the aether-resonance interaction
is explicitly \emph{bilocal} in the emergent spacetime: it is local in
the substrate variables $(\sigma,T)$ and in the pattern-space metric
$d_\sigma$, but can connect points in $M$ that are spacelike-separated
with respect to~$g_{\mu\nu}$.  As a result, exact microcausality in $M$
is not expected.  Instead, Sec.~9 derives a modified Lieb--Robinson
bound: commutators outside the usual light cone are nonzero but are
exponentially suppressed in $d_\sigma$ and retarded at the finite
substrate speed~$c_S$.

Second, Eberhard-type arguments show that introducing superluminal
signals with a \emph{Lorentz-invariant} speed generically leads to
paradoxes or to frame-dependent collapse rules.  The present model
evades these conclusions by postulating an explicit preferred substrate
frame and a global substrate time $\tau$.  The dynamics is not
Lorentz-invariant at the microscopic level; instead, Lorentz symmetry
emerges in the long-wavelength limit of the visible sector.  The
Monotonic-$(T)$ axiom and the anti-telephone rule ensure that every
allowed influence chain has strictly positive $\Delta\tau$, so that
closed causal loops in $(M\times S)$ are excluded even though
individual segments may project to superluminal displacements in~$M$.

Third, standard ``no superluminal signalling from entanglement''
results in quantum information~\cite{brunner2014_bell_nonlocality} rely on the linearity of quantum
mechanics, the Born rule, and the assumption that measurement choices
act only on local degrees of freedom in~$M$.  In our framework we do
not modify the Hilbert-space structure or the Born rule of the visible
sector.  Instead, we posit an additional channel mediated by the
$S$-sector, which couples to pattern-dependent operators $O_S$ and is
governed by the resource and causality constraints derived above.
Entanglement alone cannot be used to signal in this model; any FTL
influence must flow through the substrate channel and therefore obey
the constraints of the modified Lieb--Robinson bound and the
Monotonic-$(T)$ axiom.

An analogous situation occurs in fast-light optical media, where group velocities can exceed $c$ but carefully defined information velocities remain subluminal~\cite{stenner2003_fast_light}. Here, however, the true causal order is defined by the substrate time $\tau$, and the framework allows genuine superluminal signal fronts in $M$ provided they remain $\tau$-monotone and obey the bounds of Sec.~9.

In summary, the framework is explicitly outside the scope of
standard local QFT no-go theorems (because of bilocal couplings and a
preferred substrate frame), but it replaces exact microcausality in
$M$ by a softer, quantitatively controlled notion of causality in
$(M\times S)$ that still forbids antitelephone protocols.
