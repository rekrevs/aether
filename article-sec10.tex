\section{Causality – Formal Proof}

In this section we make precise what is meant by ``causal'' in the presence of substrate-local FTL transfer and prove that the framework excludes closed causal loops and antitelephone constructions.

\subsection{Monotonic substrate time}

The substrate dynamics comes with a global discrete update parameter $T=0,1,2,\dots$ and a coarse-grained \emph{substrate time function} $\tau$ which is strictly monotonically increasing along any physically allowed microscopic transition. We summarize this as:

\textbf{Monotonic-$(T)$ axiom:} For every elementary substrate transition
\[
  (s,T)\longrightarrow (s',T')
\]
with nonzero amplitude, the associated substrate time increment satisfies
\begin{equation}
  \Delta \tau := \tau(s',T') - \tau(s,T) > 0,
  \qquad
  \text{all kernels are retarded in $T$}.
  \tag{10.1}
\end{equation}
In particular, the mediator kernel $G_{\rm ret}$ in $S$ (Sec.~4) vanishes for $T'<T$, and all effective kernels in $M$ obtained by integrating out $S$ inherit this retardation.

\subsection{Events as objects in a causal category}

We formalize the causal structure on $M\times S$ using the language of categories, following standard approaches in algebraic quantum field theory.

\begin{itemize}
\item \textbf{Objects:} Localized events $e$, i.e.~equivalence classes of microscopic configurations supported in small regions of $M\times S$ together with a finite interval of substrate time $\tau$.
\item \textbf{Morphisms:} A morphism $f:e\to e'$ is a physically allowed influence chain produced by composing:
  \begin{enumerate}
    \item local evolution in $M$ generated by the visible-sector Hamiltonian $H_M$;
    \item substrate-mediated hops through $S$ via the $S$-mediator and selection operator $O_S$;
    \item classical communication constrained to future light cones in $(M,g)$.
  \end{enumerate}
\end{itemize}

Composition of morphisms is concatenation of influence chains. By construction, every elementary piece of a morphism is retarded in substrate time and respects the Monotonic-$(T)$ axiom.

Define a functor
\[
  \mathcal{T}:\mathcal{C}\to(\mathbb{R},\le)
\]
from the causal category $\mathcal{C}$ to the totally ordered set of real numbers, by assigning to each event $e$ the minimal substrate time $\tau(e)$ at which it can occur and to each morphism $f:e\to e'$ the increment
\[
  \mathcal{T}(f)=\tau(e')-\tau(e).
\]
By the Monotonic-$(T)$ axiom, $\mathcal{T}(f)>0$ for every non-identity morphism $f$.

\begin{theorem}[Causal monotonicity]
Under the assumptions that (i) all microscopic rules are retarded in substrate time and satisfy $\Delta\tau>0$ as in Eq.~\textup{(10.1)}, (ii) the $S$-mediator has finite propagation speed $c_S$ and exponential damping in $d_\sigma$ (Sec.~9), and (iii) every macroscopic process is built from a finite concatenation of such microscopic transitions, there exist no closed causal loops in $(M\times S)$.
\end{theorem}

\begin{proof}[Sketch]
Suppose for contradiction that there exists a nontrivial closed loop in $\mathcal{C}$, i.e.~a morphism $f:e\to e$ which is not the identity. Functoriality of $\mathcal{T}$ implies
\[
  \mathcal{T}(f\circ f) = \mathcal{T}(f) + \mathcal{T}(f).
\]
Iterating, for $n$ copies we obtain $\mathcal{T}(f^{\circ n}) = n\,\mathcal{T}(f)$. On the other hand, by the Monotonic-$(T)$ axiom every non-identity morphism has strictly positive substrate time length, so $\mathcal{T}(f)>0$. This is incompatible with $f$ being a loop based at $e$, because by definition a causal loop must return to the same event at the same substrate time, which would require $\mathcal{T}(f)=0$. Hence no such nontrivial loop exists. \qedhere
\end{proof}

The technical assumptions (ii) and (iii) ensure that the effective evolution on $M$ inherits a generalized Lieb--Robinson bound with a soft cone (Sec.~9) and that the category $\mathcal{C}$ is well-defined.

\subsection{Anti-telephone rule}

We now formulate the operational rule that forbids antitelephone constructions.

\textbf{Rule (anti-telephone):} A substrate-mediated transfer between two laboratories $A$ and $B$ is only permitted if each microscopic resonance step in the chain satisfies
\begin{equation}
  \Delta\tau > 0
\end{equation}
in the substrate frame. Chains that would require $\Delta\tau<0$ for any microscopic step are assigned zero amplitude.

This rule is local in $(M\times S)$: it is enforced at the level of individual resonance events and does not depend on the global motion of the laboratories. Because $\tau$ is a scalar time function on the substrate, it is the same for all inertial observers in $M$.

\begin{corollary}[Two-lab antitelephone exclusion]
Consider two laboratories $A$ and $B$ following arbitrary timelike worldlines in $(M,g)$ with four-velocities $U_A^\mu$ and $U_B^\mu$. Assume each laboratory can use a combination of subluminal signals in $M$ and substrate-local resonance steps in $S$. Under the Monotonic-$(T)$ axiom and the anti-telephone rule above, there exists no protocol by which $A$ and $B$ can arrange for a message to arrive at the sender's own past proper time.
\end{corollary}

\begin{proof}[Sketch]
Any protocol can be decomposed into a finite sequence of morphisms in $\mathcal{C}$ corresponding to local operations at $A$ or $B$, subluminal communications in $M$, and resonance steps in $S$. The functor $\mathcal{T}$ assigns to each such morphism a strictly positive time increment, because local operations and subluminal signals are future-directed in $(M,g)$ and resonance steps obey $\Delta\tau>0$. Therefore the total substrate time elapsed along the protocol is strictly positive as seen by both laboratories, and no closed causal loop in the sense of proper time is possible. \qedhere
\end{proof}

\subsection{Relation to the Lieb–Robinson bound}

Section~9 established a modified Lieb--Robinson inequality, Eq.~(9.1), which bounds the norm of commutators of local observables outside an effective light cone and shows that substrate-mediated effects produce only a soft tail exponentially suppressed in $d_\sigma$ and retarded by $c_S$ in $S$. Combining that bound with the categorical argument above yields:

\begin{itemize}
\item No exact, instantaneous commutators at spacelike separation in $M$; the commutator norm remains exponentially small outside the soft cone.
\item No possibility of closed causal curves in $(M\times S)$, because every influence chain has strictly positive $\Delta\tau$.
\end{itemize}

\textbf{Conclusion.} The model permits FTL transfer in the emergent spacetime $(M,g)$ while maintaining a globally well-defined causal order in substrate time. The combination of the Monotonic-$(T)$ axiom, the anti-telephone rule, and the modified Lieb--Robinson bound rules out antitelephone paradoxes.
