\section{Category-Theoretic Causality Proof – Full Version}

This appendix expands the categorical formulation sketched in Sec.~10. The goal is to make precise how a global substrate time and strictly retarded rules forbid closed causal loops and antitelephone protocols, even in the presence of substrate-local FTL.

\subsection*{D.1 Causal category}

\textbf{Category $\mathcal{C}$.}
\begin{itemize}
  \item \textbf{Objects.} Localized events $e$, i.e.\ equivalence classes of microscopic configurations supported in small regions of $M\times S$ over a finite interval of substrate time $\tau$. Concretely, an object may be represented as a triple
  \[
    e = \big(U_e,\Sigma_e,[\tau_e,\tau_e+\delta\tau_e]\big),
  \]
  where $U_e\subset M$ is open, $\Sigma_e\subset S$ is a bounded region in pattern space, and $[\tau_e,\tau_e+\delta\tau_e]$ is a closed interval in substrate time.
  \item \textbf{Morphisms.} A morphism $f:e\to e'$ is a physically allowed influence chain built from:
    \begin{enumerate}
      \item local evolution in $M$ generated by the visible-sector Hamiltonian $H_M$ and confined to future light cones in $(M,g)$;
      \item substrate-mediated hops in $S$ via the retarded $S$-mediator and selection operator $O_S$;
      \item classical communication constrained to future light cones in $(M,g)$.
    \end{enumerate}
    Composition of morphisms is concatenation of influence chains.
\end{itemize}

We assume that any macroscopic protocol (including attempted antitelephones) can be decomposed into a finite sequence of such morphisms.

\subsection*{D.2 Time functor and monotonicity}

\textbf{Monotonic-$\tau$ axiom.} Each microscopic transition in the substrate,
\[
  (s,T)\longrightarrow (s',T'),
\]
satisfies
\begin{equation}
  \tau(s',T') - \tau(s,T) > 0,
  \tag{D.1}
\end{equation}
where $\tau$ is a coarse-grained substrate time function, strictly increasing along all physically allowed transitions. In particular, the $S$-mediator Green function is retarded in $\tau$ and vanishes for $\tau'<\tau$.

\textbf{Time functor.} Define a functor
\[
  \mathcal{T}:\mathcal{C} \to (\mathbb{R},\le)
\]
by
\begin{itemize}
  \item assigning to each event $e$ the minimal substrate time $\tau(e)$ at which it can occur, and
  \item assigning to each morphism $f:e\to e'$ the increment
  \[
    \mathcal{T}(f) := \tau(e') - \tau(e).
  \]
\end{itemize}
By the monotonic-$\tau$ axiom and the locality of the microscopic rules, every non-identity morphism satisfies
\begin{equation}
  \mathcal{T}(f) > 0.
  \tag{D.2}
\end{equation}

\subsection*{D.3 Cost functor and resource monotonicity}

Beyond temporal order, we introduce a \emph{cost functor} summarizing the thermodynamic resource usage of a process.

\textbf{Cost functor.} Let
\[
  \tilde{\mathcal{K}}:\mathcal{C}\to (\mathbb{R}_{\ge 0},+)
\]
be a functor assigning to each morphism $f:e\to e'$ a nonnegative real number $\tilde{\mathcal{K}}(f)$, interpreted as the minimal pump cost or entropy production required to implement $f$. The defining properties are:
\begin{itemize}
  \item $\tilde{\mathcal{K}}(\mathrm{id}_e)=0$ for all objects $e$,
  \item $\tilde{\mathcal{K}}(g\circ f) = \tilde{\mathcal{K}}(f) + \tilde{\mathcal{K}}(g)$ for composable morphisms,
  \item $\tilde{\mathcal{K}}(f)\ge 0$ for all morphisms $f$ (no negative cost).
\end{itemize}
Assumption (A2) in Appendix~E guarantees resource monotonicity: nontrivial influence chains cannot be implemented at zero cost.

\subsection*{D.4 No closed causal loops}

\begin{theorem}[No nontrivial loops in $\mathcal{C}$]
Under the monotonic-$\tau$ axiom and the definition of $\mathcal{T}$ above, there exists no nontrivial morphism $f:e\to e$ with $\mathcal{T}(f)=0$.
\end{theorem}

\begin{proof}
Assume for contradiction that there exists a non-identity $f:e\to e$ with $\mathcal{T}(f)=0$. Consider the $n$-fold composition $f^{\circ n}=f\circ f\circ\cdots\circ f$. Functoriality of $\mathcal{T}$ implies
\[
  \mathcal{T}(f^{\circ n}) = n\,\mathcal{T}(f) = 0
\]
for all $n\in\mathbb{N}$. On the other hand, $f$ is a composition of microscopic steps, each with positive $\Delta\tau$ by Eq.~(D.1), so
\[
  \mathcal{T}(f) > 0,
\]
and hence $\mathcal{T}(f^{\circ n})=n\,\mathcal{T}(f) > 0$ for all $n$. This contradiction shows that no such nontrivial loop can exist. \qedhere
\end{proof}

Thus, even though the projection of a morphism to $M$ can have segments that look superluminal with respect to the light cones of $(M,g)$, the global substrate time functor $\mathcal{T}$ provides a strict temporal ordering that rules out closed causal curves in $(M\times S)$.

\subsection*{D.5 Anti-telephone rule and two-lab protocols}

We now state the operational anti-telephone rule and recast it categorically.

\textbf{Anti-telephone rule.} Any microscopic resonance step contributing to a morphism $f:e\to e'$ must satisfy $\Delta\tau>0$. Attempts to construct a process in which a signal returns to the sender's own past proper time necessarily require a morphism whose projection to $M$ forms a closed loop in the usual sense, but whose lift to $(M\times S)$ would have to violate $\Delta\tau>0$ at some step. Such morphisms are assigned zero amplitude.

\begin{corollary}[Two-lab antitelephone exclusion]
Consider two laboratories $A$ and $B$ following arbitrary timelike worldlines in $(M,g)$, exchanging messages via any combination of:
\begin{itemize}
  \item local operations at $A$ and $B$,
  \item subluminal signals in $M$,
  \item substrate-local resonance steps in $S$.
\end{itemize}
Then the induced process in $\mathcal{C}$ has strictly positive $\mathcal{T}(f)$ and cannot form a loop that returns to the sender's past proper time.
\end{corollary}

\begin{proof}[Sketch]
Every elementary move in such a protocol is either:
\begin{itemize}
  \item a local operation or subluminal signal in $M$, which is future-directed in $(M,g)$ and hence non-decreasing in $\tau$ when lifted to $(M\times S)$, or
  \item a resonance step in $S$ satisfying $\Delta\tau>0$ by the monotonic-$\tau$ axiom.
\end{itemize}
Therefore the total substrate time increment along the protocol is strictly positive, $\mathcal{T}(f)>0$, and the previous theorem excludes $f:e\to e$ with $\mathcal{T}(f)=0$. \qedhere
\end{proof}

\subsection*{D.6 Relation to the modified Lieb–Robinson bound}

The categorical argument above uses only the global time ordering and the local form of the rules. The modified Lieb–Robinson bound of Sec.~9 refines this by quantifying the size of commutators outside the emergent light cone:
\begin{itemize}
  \item the first term in Eq.~(9.1) reproduces the usual light-cone behaviour with effective velocity $v$ close to $c$;
  \item the second term is a substrate-mediated tail, retarded by $c_S$ in $S$ and exponentially suppressed in $d_\sigma$.
\end{itemize}
Together with the monotonic-$\tau$ axiom, this guarantees both:
\begin{enumerate}
  \item no exact instantaneous commutators at spacelike separation in $M$, and
  \item no closed causal loops or antitelephones when protocols are viewed in the causal category $\mathcal{C}$.
\end{enumerate}
