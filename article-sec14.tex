\section{Limitations, Open Questions, and Future Work}

\textbf{Limitations.}

\begin{itemize}
\item \textbf{Emergent continuum not derived.}
Perhaps the most important conceptual limitation is that we do not derive the existence of an
approximately continuous spacetime manifold $(M,g)$ from a specific discrete substrate model.
The framework is constructed ``one level down'' from whatever microphysics is responsible for
continuum emergence, and simply assumes that such an emergent Lorentzian background exists
and is described at low energies by standard local EFT. Any eventual microscopic model claiming
to realize the assumptions of Appendix~\ref{app:assumptions} would have to explain in detail why
this is the case; here we only analyse the consequences of coupling such a background to an
additional pattern space $S$.

\item \textbf{$\alpha$ fixed to unity.} We set $\alpha\equiv 1$ in the Einstein equations, and Sec.~\ref{subsec:alpha-constraint} shows that this is not a free choice but a consistency requirement imposed by the Bianchi identity in the presence of the exchange current $J^\nu_\sigma$. This places the entire FTL mechanism in the redistribution of stress--energy via the $S$-sector: there is no modified gravitational coupling to tune. While clean conceptually, this also means that any attempt to explain small effects must work through the dynamics of $T^S_{\mu\nu}$ rather than through a small $\alpha$.
\item \textbf{S-mediator implementation.} The existence of a dynamical mediator field $\chi(\sigma,T)$ on $S$ with finite propagation speed and exponential kernel decay is a structural assumption. We have given a consistent effective-field-theory realization in Sec.~4, but we do not yet provide a concrete microscopic substrate model (e.g.\ a specific cellular automaton or graph-rewriting rule set) that realizes such a mediator.
\item \textbf{$Q$-factor in practice.} The phenomenology relies on the existence of engineered systems with pattern quality factors $\mathcal Q$ approaching $10^{-2}$--$10^{-1}$ over macroscopic time scales. It is currently unknown whether such high, stable $\mathcal Q$ can be achieved in realistic platforms (e.g.\ Josephson junction networks, photonic crystals near band edge, or Bose--Einstein condensates near phase separation) without introducing uncontrolled noise or decoherence. There is also a genuine tension here: the same class of operators $O_S$ that must strongly suppress signals in homogeneous and thermal states (via $Q\sim 1/N\to 0$) must also allow $Q\sim 10^{-2}\text{--}10^{-1}$ in carefully prepared ``islands'' of structure. One of the central roles of explicit substrate models and experimental platform studies is to determine whether this tension can be resolved without fine tuning; if it cannot, large parts of the parameter space considered in Sec.~12 would be ruled out independently of direct FTL tests.
\item \textbf{Naturalness of small couplings.} While the $Q_*^2$ sequestering mechanism (Sec.~5.6) protects the vacuum from radiative instability, it enters the theory as a phenomenological assumption about the coupling structure ("Pattern Parity") rather than a derived property of a specific microscopic substrate. Deriving this spurion structure from first principles remains an open challenge for UV completion.
\item \textbf{UV analyticity and positivity.} Because the effective interaction is bilocal in time after integrating out the mediator, the usual dispersion-relation and positivity arguments for local Lorentz-invariant EFTs do not directly apply. In this work we only require that the retarded kernel $\mathcal{K}_{\rm eff}$ remain positive and causal below a cutoff $\Lambda_S$; we do not attempt to construct a full S-matrix or to prove positivity bounds. Any UV-complete substrate model would need to address analyticity and unitarity of high-energy scattering explicitly.
\end{itemize}

\textbf{Open questions.}

\begin{enumerate}
\item \textbf{Substrate specification.} Which explicit rule sets for the discrete substrate (cellular automaton, hypergraph rewriting, tensor-network update rules, \emph{etc.}) simultaneously yield:
   \begin{itemize}
   \item emergent Lorentz symmetry at low energies in $M$,
   \item a local selection operator $O_S$ built from higher-dimension ($\Delta>4$) operators as in Sec.~5,
   \item an $S$-sector whose stress--energy $T^S_{\mu\nu}$ is generically small and only becomes appreciable in rare, near-critical, pattern-rich configurations?
   \end{itemize}
   Constructing even a toy model with these properties would strongly sharpen the proposal.
\item \textbf{$Q$-platforms.} Can we experimentally map realistic ranges of $\mathcal Q$ in specific platforms, such as:
   \begin{itemize}
   \item Josephson-junction arrays near a superradiant transition,
   \item photonic crystals tuned near a band edge,
   \item Bose--Einstein condensates near phase separation or other critical points?
   \end{itemize}
   Systematic studies of pattern complexity and long-lived coherence in such systems are valuable independently of aether resonance and would test key assumptions of the model.
\item \textbf{Entropy and algorithmic information.} We have used a coarse-grained pattern entropy $\Sigma_S$ and argued that minimum-description-length (MDL)–type proxies can stand in for true algorithmic complexity. A more microscopic treatment would model explicitly how $\Sigma_S$ couples to a physical heat bath and quantify when MDL-like measures faithfully track the relevant information-theoretic cost. Is full-blown algorithmic information theory required, or are simpler statistical surrogates sufficient for all observable consequences?
\item \textbf{Coupling to emergent gravitation.} In an emergent-gravity scenario, the metric $g_{\mu\nu}$ and Einstein tensor $G_{\mu\nu}$ arise from coarse-graining microscopic substrate degrees of freedom. How, in such a picture, does the $S$-sector stress--energy $T^S_{\mu\nu}$ emerge, and can its smallness in generic states be derived from the same mechanism that generates $G_{\mu\nu}$? In other words, can the rarity of large $T^S_{\mu\nu}$ configurations be explained from first principles rather than imposed phenomenologically?
\end{enumerate}

\textbf{Future work.}

\begin{itemize}
\item \textbf{Explicit substrate models.} Construct toy models (e.g.\ in 2D/3D cellular automata or graph-based systems) that exhibit emergent Lorentz symmetry, a well-defined $S$-mediator, and a nontrivial pattern-selection operator $O_S$. Even partial success would clarify which ingredients are essential and which are artefacts of our continuum description.
\item \textbf{Material screening.} Identify and characterize candidate experimental platforms with high, controllable $\mathcal Q$ and well-understood thermodynamics. This includes measuring $d_\sigma$-like distances experimentally (via distance ladders) and quantifying how $\mathcal Q$ decays under realistic noise.
\item \textbf{Pilot E1/E2 tests.} Carry out low-cost proof-of-concept experiments, such as electrical RC networks for E1 or small-scale rotating calorimetric setups for E2, to validate analysis pipelines, environmental veto strategies, and distance-ladder calibration, even if their raw sensitivity is insufficient to test the most interesting parameter ranges.
\end{itemize}
