\section{Operationalizing Structural Distance}

\begin{quote}
\textbf{Core Concept:} ``Algorithmic similarity'' sounds abstract. We need a practical way to measure $d_\sigma$.
\end{quote}

\subsection{The Challenge}

\textbf{Algorithmic similarity} (Kolmogorov complexity) is:
\begin{enumerate}
\item Uncomputable (proven by Turing, Chaitin)
\item Unambiguously defined mathematically
\item Useless for experiments
\end{enumerate}

We need a \textbf{computable proxy} that captures the essence.

\subsection{Signature Extraction}

From a system's dynamics $s$ (time series, spatial configuration, etc.), extract features:

\textbf{Spectral signature:}
\begin{itemize}
\item FFT $\to$ power spectrum $S(f)$
\item Dominant frequencies: $f_1, f_2, \ldots, f_m$
\item Spectral entropy: $H_{spec} = -\int (S/\int S) \log(S/\int S) df$
\end{itemize}

\emph{What it captures:} Rhythms, periodicities, characteristic timescales

\textbf{Topological signature:}
\begin{itemize}
\item Persistent homology $\to$ Betti numbers $\beta_0(r), \beta_1(r), \ldots$
\item $\beta_0$: connected components
\item $\beta_1$: loops/holes
\item $\beta_2$: voids
\end{itemize}

\emph{What it captures:} Global shape and structure independent of coordinates

\textbf{Statistical signature:}
\begin{itemize}
\item Autocorrelation: $C(\tau) = \langle s(t)s(t+\tau) \rangle$
\item Correlation time: $\tau_c$ where $C$ falls to $1/e$
\item Lyapunov exponents: $\lambda_i$ (for chaotic systems)
\item Moments: $\mu_2, \mu_3, \mu_4$ (variance, skewness, kurtosis)
\end{itemize}

\emph{What it captures:} Memory, predictability, statistical regularities

\textbf{Combined signature:}\\
$\sigma(s) = (f_1, \ldots, f_m, H_{spec}, \beta_0, \beta_1, \ldots, \tau_c, \lambda_1, \ldots, \mu_2, \mu_3, \mu_4) \in \mathbb{R}^d$

This is a point in a $d$-dimensional feature space.

\subsection{The Metric Definition}

\begin{equation}
d_\sigma(s,s') := \ell_0 \left[ \|\sigma(s) - \sigma(s')\|_2 + \alpha_W \, W(\mu_s, \mu_{s'}) \right],
\tag{7.1}
\end{equation}

\textbf{Components:}

\textbf{$\|\sigma(s) - \sigma(s')\|_2$:} Euclidean distance in feature space
\begin{itemize}
\item Standard L2 norm: $\sqrt{\sum_i(\sigma_i - \sigma_i')^2}$
\item Measures ``how different are the features?''
\end{itemize}

\textbf{$W(\mu_s, \mu_{s'})$:} Wasserstein distance between probability distributions
\begin{itemize}
\item Also called ``Earth Mover's Distance''
\item Measures ``how much work to reshape distribution $\mu_s$ into $\mu_{s'}$?''
\item Captures distributional differences beyond moments
\end{itemize}

\textbf{$\ell_0$:} Embedding scale
\begin{itemize}
\item Units: meters
\item Typically $\sim 1$ micrometer
\item Converts dimensionless feature-space distance into physical length
\end{itemize}

\textbf{$\alpha_W$:} Weight for Wasserstein component
\begin{itemize}
\item Dimensionless, $\alpha_W \geq 0$
\item Tunable (could be fit from data)
\end{itemize}

\textbf{Why this works:}
\begin{itemize}
\item Computable from data
\item Respects symmetries (if $s$ and $s'$ are related by symmetry, $d_\sigma$ is small)
\item Has the right units (meters)
\item Can be calibrated experimentally (see next section)
\end{itemize}

\subsection{Distance Ladder (Calibration)}

We construct controlled distortions of a reference system to map $d_\sigma$:

\begin{center}
\begin{tabular}{clcc}
\toprule
Level & Transformation & Expected $d_\sigma$ & Expected $K = \exp[-d_\sigma/\lambda_\sigma]$ \\
\midrule
0 & Identical ($s' = s$) & 0 & 1.00 \\
1 & Phase rotation (spectrum preserved) & $\varepsilon_1 \ll \lambda_\sigma$ & $\approx 0.90$ \\
2 & Permuted label & $\varepsilon_2 \approx 0.3 \lambda_\sigma$ & $\approx 0.70$ \\
3 & Block-scramble (temporal/spatial) & $\varepsilon_3 \approx 0.7 \lambda_\sigma$ & $\approx 0.50$ \\
4 & Additive noise (SNR = 10 dB) & $\varepsilon_4 \approx \lambda_\sigma$ & $\approx 0.37$ \\
5 & Independent realization & $\varepsilon_5 \gg \lambda_\sigma$ & $\ll 0.10$ \\
\bottomrule
\end{tabular}
\end{center}

\textbf{How to use this:}

\begin{enumerate}
\item \textbf{Prepare reference system} in state $s_0$
\item \textbf{Apply transformations} to create $s_1, s_2, \ldots, s_5$
\item \textbf{Compute signatures} $\sigma(s_0), \sigma(s_1), \ldots$
\item \textbf{Measure $d_\sigma$} using equation (7.1)
\item \textbf{Run experiment E1} (\S13) with each pair $(s_0, s_i)$
\item \textbf{Observe correlation strength} vs. $i$
\item \textbf{Fit model} $K(i) = \exp[-d_\sigma(s_0,s_i)/\lambda_\sigma]$ to extract $\lambda_\sigma$
\end{enumerate}

\textbf{Expected result:}
\begin{itemize}
\item Level 0: Strong correlation ($K = 1$)
\item Levels 1-4: Monotonically decreasing correlation
\item Level 5: No correlation ($K \approx 0$)
\end{itemize}

This directly links observable effects to calibrated structural distances.

\textbf{Pre-registration (crucial):}\\
The distance ladder is run as a \textbf{separate, blind pilot study} before the main experiments. Level labels are sealed until analysis. This prevents unconscious bias and ensures scientific rigor.

\textbf{Practical benefit:}\\
After calibration, you can:
\begin{itemize}
\item Take any two systems $s, s'$
\item Compute $d_\sigma(s,s')$ from their signatures
\item Predict coupling strength $K = \exp[-d_\sigma/\lambda_\sigma]$
\item Test whether observed correlation matches prediction
\end{itemize}

\begin{center}\rule{0.5\linewidth}{0.5pt}\end{center}

