\section{Modified Lieb–Robinson Bound – Proof (Sketch)}

We outline the proof of the soft-cone bound stated in Sec.~9, under the assumptions (A1)–(A9) summarized in Appendix~E.

\subsection*{C.1 Operator algebra and norm choice}

We work on a tensor-product Hilbert space
\[
  \mathcal{H} = \bigotimes_{x\in\Lambda} \mathcal{H}_x
\]
with local, bounded operators. The operator norm is
\[
  \|A\| := \sup_{\|\psi\|=1} \|A\psi\|.
\]

The substrate-mediated part of the Hamiltonian is written as
\[
  \delta H_S
  =
  \sum_{e=(x,y)\in E_S} J_e\,O_x O_y,
\]
where $E_S$ is the set of $S$-edges, $\|O_x\|\le 1$, and the coupling strengths satisfy
\[
  \sum_{e\ni x} |J_e| \le \eta
  \qquad \text{for all sites }x.
\]

\textbf{Sparsity and strength.} Let $g$ be the maximum degree of the $S$-graph (at most $g$ substrate edges per node). The parameter $\eta$ bounds the total coupling strength per node. Assumption (A3) states that $g$ is finite and $\eta$ is small.

\textbf{Extensivity of $d_\sigma$.} We also require:

\begin{assumption}[Extensivity of $d_\sigma$]
\label{ass:extensive_dsigma}
There exists a constant $\delta>0$ such that along any path in $S$ consisting of $m$ $S$-edges connecting $\sigma_x$ to $\sigma_y$ we have
\[
d_\sigma(\sigma_x,\sigma_y) \ge m\,\delta .
\]
In other words, $d_\sigma$ is extensive along $S$-paths: the pattern distance between two nodes is bounded below by a constant times the minimal number of graph steps between them.
\end{assumption}

\subsection*{C.2 Path sums and the $\Phi$-function}

Let $P_m(x\to y)$ be the set of $S$-paths with $m$ hops from $x$ to $y$ (i.e.\ sequences of edges in $E_S$ connecting the two sites). We assume that each hop contributes an average suppression factor $e^{-\mu}$ coming from the kernel
\[
  K_\sigma = \exp[-d_\sigma/\lambda_\sigma],
\]
so that a path with $m$ hops is weighted by $e^{-\mu m}$ for some $\mu>0$.

Using the Duhamel expansion for the Heisenberg evolution and iterating commutators with $\delta H_S$, one can show that the commutator between local observables $A(x,t)$ and $B(y,0)$ satisfies
\begin{equation}
  \big\|[A(x,t),B(y,0)]\big\|
  \;\le\;
  C\,e^{-\kappa(|x-y|-vt)}
  + C'\,\Theta\!\left(t-\frac{d_\sigma}{c_S}\right)
  \sum_{m\ge 1}
  \sum_{p\in P_m(x\to y)}
  \frac{(\eta t/\hbar)^m}{m!}\,
  e^{-\mu m},
  \tag{C.7}
\end{equation}
where $C,C',\kappa,v>0$ are constants coming from the usual Lieb–Robinson analysis for the visible-sector Hamiltonian $H_M$.

The number of distinct $S$-paths with $m$ hops is bounded by
\[
  |P_m(x\to y)| \le g^m,
\]
since at each hop there are at most $g$ outgoing edges. Therefore
\begin{align*}
  \sum_{m\ge 1}
    \sum_{p\in P_m(x\to y)}
    \frac{(\eta t/\hbar)^m}{m!}\,
    e^{-\mu m}
  &\le
  \sum_{m\ge 1}
    \frac{(\eta t/\hbar)^m}{m!}\,
    (g e^{-\mu})^m
  \\[4pt]
  &= \exp\!\big[(g e^{-\mu})\,\eta t/\hbar\big] - 1
   \;\equiv\;
   \Phi\!\left(g,\frac{\eta t}{\hbar}\right).
  \tag{C.8}
\end{align*}

Choosing $\mu>\ln g$ (meaning that the kernel damping in $S$ is strong enough that $g e^{-\mu}<1$) ensures that $\Phi$ grows at most exponentially in $t$ and never saturates to a distance-independent constant for fixed $t$.

\subsection*{C.3 Closure under time evolution}

Assumptions (i)–(iv) in Sec.~9 (sparsity, norm bounds, causality in $\tau$, weak coupling) are stable under time evolution generated by $H_M+\delta H_S$:
\begin{itemize}
  \item $\|O_x(t)\|\le \|O_x\|$ for all local operators $O_x$.
  \item The bound $\sum_{e\ni x}|J_e|\le \eta$ is time independent by construction.
  \item The mediator kernel remains retarded in substrate time $\tau$ with finite speed $c_S$, so the Heaviside factor $\Theta(t-d_\sigma/c_S)$ is preserved.
\end{itemize}
Thus the assumptions propagate through the Duhamel series.

Combining Eqs.~(C.7) and (C.8), and using the relation between path length and structural distance (extensivity of $d_\sigma$ along paths), we obtain the bound quoted in Sec.~9:
\begin{equation}
\begin{split}
  \big\|[A(x,t),B(y,0)]\big\|
  &\le
  C\,e^{-\kappa(|x-y|-vt)}
  \\
  &\quad
  +\;
  C'\,
  \Theta\!\left(t-\frac{d_\sigma(\sigma_x,\sigma_y)}{c_S}\right)\,
  e^{-d_\sigma(\sigma_x,\sigma_y)/\lambda_\sigma}\,
  \Phi\!\left(g,\frac{\eta t}{\hbar}\right),
\end{split}
\tag{C.12}
\end{equation}
with
\[
  \Phi\!\left(g,\frac{\eta t}{\hbar}\right)
  =
  \exp\!\big[(g e^{-\mu})\,\eta t/\hbar\big] - 1.
\]

This is the ``soft cone'' form used in Sec.~9: the usual Lieb–Robinson light cone with effective velocity $v$ plus a substrate-mediated tail that
\begin{itemize}
  \item is retarded in $\tau$ with speed $c_S$,
  \item decays exponentially in the structural distance $d_\sigma$,
  \item remains bounded by a non-saturating function $\Phi$ for fixed $t$.
\end{itemize}
Together with the monotonic substrate time $\tau$, this guarantees that the presence of $S$-edges does not introduce exact instantaneous commutators or closed causal loops.
