\section{Continuity over $(M \times S)$ – Derivation}

We sketch how local conservation in $M$ and $S$ combines into the covariant continuity equations used in Sec.~3.

\subsection{Discrete picture}

Consider discrete time steps $T=0,1,2,\dots$, with:
\begin{itemize}
  \item energy density $\rho_M$ in spacetime cells $c\in M$,
  \item energy density $\rho_\sigma$ in substrate nodes $s\in S$.
\end{itemize}
Let $\Delta E_M(c)$ be the net change in cell $c$ during one tick and let $\sum_{e\in\partial s} J_\sigma(e)$ be the net flow into node $s$ from its $S$-edges $e$.

\textbf{Global conservation over one tick:}
\begin{equation}
  \sum_c \Delta E_M(c) + \sum_s \Delta E_\sigma(s) = 0.
  \tag{B.1}
\end{equation}

\subsection{Continuum limit}

Taking the continuum limit $T\to t$, $c\to x$, $s\to \sigma$ and assuming smooth coarse-grained fields, Eq.~(B.1) yields
\begin{equation}
  \frac{\partial \rho_M}{\partial t} + \nabla\!\cdot\!J_M = - \nabla_\sigma\!\cdot\!J_\sigma,
  \tag{B.2}
\end{equation}
where $J_M$ is the spatial energy flux in $M$ and $J_\sigma$ is the flux in $S$. The right-hand side encodes local exchange between the two sectors.

\subsection{Covariant lift}

To express this covariantly, we consider a bundle $E\to M$ with fiber $S$ and local coordinates $(x^\mu,\sigma^a)$. Introducing a covariant derivative $\nabla_\mu$ on $M$ and interpreting $J^\nu_\sigma$ as an exchange four-current, we split the total energy-momentum tensor into visible and $S$-sector parts,
\[
  T^{\mu\nu}_{\mathrm{tot}} = T^{\mu\nu}_{\mathrm{vis}} + T^{\mu\nu}_S,
\]
and write the continuity equations as
\begin{equation}
  \nabla_\mu T^{\mu\nu}_{\mathrm{vis}} = - J^\nu_\sigma,
  \qquad
  \nabla_\mu T^{\mu\nu}_S = + J^\nu_\sigma.
  \tag{B.3}
\end{equation}
Summing gives total conservation
\begin{equation}
  \nabla_\mu (T^{\mu\nu}_{\mathrm{vis}} + T^{\mu\nu}_S) = 0,
  \tag{B.4}
\end{equation}
which is the covariant version of Eq.~(B.2).

\subsection{Relation to the Einstein equations and $\alpha$}

In the main text we write the Einstein equations as
\begin{equation}
  G_{\mu\nu}
  =
  \frac{8\pi G}{c^4}\,
  \bigl(
    T^{\mathrm{vis}}_{\mu\nu} + \alpha\,T^S_{\mu\nu}
  \bigr),
  \tag{B.5}
\end{equation}
where $\alpha$ is a dimensionless factor multiplying the $S$-sector stress--energy.

The Bianchi identity implies $\nabla_\mu G^{\mu\nu}=0$. Taking the covariant divergence of Eq.~(B.5) and using the split conservation laws (B.3) gives
\[
  0 = \nabla_\mu G^{\mu\nu}
  = \frac{8\pi G}{c^4}
    \bigl(
      \nabla_\mu T^{\mu\nu}_{\mathrm{vis}}
      + \alpha\,\nabla_\mu T^{\mu\nu}_S
    \bigr)
  = \frac{8\pi G}{c^4}(\alpha - 1) J^\nu_\sigma.
\]

For generic configurations with nonzero exchange current $J^\nu_\sigma\neq 0$ this enforces
\begin{equation}
  \alpha \equiv 1,
  \tag{B.6}
\end{equation}
so the metric must couple with the \emph{same} strength to $T^{\mathrm{vis}}_{\mu\nu}$ and $T^S_{\mu\nu}$. In other words, the total conservation law (B.4) is compatible with the Bianchi identity only if the gravitational coupling is universal. All apparent superluminal behaviour thus arises from substrate-local redistribution of stress--energy, not from a modified gravitational constant.
