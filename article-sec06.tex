\section{Coupling Strength and Dimensional Analysis}

At the level of coarse-grained phenomenology it is convenient to express the substrate flow as a scalar power assigned to each active edge or calibrated pair $e$ in the experimental setup. We denote this by $\Psig(e)$ to stress that it is a scalar \emph{power} (units of W), distinct from the 4-current $\Jsig$ appearing in the covariant conservation equations.

\subsection{Power form}

We parameterize the power flowing into the $S$-sector along edge $e$ as
\begin{equation}
  \Psig(e)
  \;=\;
  \varepsilon\,
  \Ksig(e)\,
  \mathcal{Q}(e,t)\,
  \Delta\tilde{\Phi}(e)\,
  P_{\mathrm{pump}}(e)
  \quad\text{(units: W)},
  \tag{6.1}
\end{equation}
where:
\begin{itemize}
  \item $\varepsilon$ is the dimensionless microscopic coupling strength entering the interaction Lagrangian;
  \item $\Ksig(e)=\exp[-d_\sigma(e)/\lambda_\sigma]$ is the dimensionless similarity factor for edge $e$, constructed from the $d_\sigma$-metric of Sec.~7;
  \item $\mathcal{Q}(e,t)\in[0,1]$ is a dimensionless pattern quality factor for that edge at time $t$;
  \item $\Delta\tilde{\Phi}(e)$ is a dimensionless free-energy increment associated with the driven transition on edge $e$;
  \item $P_{\mathrm{pump}}(e)$ is the physical pump power (in W) available to drive the transition.
\end{itemize}
By construction, the right-hand side has dimension of power, so Eq.~(6.1) is dimensionally consistent.

\subsection{Rate form}

It is often convenient to re-express the pump in terms of a rate. For a characteristic quantum $\hbar\omega_0$ of the driven transition, define
\begin{equation}
  \PumpRate(e)
  :=
  \frac{P_{\mathrm{pump}}(e)}{\hbar\omega_0}
  \quad [\mathrm{s^{-1}}],
\end{equation}
so that Eq.~(6.1) can be written as
\begin{equation}
  \Psig(e)
  =
  \varepsilon\,
  \hbar\omega_0\,
  \Ksig(e)\,
  \mathcal{Q}(e,t)\,
  \PumpRate(e)\,
  \Delta\tilde{\Phi}(e).
  \tag{6.2}
\end{equation}
This form will be useful when connecting to the thermodynamic and information-theoretic bounds in Sec.~8.

\subsection{Operational Definition of $Q$}

We formally define the quality factor $\mathcal{Q}$ as a normalized expectation value. Let $|\psi_{id}\rangle$ be a reference "ideal pattern" state in which the pattern selected by $O_S$ is realized maximally. We define the normalization constant $O_0 \equiv \langle \psi_{id} | O_S | \psi_{id} \rangle$.

For any (possibly mixed) state $\rho$ of the visible sector, the pattern quality factor is:

\begin{equation}
\mathcal{Q}[\rho] := \frac{|\mathrm{Tr}(\rho O_S)|}{O_0}
\end{equation}

By construction, $0 \le \mathcal{Q} \le 1$.

In homogeneous systems with $N$ equivalent pattern placements, the state $\rho$ is a mixture over all embeddings. As derived in Appendix I, this yields:

\begin{equation}
\mathcal{Q} \sim \frac{1}{N}
\end{equation}

In the thermodynamic limit ($N \to \infty$), $\mathcal{Q} \to 0$ and therefore $P_\sigma(e) \to 0$. This ensures the S-channel is absent in ordinary matter. Conversely, in engineered systems where the pattern is realized in $O(1)$ locations with aligned phases, $\mathcal{Q}$ can approach unity.
