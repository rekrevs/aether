\section{Ward Identity for (3.7) – Zero Net Momentum}

This appendix fills in the steps leading from translation invariance of the interaction action to the momentum-neutrality condition (3.7).

\subsection{Interaction action and translation invariance}

We assume that the interaction action $S_{\mathrm{int}}$ is exactly translationally invariant in $M$, so that a global spatial translation $x^i \mapsto x^i + a^i$ leaves the interaction kernel $K_{\mathrm{eff}}(x,x')$ invariant. This assumption is crucial: it ensures that the corresponding Noether current is conserved and that the total spatial momentum $P^i_{\mathrm{tot}}$ is strictly conserved, thereby forbidding any reactionless drive.

Consider the effective interaction action between visible-sector fields and the substrate channel,
\begin{equation}
  S^{\rm eff}_{\rm int}
  =
  \varepsilon
  \iint d^4x\,d^4x'\,
  O_S(x)\,\mathcal{K}_{\rm eff}(x,x')\,O_S(x'),
  \tag{H.1}
\end{equation}
where $O_S$ is the local selection operator and $\mathcal{K}_{\rm eff}(x,x')$ is the bilocal kernel defined in Appendix~G. We assume:
\begin{itemize}
  \item the underlying substrate rules are translationally invariant in $M$;
  \item the smearing kernel $f_\ell$ used in the pushforward is translationally invariant;
  \item the structural kernel $\Kern(\sigma,\sigma')$ depends only on $d_\sigma(\sigma,\sigma')$.
\end{itemize}
Then $\mathcal{K}_{\rm eff}(x,x')$ depends only on $x-x'$ and on the difference $\tau(x')-\tau(x)$, and $S^{\rm eff}_{\rm int}$ is invariant under global translations $x^\mu\to x^\mu+a^\mu$.

\subsection{Noether current and interaction stress--energy}

Under an infinitesimal translation $x^\mu\mapsto x^\mu+a^\mu$ the fields transform as
\[
  \delta O_S(x) = a^\nu \partial_\nu O_S(x),
\]
and the variation of the interaction action can be written as
\begin{equation}
  \delta S^{\rm eff}_{\rm int}
  =
  a_\nu \int d^4x\,\partial_\mu\big( T^{\mu\nu}_{\rm int}(x) \big),
  \tag{H.2}
\end{equation}
where $T^{\mu\nu}_{\rm int}$ is the interaction contribution to the total stress--energy tensor. Translation invariance of the action ($\delta S^{\rm eff}_{\rm int}=0$ for constant $a^\nu$) implies
\begin{equation}
  \partial_\mu T^{\mu\nu}_{\rm int} = 0
\end{equation}
in the absence of exchange with the substrate.

In the presence of substrate exchange, the divergence of $T^{\mu\nu}_{\rm int}$ is balanced by the exchange current $J^\nu_\sigma$:
\begin{equation}
  \partial_\mu T^{\mu\nu}_{\rm int} = - J^\nu_\sigma.
  \tag{H.3}
\end{equation}
This relation is the field-theoretic expression of the fact that any momentum carried into the substrate channel must come from the visible sector and vice versa.

\subsection{Total momentum conservation and (3.7)}

The total stress--energy tensor is
\[
  T^{\mu\nu}_{\rm tot}
  =
  T^{\mu\nu}_{\rm vis}
  +
  T^{\mu\nu}_S
  +
  T^{\mu\nu}_{\rm int},
\]
and the split conservation laws in Sec.~3 imply
\[
  \partial_\mu T^{\mu\nu}_{\rm vis} = -J^\nu_\sigma,
  \qquad
  \partial_\mu T^{\mu\nu}_S = +J^\nu_\sigma.
\]
Adding Eq.~(H.3) yields
\begin{equation}
  \partial_\mu T^{\mu\nu}_{\rm tot} = 0.
  \tag{H.4}
\end{equation}

For the spatial components ($\nu=i$) this gives
\[
  \partial_\mu T^{\mu i}_{\rm tot} = 0.
\]
Integrating over a spatial hypersurface at fixed time and assuming appropriate boundary conditions (e.g.\ fields vanish sufficiently fast at spatial infinity) yields
\begin{equation}
  \frac{d}{dt} P^i_{\rm tot}
  =
  \frac{d}{dt} \int d^3x\; T^{0i}_{\rm tot}(t,\mathbf{x})
  = 0,
  \tag{H.5}
\end{equation}
which is precisely Eq.~(3.7) in the main text: the total momentum of the combined visible+substrate system is conserved, and there is no possibility of a reactionless drive. Any apparent momentum gain in one region requires a compensating momentum transfer elsewhere in $M$ or into $T^S_{\mu\nu}$, consistent with translational invariance.
