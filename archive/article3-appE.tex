\section{Assumptions (Summary)}

\begin{itemize}
\item \textbf{(A1)} Global ordering $(\tau)$ with strict retardation,
\item \textbf{(A2)} $\tilde{\mathcal K}\ge0$ (resource monotonicity),
\item \textbf{(A3)} Sparse and weak $(S)$-links: max degree $(g \ll N)$, total strength $(\eta)$ small,
\item \textbf{(A4)} $(O_S)$ RG irrelevant $(\Delta > 4)$ and $(\langle O_S\rangle \approx 0)$ in homogeneous states,
\item \textbf{(A5)} $(W_\sigma)$ positive semidefinite and causal in $(\tau)$,
\item \textbf{(A6)} $(c_T = c)$ in absence of resonance (minimal Lorentz breaking),
\item \textbf{(A7)} Momentum-neutrality (3.7): $(\int d^4x J^i_\sigma = 0)$,
\item \textbf{(A8)} Gravitational coupling: \textbf{$\alpha=1$} (metric responds light-speed causally; FTL lies in S-locality),
\item \textbf{(A9)} Length dimension: $(d_\sigma)$ and $(\lambda_\sigma)$ have meters via embedding scale $(\ell_0)$.
\end{itemize}

\subsection*{Method Note}

For experimental proposals, pre-registration (Open Science Framework or equivalent), blinding, strict environmental isolation (triple-Faraday, optical isolation, battery power), independent replication, and open data/analysis pipelines are recommended to handle very small effect sizes and exclude leak channels. All predictions should be quantitative and all null results should translate to upper bounds on coupling parameters with explicit statistical analysis (Bayes factors, p-values, SPRT, permutation tests $\geq 10^6$, \textbf{multiple-test correction via Holm-Bonferroni or FDR}).

\vspace{1em}
\noindent\textit{This article describes a speculative but internally consistent mechanism with explicit falsifiability criteria. Either it results in stringent upper bounds – or in a new class of reproducible non-local effects. Both outcomes deserve careful testing.}

