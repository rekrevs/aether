\section{The Setup: Two Spaces, One Reality}

\begin{quote}
\textbf{Core Concept:} Reality has two distance metrics - ordinary distance in space, and structural distance in pattern space.
\end{quote}

\subsection{Postulate 1: Discrete Dynamics}

\textbf{The Technical Statement:}\\
The substrate evolves in discrete steps T = 0, 1, 2, ... All causality is monotonic in T.

\textbf{What This Means:}\\
Imagine reality as a cosmic spreadsheet that updates row by row. Each row is one ``tick'' of the substrate's clock. You can't have an effect from row 100 influence row 99. Causality only flows forward.

\emph{Mental model:} Like frames in a movie. Frame 100 can't affect frame 99. But within the movie (the emergent spacetime), characters can walk backward, time zones differ, etc. The frame number is separate from the in-movie time.

This is our protection against time paradoxes. Even if a signal goes ``backward in time'' within spacetime, it always goes \emph{forward in substrate-time T}.

\subsection{Postulate 2: Two Proximities}

\textbf{The Technical Statement:}
\begin{itemize}
\item \textbf{(M)}: Emergent spacetime with metric $g_{\mu\nu}$, where ordinary matter moves locally and obeys relativity
\item \textbf{(S)}: Pattern space where distance $d_\sigma$ measures algorithmic similarity
\item A projection $\pi: S \to M$ specifies where substrate states appear in spacetime
\end{itemize}

\textbf{Translation:}

\textbf{M} is the world you experience. Space (three dimensions), time (one dimension), woven together into spacetime. Matter here obeys Einstein's rules. Light travels at $c \approx 300{,}000$ km/s. This is the ``screen'' where reality is displayed.

\textbf{S} is the substrate's internal structure - the ``code'' running underneath. Points in S represent different configurations of the substrate's data structures. The distance $d_\sigma$ between two points measures ``how different are these configurations?''

\emph{Concrete example:}
\begin{itemize}
\item You have two quantum systems, both in the state $|\psi\rangle = (|0\rangle + |1\rangle)/\sqrt{2}$
\item In \textbf{M} they're 10 light-years apart (separation = $10^{17}$ meters)
\item In \textbf{S} they're identical ($d_\sigma = 0$ because same quantum state)
\end{itemize}

\textbf{The projection $\pi$} is like the rendering engine in a game. It takes the substrate state (a giant configuration of data) and outputs ``here's where particle A appears in spacetime, here's where particle B appears.''

\textbf{Crucial point about units:}\\
$d_\sigma$ has \textbf{length units} (meters). But it's not measuring meters in ordinary space. There's an embedding scale $\ell_0 \sim 1$ micrometer that converts ``algorithmic dissimilarity'' into a distance scale.

\emph{Why this matters:} It makes the theory predictive. We can say ``if two systems differ by $d_\sigma = 10$ micrometers of structural distance, the coupling falls off exponentially.''

\subsection{Postulate 3: Aether Resonance}

\textbf{The Technical Statement:}\\
There exists a coupling that, within one tick T, allows energy/information flow between points $(s, s' \in S)$ with small $d_\sigma(s,s')$, independent of the spacetime separation $|\pi(s) - \pi(s')|$.

\textbf{In Plain English:}\\
If two things are structurally similar (small $d_\sigma$), the substrate can transfer energy/information between them in one step, even if they're far apart in ordinary space.

\textbf{The crucial analogy: Subways vs Streets}

Imagine two buildings in a city:
\begin{itemize}
\item \textbf{Surface route} (spacetime): Walk through streets, obey traffic lights, takes 30 minutes to go 2 km. This is like light traveling through space.
\item \textbf{Subway route} (substrate): Both buildings are next to the same subway station. Take the subway, arrive in 5 minutes. This ``shortcut'' doesn't violate surface-world geography - it uses underground infrastructure.
\end{itemize}

Aether resonance is like the subway. It's a transfer that happens ``underneath'' spacetime, in the substrate's own structure.

\textbf{But there's a catch:}
\begin{itemize}
\item The subway only connects buildings near subway stations (need small $d_\sigma$ - structural similarity)
\item The subway costs energy to run (thermodynamic cost)
\item The subway has its own rules (substrate causality)
\end{itemize}

\subsection{Postulate 4: Conservation}

\textbf{The Technical Statement:}\\
Total energy/information is conserved over the combined dynamics, even though local budgets in M may vary via flows in S.

\textbf{What This Means:}\\
Energy doesn't disappear or appear from nowhere. It just has two places it can ``move'':
\begin{enumerate}
\item Through ordinary spacetime (M) - what we usually see
\item Through pattern space (S) - the new mechanism
\end{enumerate}

\emph{Analogy:} Money flowing between bank accounts. The total is conserved, but:
\begin{itemize}
\item Regular transfers: Check in the mail (slow, through space)
\item Wire transfers: Electronic (fast, through the banking network's internal structure)
\end{itemize}

The substrate keeps track of the total energy budget, making sure nothing is created or destroyed.

\begin{center}
\rule{0.5\linewidth}{0.5pt}
\end{center}

