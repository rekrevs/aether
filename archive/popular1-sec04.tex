\section{How the Substrate Does It: The S-Mediator}

\begin{quote}
\textbf{Core Concept:} The substrate can't just ``magically know'' which points in $S$ are nearby. It needs a mechanism. That mechanism is a field that propagates locally in $S$.
\end{quote}

\subsection{The Problem Statement}

\textbf{Challenge:} The substrate is like a computer with only local operations. Each CPU core only knows about its neighbors. How can it implement ``proximity in pattern space'' without a global lookup table?

\emph{Analogy:} You're in a massive multiplayer game. How does the game server know which players should be able to see each other, without checking every pair of players against every other?

\textbf{Answer in games:} Spatial partitioning. Divide the world into cells, only check players in nearby cells.

\textbf{Answer in our substrate:} A \textbf{dynamic mediator field $\chi(\sigma,T)$} that propagates through pattern space $S$.

\subsection{The Mediator Dynamics}

Each point $\sigma$ in pattern space carries a field $\chi(\sigma,T)$ that obeys a wave equation:

\begin{equation}
\partial_T^2 \chi - c_S^2 \nabla_\sigma^2 \chi + m_\chi^2 \chi = J_S(\sigma,T),
\tag{4.1}
\end{equation}

\textbf{Let's decode this:}

\textbf{$\chi(\sigma,T)$}: The mediator field
\begin{itemize}
\item A number assigned to each pattern-space point $\sigma$
\item Evolves with substrate time $T$
\end{itemize}

\textbf{$\partial_T^2 \chi$}: Second derivative in time
\begin{itemize}
\item Measures ``acceleration'' of the field
\end{itemize}

\textbf{$\nabla_\sigma^2 \chi$}: Laplacian in pattern space
\begin{itemize}
\item Measures ``how much $\chi$ curves'' in the $S$ directions
\item Think of it as diffusion or wave propagation
\end{itemize}

\textbf{$c_S$}: Propagation speed in $S$
\begin{itemize}
\item Dimensionless (or in units of substrate-ticks/length-in-$S$)
\item Determines how fast signals travel through pattern space
\end{itemize}

\textbf{$m_\chi$}: Effective mass
\begin{itemize}
\item Gives the field a ``range'' $\lambda_\sigma = c_S / m_\chi$
\item Massive fields have finite range (exponential decay)
\end{itemize}

\textbf{$J_S(\sigma,T)$}: Source term
\begin{itemize}
\item Comes from visible matter via the selection operator $O_S$
\item Says ``matter at spacetime point $x$ (which projects to pattern $\sigma$) excites the mediator field''
\end{itemize}

\textbf{The analogy:}\\
This equation is exactly like electromagnetism or gravitational waves, but it lives in pattern space $S$ instead of ordinary space $M$.

\begin{itemize}
\item \textbf{Electromagnetism:} Photon field propagates through space at speed $c$, sourced by charges
\item \textbf{Mediator field:} $\chi$ propagates through pattern space at speed $c_S$, sourced by matter configurations
\end{itemize}

\textbf{Why a wave equation?}\\
Wave equations naturally give you:
\begin{enumerate}
\item \textbf{Locality:} The field at point $\sigma$ only depends on nearby points
\item \textbf{Retardation:} Effects propagate at finite speed $c_S$
\item \textbf{Exponential decay:} With mass $m_\chi$, distant effects are suppressed as $\exp[-m_\chi d_\sigma/c_S]$
\end{enumerate}

\subsection{The Retarded Solution}

The solution to (4.1) is:

\begin{equation}
\chi(\sigma',T') = \int d\mu(\sigma) \int dT \, G_{\rm ret}(\sigma',T'; \sigma,T) \, J_S(\sigma,T),
\tag{4.2}
\end{equation}

with the \textbf{retarded Green's function}:

\begin{equation}
G_{\rm ret}(\sigma',T';\sigma,T) \propto \frac{e^{-m_\chi d_\sigma(\sigma,\sigma')/c_S}}{d_\sigma(\sigma,\sigma')} \, \Theta(T'-T - d_\sigma(\sigma,\sigma')/c_S).
\tag{4.3}
\end{equation}

\textbf{Breaking this down:}

\textbf{Green's function $G_{ret}$:} The ``response function''
\begin{itemize}
\item Tells you: ``If there's a source at $(\sigma,T)$, how much does it contribute to the field at $(\sigma',T')$?''
\end{itemize}

\textbf{$\exp[-m_\chi d_\sigma/c_S]$:} Exponential falloff
\begin{itemize}
\item Makes distant points (large $d_\sigma$) contribute almost nothing
\item This is the ``range'' of the field: $\lambda_\sigma = c_S/m_\chi$
\end{itemize}

\textbf{$1/d_\sigma$:} Standard wave-field decay
\begin{itemize}
\item Like how light intensity falls off as $1/r^2$
\end{itemize}

\textbf{$\Theta(T' - T - d_\sigma/c_S)$:} The Heaviside step function (retardation)
\begin{itemize}
\item Equals 1 if $T' > T + d_\sigma/c_S$ (signal has had time to propagate)
\item Equals 0 otherwise (signal hasn't arrived yet)
\item \textbf{This ensures substrate causality:} No backward-in-$T$ propagation
\end{itemize}

\textbf{Visual metaphor:}\\
Imagine dropping a pebble in a pond:
\begin{itemize}
\item Ripples spread outward at speed $c_S$ (in $S$)
\item Ripple amplitude decays exponentially with distance (if $m_\chi > 0$)
\item Ripples only exist after the pebble drops (retardation)
\end{itemize}

The mediator field $\chi$ does the same thing, but in pattern space.

\subsection{The Emergent Kernel $\Kern$}

The effective coupling kernel in equation (3.2) emerges from the mediator dynamics:

\begin{equation}
\Kern(\sigma,\sigma') = \int dT \, G_{\rm ret}(\sigma',T'; \sigma,T)
\approx e^{-d_\sigma(\sigma,\sigma')/\lambda_\sigma} \equiv \Ksig(d_\sigma(\sigma,\sigma')).
\tag{4.4}
\end{equation}

The retarded Green's function $G_{\rm ret}$ already contains the Heaviside step function, ensuring substrate retardation. For phenomenological applications involving a single edge or calibrated pair, we write $\Ksig \equiv e^{-d_\sigma/\lambda_\sigma}$ as the similarity factor.


\noindent\textbf{``Integrating Out'' the Mediator Field $\chi$}

\textbf{What does ``integrate out'' mean?}

Think of it like this:
\begin{enumerate}
\item \textbf{Start with:} A local field $\chi(\sigma,T)$ that lives at every point in pattern space and evolves in substrate time.
\item \textbf{Solve its equation:} The wave equation (4.1) tells us exactly how $\chi$ responds to sources $J_S$.
\item \textbf{Substitute back:} Replace $\chi$ everywhere with its solution (the retarded Green's function).
\item \textbf{Result:} The local field $\chi$ disappears, leaving behind a \textbf{nonlocal kernel} $\Kern(\sigma,\sigma')$ connecting distant points.
\end{enumerate}

\textbf{Analogy from quantum field theory:}

This is exactly like ``integrating out virtual particles'':
\begin{itemize}
\item In QED, electrons interact by exchanging virtual photons
\item You can solve for the photon field explicitly
\item The result is an \textbf{effective interaction} between electrons (the Coulomb potential)
\item The photons are still ``there'' (in the dynamics), but you don't see them explicitly in the final formula
\end{itemize}

\textbf{What we gain:}

\begin{itemize}
\item \textbf{Exponential falloff:} $\Kern \sim e^{-d_\sigma/\lambda_\sigma}$ emerges naturally from massive mediator.
\item \textbf{Automatic retardation:} $\Theta(T' - T - d_\sigma/c_S)$ built into Green's function.
\item \textbf{Simpler phenomenology:} Work directly with kernel $\Kern$ instead of tracking field $\chi$.
\end{itemize}

\textbf{Physical interpretation:} The kernel $\Kern$ encodes the \emph{effective strength} of coupling between two substrate points after accounting for all the intermediate propagation dynamics of the mediator field.

\medskip

\textbf{Properties of the resonance kernel $\Kern(\sigma,\sigma')$:}
\begin{enumerate}
\item $\Kern(\sigma,\sigma') \ge 0$ (positivity)
\item Symmetric and positive-definite (Mercer kernel)
\item Retarded in $T$ (no advanced support, inherited from $G_{\rm ret}$)
\end{enumerate}

Because $\Kern$ is positive semidefinite and retarded in $T$, the spacetime-smeared kernel $\mathcal{K}_{\rm eff}$ inherits these properties; adding the $\mathcal{K}_{\rm eff}$ coupling preserves unitarity and respects substrate-time causality.

\textbf{What this says:}\\
The kernel $\Kern$, which appeared mysteriously in our action (3.2), actually \textbf{emerges} from the mediator field dynamics. It's not put in by hand - it's derived.

\textbf{The approximate form:}\\
After integrating over substrate time $T$ and doing some math, we get:

$\Kkernel(\sigma,\sigma') \approx \exp[-d_\sigma(\sigma,\sigma')/\lambda_\sigma]$

\textbf{Interpretation:}
\begin{itemize}
\item If $d_\sigma = 0$ (identical patterns): $\Kkernel = 1$ (maximum coupling)
\item If $d_\sigma = \lambda_\sigma$ (patterns differ by one coherence length): $\Kkernel \approx 0.37$ (coupling drops by $1/e$)
\item If $d_\sigma \gg \lambda_\sigma$ (very different patterns): $\Kkernel \approx 0$ (essentially no coupling)
\end{itemize}

\textbf{The Heaviside factor:}\\
The $\Theta$ factor in (4.4) ensures explicit substrate retardation at the kernel level. Even in the effective theory, we see that coupling respects causality in substrate-time $T$.

\subsection{Three Key Results}

This mediator mechanism gives us:

\textbf{1. No Global Search}\\
Each point $\sigma$ in pattern space only needs to ``know about'' its immediate neighbors via $\nabla_\sigma^2$. The field $\chi$ propagates locally, step by step, through $S$. There's no ``bulletin board'' where all configurations are compared.

\emph{Analogy:} Like gossip spreading through a social network. Each person tells their neighbors, who tell their neighbors. Eventually information spreads, but nobody needs a global directory.

\textbf{2. Retarded in $T$}\\
Signals reach $\sigma'$ from $\sigma$ only after $T' \geq T + d_\sigma/c_S$. This is \textbf{substrate causality} - the fundamental protection against paradoxes.

Even though the effect can be FTL in spacetime $M$, it's always causal in substrate-time $T$.

\textbf{3. Exponential Decay}\\
$\Kkernel$ falls off as $\exp[-d_\sigma/\lambda_\sigma]$, naturally making the coupling local in pattern space. For massive mediators ($m_\chi > 0$), the range $\lambda_\sigma$ is finite.

\textbf{The beauty:}\\
This is exactly how known forces work in spacetime:
\begin{itemize}
\item Photons (massless): $1/r$ falloff, infinite range
\item W/Z bosons (massive): $\exp[-mr]/r$ falloff, short range
\end{itemize}

We're applying the same principle, but in pattern space instead of ordinary space.

\begin{center}\rule{0.5\linewidth}{0.5pt}\end{center}

