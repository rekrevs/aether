\section{Introduction and Motivation}

Relativity and quantum mechanics provide a consistent, locally causal description of nature. At the same time, discrete, rule-based substrate models (e.g., cellular automata, hypergraphs) offer a natural framework for thinking about emergence. Here we investigate the speculative but internally consistent hypothesis that:

\begin{enumerate}
\item Observable spacetime $(M)$ with light speed $(c)$ arises as an effective description of a discrete substrate with global update ordering $(T=0,1,2,\ldots)$.
\item There exists a second notion of distance — \textbf{structural proximity} — in a pattern space $(S)$ where two substructures are ``close'' if they are algorithmically isomorphic.
\item A weak, substrate-local coupling in $(S)$ — \textbf{aether resonance} — can transport energy and information ``in place'' in $(S)$, which is experienced as FTL in $(M)$.
\end{enumerate}

The question is whether this can be made \textbf{physically coherent}: conservation laws, momentum-neutrality, absence of time paradoxes, compatibility with negative experimental results, and \textbf{falsifiable consequences}.

