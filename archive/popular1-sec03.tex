\section{The Mathematical Core: The Action Principle}

\begin{quote}
\textbf{Core Concept:} We describe the entire framework using a ``master equation'' that determines how fields and spacetime behave.
\end{quote}

\subsection{What is an Action?}

Before we dive into equations, let's understand what physicists mean by an ``action.''

\textbf{The Big Idea:}\\
Nature is lazy. Of all possible ways a system could evolve from state A to state B, it chooses the path that minimizes (or more precisely, extremizes) a quantity called the \textbf{action}.

\emph{Analogy:} Light traveling between two points always takes the path requiring the least time. That's why it bends in water (refraction) - the bent path is faster than a straight line through the slower medium.

The action is like a ``cost function.'' Nature computes the cost of every possible path and picks the cheapest one.

\textbf{For our theory:}\\
We write down an action $S_{tot}$ that includes:
\begin{enumerate}
\item Einstein's gravity (how spacetime curves)
\item Regular matter and fields (particles, light, etc.)
\item The substrate structure (the ``clock'' and ``preferred frame'')
\item \textbf{The resonance interaction} (the new part - coupling through S)
\end{enumerate}

From this action, we derive all the equations of motion.

\subsection{The Total Action}

Here's the master equation (don't panic - we'll unpack it):

\begin{equation}
\begin{split}
S_{tot} = \int d^4x \, \sqrt{-g} \left[ \frac{1}{16\pi G} R + \mathcal{L}_{vis}[\phi, g] + \mathcal{L}_S[\tau, u^\mu, g] \right] \\
 + \varepsilon \int d^4x \sqrt{-g}\!\int\! d\mu(\sigma)\,d\mu(\sigma')\,\frac{O_S(x,\sigma)\,\Kern(\sigma,\sigma')\,O_S(x,\sigma')}{\LamUV^{4}},
\end{split}
\tag{3.1}
\end{equation}

\textbf{Let's decode this piece by piece:}

\textbf{First line:} $\int d^4x \sqrt{-g}$ [...]
\begin{itemize}
\item This says ``integrate over all of spacetime''
\item $d^4x$ means ``a little chunk of space and time'' (4 dimensions: x, y, z, t)
\item $\sqrt{-g}$ is a technical factor ensuring we measure volumes correctly in curved spacetime
\end{itemize}

\textbf{$R/(16\pi G)$}: This is Einstein's gravity term. R is the ``Ricci scalar'' - a measure of how curved spacetime is. G is Newton's gravitational constant. This term says ``curved spacetime has energy'' (mass curves space, curved space affects mass - that's gravity).

\textbf{$\mathcal{L}_{vis}[\phi, g]$}: This represents all ordinary matter and fields
\begin{itemize}
\item $\phi$ stands for all the quantum fields (electron field, photon field, quark fields, etc.)
\item $g$ is the spacetime metric (tells you distances and times)
\item This is the ``Standard Model plus gravity'' - known physics
\end{itemize}

\textbf{$\mathcal{L}_S[\tau, u^\mu, g]$}: This represents the substrate structure
\begin{itemize}
\item $\tau$ (tau) is the ``foliation scalar'' - a field that defines the substrate's universal clock
\item $u^\mu$ is a unit timelike vector - the ``preferred direction in time'' from the substrate's perspective
\item This gives spacetime an additional subtle structure
\end{itemize}

\textbf{The last term} ($\varepsilon \times$ integral $\times$ integral $\times$ ...): This is the \textbf{resonance interaction} - the new physics!

Let's zoom in on this term:

\begin{equation}
S_{int} = \varepsilon \int d^4x \sqrt{-g}\!\int\! d\mu(\sigma)\,d\mu(\sigma')\;
\frac{ O_S(x,\sigma)\,\Kern(\sigma,\sigma')\,O_S(x,\sigma') }{\LamUV^{\,4}},
\tag{3.2}
\end{equation}

\textbf{Breaking it down:}

\textbf{$\varepsilon$ (epsilon)}: The coupling strength. A tiny dimensionless number $\sim 10^{-15}$. This is why the effect is so weak!

\textbf{$\int d^4x$}: Integrate over spacetime points (x,y,z,t)

\textbf{$\int d\mu(\sigma) d\mu(\sigma')$}: Integrate over pairs of points in pattern space S
\begin{itemize}
\item Think of this as ``consider all possible pairs of substrate configurations''
\item $d\mu$ is the ``measure'' - how we count configurations (it's dimensionless)
\end{itemize}

\textbf{$O_S(x,\sigma)$}: The ``selection operator''
\begin{itemize}
\item Takes a spacetime point x and a pattern-space point $\sigma$
\item Returns a number saying ``how much does the matter at spacetime point x match the pattern $\sigma$?''
\item Has mass dimension 4 (this is why we divide by $\LamUV^4$ - for dimensional consistency)
\item We'll discuss this much more in \S 5
\end{itemize}

\textbf{$\Kern(\sigma,\sigma')$}: The ``bilocal resonance kernel''
\begin{itemize}
\item Takes two points in pattern space
\item Returns how strongly they can resonate with each other
\item Approximately: $\Kern \sim \exp[-d_\sigma/\lambda_\sigma]$, or $\Ksig$ for a single edge
\item If $d_\sigma$ is small (patterns similar), $\Kern \approx 1$ (strong coupling)
\item If $d_\sigma$ is large (patterns different), $\Kern \approx 0$ (no coupling)
\end{itemize}

\textbf{$\Lambda_*$ (Lambda-star)}: A high-energy scale (mass dimension 1)
\begin{itemize}
\item This makes the dimensions work out correctly
\item Represents the energy scale where new physics kicks in
\item Probably very large (way beyond what we can probe in accelerators)
\end{itemize}

\textbf{Putting it together:}

The interaction term says: ``At each spacetime point x, look at the matter configuration. Calculate which pattern $\sigma$ it matches (via $O_S$). Find other spacetime points that match similar patterns $\sigma'$. Connect them via the kernel $\Kern$. The strength of this connection falls off exponentially with structural distance $d_\sigma$.''

\textbf{Visual metaphor:}\\
Imagine spacetime as a giant room full of tuning forks. Each tuning fork is a quantum system at a spacetime location.

\begin{itemize}
\item The selection operator $O_S$ measures ``how loud is this tuning fork ringing, and at what pitch?''
\item The kernel $\Kern(\sigma,\sigma')$ says ``if one fork rings at pitch $\sigma$ and another at pitch $\sigma'$, how strongly do they resonate?''
\item The integral connects every pair of forks that can ``hear'' each other
\end{itemize}

\textbf{Why this is manifestly covariant:}\\
Notice that the interaction term uses the metric $\sqrt{-g}$ and integrates over $d^4x$. This means it respects spacetime's geometry - it works the same way in any reference frame. The coupling is \emph{added to} Einstein's gravity and regular quantum fields, not breaking their structure.

The term $\int d\mu(\sigma) d\mu(\sigma')$ is purely internal to pattern space S. It doesn't depend on spacetime coordinates except through $O_S(x,\sigma)$, which probes what's at each spacetime point.

\textbf{Key insight:}\\
This action describes a ``bilayer'' theory:
\begin{itemize}
\item \textbf{Top layer} (spacetime M): Standard relativity and quantum mechanics
\item \textbf{Bottom layer} (substrate S): Discrete updates, structural proximity
\item \textbf{Coupling between layers}: The $\varepsilon \times O_S \times \Kern \times O_S$ term
\end{itemize}

The beauty is that everything follows from this one action via the variational principle.

\subsection{Specifying the Substrate Structure $\mathcal{L}_S$}

We need to be concrete about $\mathcal{L}_S$. Here are two options:

\textbf{Option A: Minimal Khronon (Simple Version)}

\begin{equation}
\mathcal{L}_S^{\text{min}} = \frac{M_S^2}{2}\,\lamL(x)\,\big(u^\mu u_\mu + 1\big),\qquad
u^\mu:=\frac{\nabla^\mu \tau}{\sqrt{-\,\nabla_\alpha \tau \nabla^\alpha \tau}}.
\tag{3.1A}
\end{equation}

\textbf{What this says:}
\begin{itemize}
\item $\tau$ (tau) is a scalar field - the ``universal clock''
\item $u^\mu$ is the unit vector pointing in the direction $\tau$ increases
\item The Lagrange multiplier $\lamL(x)$ enforces $u^\mu u_\mu = -1$ (it's timelike and unit-normalized)
\item No kinetic terms - the preferred frame is ``just there'' but doesn't carry new dynamics
\end{itemize}

\emph{Mental model:} Like GPS time. Underneath your relativistic spacetime, there's a universal clock ticking. You don't usually notice it, but it's there, defining a preferred notion of ``now'' and ``the flow of time.''

\textbf{Option B: Einstein-Æther (Full Version)}

\begin{equation}
\begin{split}
\mathcal{L}_S^{\text{EA}}=\frac{M_S^2}{2}
\big[c_1(\nabla_\mu u_\nu)(\nabla^\mu u^\nu)
 +c_2(\nabla_\mu u^\mu)^2 \\
 +c_3(\nabla_\mu u_\nu)(\nabla^\nu u^\mu)
 +c_4\,u^\mu u^\nu(\nabla_\mu u_\alpha)(\nabla_\nu u^\alpha)\big]
 +\frac{M_S^2}{2}\,\lamL(x)\,(u^\mu u_\mu+1).
\end{split}
\tag{3.1B}
\end{equation}

\textbf{What this says:}\\
This version allows the preferred frame $u^\mu$ to have its own dynamics (it can ``ripple'' and ``bend'').

The coefficients $c_1, c_2, c_3, c_4$ control different aspects:
\begin{itemize}
\item $c_1$: shear (how much $u^\mu$ twists)
\item $c_2$: expansion (how much $u^\mu$ stretches)
\item $c_3$: twist coupling
\item $c_4$: acceleration
\end{itemize}

\textbf{Constraints we impose:}
\begin{equation}
  c_{13}:=c_1+c_3=0 \;\Rightarrow\; c_T=c,
  \qquad |c_i|\ll 1,
  \qquad \alpha_1\simeq 0,\;\alpha_2\simeq 0 \text{ (linear order)}.
  \tag{3.1C}
\end{equation}

\textbf{What this constraint does:}
\begin{itemize}
\item $c_{13} = 0$ ensures gravitational waves travel at the speed of light ($c_T = c$)
\item $|c_i| \ll 1$ ensures tiny deviations from general relativity
\item $\alpha_1\simeq 0,\;\alpha_2\simeq 0$ are the PPN (parametrized post-Newtonian) parameters that ensure compatibility with solar system tests
\end{itemize}

\textbf{Why we have two options:}
\begin{itemize}
\item Option A is simpler and safer (minimal new structure)
\item Option B is more general (allows exploration of substrate dynamics)
\item Both are compatible with the resonance mechanism
\end{itemize}

\textbf{The key point:}\\
We're adding a subtle structure to spacetime - a preferred time direction - without breaking relativity's tested predictions. The substrate clock $\tau$ defines an absolute ``cosmic time,'' but its effects are normally undetectable. Only through the resonance term does it matter.

\begin{center}
\rule{0.5\linewidth}{0.5pt}
\end{center}

\subsection{What Falls Out: The Equations of Motion}

Once we have the action $S_{tot}$, we can derive everything using the \textbf{variational principle}: vary the action with respect to each field, set the variation to zero, and you get the equations of motion.

\textbf{1. Einstein's Equations (Modified)}

\begin{equation}
G_{\mu\nu} = \frac{8\pi G}{c^4}\big(T^{\mu\nu}_{vis}+T^{\mu\nu}_S\big),
\tag{3.3}
\end{equation}

\textbf{Translation:}
\begin{itemize}
\item $G_{\mu\nu}$ is the ``Einstein tensor'' - describes spacetime curvature
\item $T^{\mu\nu}_{vis}$ is the energy-momentum tensor of visible matter (what's usually on the right side of Einstein's equations)
\item $T^{\mu\nu}_S$ is a \textbf{new} energy-momentum tensor from the substrate structure
\end{itemize}

\textbf{What this means:}\\
The substrate contributes to spacetime curvature! The preferred frame $u^\mu$ and the clock field $\tau$ carry energy and momentum, which gravity responds to.

\emph{Analogy:} Spacetime curvature is like a rubber sheet. Normally only matter (balls placed on the sheet) causes curvature. Now we're saying the sheet itself has internal structure (fibers running through it) that also contributes to its shape.

\textbf{2. Energy-Momentum Conservation (With a Twist)}

\begin{equation}
\nabla_\mu T^{\mu\nu}_{vis} = -J^\nu_{\sigma}, \quad \nabla_\mu T^{\mu\nu}_{S} = +J^\nu_{\sigma},
\tag{3.4}
\end{equation}

\textbf{Translation:}
\begin{itemize}
\item $\nabla_\mu$ is the ``covariant derivative'' - a way to take derivatives in curved spacetime
\item $J^\nu_\sigma$ is the ``four-current'' from the resonance interaction
\end{itemize}

\textbf{What this means:}\\
Energy-momentum is \textbf{not} separately conserved for visible matter or the substrate individually! Instead:
\begin{itemize}
\item Visible matter can lose energy: $\nabla_\mu T^{\mu\nu}_{vis} = -J^\nu_\sigma$
\item The substrate gains that energy: $\nabla_\mu T^{\mu\nu}_S = +J^\nu_\sigma$
\end{itemize}

Energy flows from matter $\to$ substrate (or vice versa) through the resonance coupling.

\emph{Analogy:} Two bank accounts. Money flows between them, but the total is conserved.

\textbf{3. Total Conservation}

\begin{equation}
\nabla_\mu (T^{\mu\nu}_{vis} + T^{\mu\nu}_S) = 0.
\tag{3.5}
\end{equation}

\textbf{Translation:}\\
The \textbf{total} energy-momentum is conserved. The sum of visible matter plus substrate obeys standard conservation laws.

\textbf{Why this is crucial:}\\
This ensures the theory doesn't violate conservation of energy or momentum overall. Energy can ``move'' from spacetime to the substrate and back, but the total budget is fixed.

\begin{center}\rule{0.5\linewidth}{0.5pt}\end{center}

\subsection{Making S-Flows Observable in Spacetime}

\textbf{The Problem:}\\
Flows in pattern space $S$ are abstract - they happen ``underneath'' spacetime. How do they manifest as observable effects in $M$?

\textbf{The Solution: Pushforward with a Worldtube}

We use a ``smeared projection'' with a compact support function $f_\ell$ (normalized so that $\int d^4x\, f_\ell(x) = 1$):

\begin{equation}
S^\nu(x)=\!\int\! d\mu(\sigma)\, f_\ell\!\big(x-\pi(\sigma)\big)\,(\nabla_\sigma\!\cdot\!J_\sigma)^\nu(\sigma),
  \qquad \int d^4x\, f_\ell(x) = 1.
\tag{3.6}
\end{equation}

\textbf{Unpacking this:}

\textbf{$\pi(\sigma)$}: The projection from pattern space to spacetime
\begin{itemize}
\item Takes a substrate configuration $\sigma$
\item Outputs where that configuration ``appears'' in spacetime
\end{itemize}

\textbf{$f_\ell(x - \pi(\sigma))$}: A ``worldtube'' or ``smearing function''
\begin{itemize}
\item Centered at $\pi(\sigma)$
\item Has width $\ell$ (small, much smaller than experimental scales)
\item Says ``the substrate configuration at $\sigma$ influences spacetime in a small neighborhood around $\pi(\sigma)$''
\end{itemize}

\emph{Analogy:} Like pixels on a screen. Each pixel (spacetime point $x$) gets contributions from nearby substrate points $\sigma$, weighted by how close they are.

\textbf{$\nabla_\sigma \cdot J_\sigma$}: The divergence of the current in pattern space
\begin{itemize}
\item Measures ``net flow out of'' a point in $S$
\end{itemize}

\textbf{The result $S^\nu(x)$}: A four-vector (energy-momentum current) at spacetime point $x$, derived from flows in pattern space.

\textbf{Why this works:}
\begin{itemize}
\item Respects diffeomorphism invariance (coordinate independence)
\item Makes the coupling between $M$ and $S$ explicit and calculable
\item Ensures well-defined behavior under variations (smooth, no singularities)
\end{itemize}

\textbf{Visual metaphor:}\\
Imagine the substrate as underground water pipes. The flow through the pipes (in $S$) eventually emerges as fountains in the visible city (in $M$). The worldtube $f_\ell$ specifies how underground flow at location $\sigma$ manifests as visible water at spacetime location $x$.

\textbf{Normalization note:}\\
Throughout \S3, the measure $d\mu(\sigma)$ is \textbf{dimensionless}. All mass dimensions come from $O_S$ (dimension 4) and $\Lambda_*$ (dimension 1). This keeps the dimensional bookkeeping clean.

\begin{center}\rule{0.5\linewidth}{0.5pt}\end{center}

\subsection{The Momentum Neutrality Condition}

\textbf{The Statement:}
\begin{equation}
\int d^4x\, J^i_\sigma(x)=0,
\tag{3.7}
\end{equation}

\textbf{Translation:}\\
The spatial components ($i = 1, 2, 3$) of the resonance current, when integrated over all spacetime, sum to zero.

\textbf{What this means - The ``No Reactionless Drive'' Theorem:}

Imagine you could push on the substrate to create momentum without an equal and opposite reaction in spacetime. You could accelerate forever without propellant - a reactionless drive, like pulling yourself up by your bootstraps.

Equation (3.7) says: \textbf{This is forbidden.}

\textbf{Why it follows:}\\
The interaction term $S_{int}$ is \textbf{bilocal} (couples two points) and \textbf{translationally symmetric} (doesn't prefer any spatial location). By Noether's theorem, this implies:
\begin{itemize}
\item Translations in space are a symmetry
\item Symmetry implies a conserved quantity (momentum)
\item The conserved quantity for the interaction is $\int J^i_\sigma d^4x = 0$
\end{itemize}

\textbf{Technical note:} This requires that the kernel $\Kern$ is symmetric under $\sigma \leftrightarrow \sigma'$ and that boundary conditions produce no net flux at infinity (no preferred direction for substrate transfers).

\textbf{Concrete example:}\\
Suppose you build a machine that transfers energy between two labs via aether resonance:
\begin{itemize}
\item Lab A pumps energy into the substrate
\item Lab B receives energy from the substrate
\item Energy flows A $\to$ substrate $\to$ B
\end{itemize}

Can Lab A use the ``recoil'' from pumping energy to push its building across the floor? \textbf{No.} For every bit of momentum Lab A gets from the substrate, Lab B gets equal and opposite momentum. The net is zero.

\emph{Analogy:} Two people on ice skates, connected by a rope. One pulls the rope (transferring energy). Both move toward each other (equal and opposite momentum). No net momentum is created.

\textbf{Experimental test:}\\
In experiment E2 (\S13), we verify this with precision force meters. If the theory is right, we should see:
\begin{itemize}
\item Energy transferred between A and B
\item But \textbf{zero net force} on the combined system A + B
\item The momentum budget balances locally at each site
\end{itemize}

This is one of the key testable predictions distinguishing our framework from ``magic'' or unphysical proposals.

\begin{center}\rule{0.5\linewidth}{0.5pt}\end{center}

\subsection{The $\alpha$-Factor: Why Gravity Stays Light-Speed}

\textbf{The Question:}\\
We've added new energy-momentum ($T^{\mu\nu}_S$) that sources gravity via Einstein's equations (3.3). Does this mean gravity could propagate faster than light?

\textbf{The Answer:}\\
No. The Bianchi identity (a fundamental geometric consistency requirement) forces:

\begin{equation}
\boxed{\;\alphaG\equiv 1\ \text{(exact)}\;}
\tag{3.8}
\end{equation}

This is not a choice we make, but a \textbf{mathematical necessity}.

\textbf{What $\alphaG$ means:}\\
$\alphaG$ is the ``gravitational coupling factor'' - the strength with which the substrate's energy-momentum $T^{\mu\nu}_S$ sources spacetime curvature.

\textbf{Why $\alphaG$ must equal 1:}

The mathematics requires this for consistency with the \textbf{Bianchi identity} - a fundamental geometric fact about spacetime curvature.

When we have:
\begin{itemize}
\item $\nabla_\mu T^{\mu\nu}_{vis} = -J^\nu_\sigma$ (matter loses energy-momentum)
\item $\nabla_\mu T^{\mu\nu}_S = +J^\nu_\sigma$ (substrate gains energy-momentum)
\end{itemize}

The Bianchi identity (a geometric necessity) requires:
\begin{itemize}
\item $\nabla_\mu G^{\mu\nu} = 0$ (a mathematical identity, true by definition)
\end{itemize}

For this to be consistent with Einstein's equations (3.3), we need:
\begin{itemize}
\item $G_{\mu\nu} = (8\pi G/c^4)(T_{vis} + \alpha T_S)$
\item $\nabla_\mu(T_{vis} + \alpha T_S)^{\mu\nu} = \nabla_\mu T^{\mu\nu}_{vis} + \alpha \nabla_\mu T^{\mu\nu}_S = -J^\nu_\sigma + \alpha J^\nu_\sigma$
\end{itemize}

For $\nabla_\mu(T_{vis} + \alpha T_S)^{\mu\nu} = 0$, we need:
\begin{itemize}
\item $-J^\nu_\sigma + \alpha J^\nu_\sigma = 0$
\item Therefore $\alpha = 1$
\end{itemize}

\textbf{The implication:}\\
Metric response (gravity, gravitational waves) is \textbf{light-speed causal}. All perceivable FTL comes \textbf{only} from S-locality, not from gravitational effects.

\emph{Mental model:} The ``screen'' (spacetime) still refreshes at light speed. The FTL communication happens ``through the computer's internal memory'' (the substrate), not ``on the screen'' (spacetime).

\textbf{Why this matters:}\\
It means we're not proposing that gravity travels faster than light, which would contradict a century of tests. Instead:
\begin{itemize}
\item Gravity: always light-speed ($\alpha = 1$)
\item Resonance: can be FTL in spacetime but is always retarded in substrate-time $T$
\end{itemize}

This keeps the framework compatible with gravitational wave observations (LIGO/Virgo), binary pulsar timing, and all other tests of general relativity.

\begin{center}\rule{0.5\linewidth}{0.5pt}\end{center}

