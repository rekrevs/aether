\section{Thermodynamics: You Must Pay to Play}

\begin{quote}
\textbf{Core Concept:} Aether resonance isn't free. It requires energy, and the energy budget obeys thermodynamic laws.
\end{quote}

\subsection{The Thermodynamic Puzzle}

If you can transfer information via substrate coupling, that's potentially useful work. But the Second Law of Thermodynamics says:

\textbf{You can't extract work from a system in thermal equilibrium.}

How do we reconcile aether resonance with thermodynamics?

\textbf{Answer:} Aether resonance requires \textbf{active driving} - you must pump energy into the system. The thermodynamic cost is paid at the pump.

\subsection{Pattern Free Energy}

We define:

\[
\mathcal{F}_S = \langle E_S \rangle - \Temp \, \Sigma_S
\]

\textbf{Components:}

\textbf{$\langle E_S \rangle$:} Average energy stored in the substrate configuration
\begin{itemize}
\item Quantum energy levels, coherence, etc.
\end{itemize}

\textbf{$\Sigma_S$:} Entropy of the pattern
\begin{itemize}
\item Approximated via \textbf{minimum description length} (MDL)
\item Or compression ratio: $\Sigma_S \sim$ (compressed size)/(raw size) $\times \log_2$(states)
\end{itemize}

\textbf{$\Temp$:} Temperature of the effective heat bath

\textbf{$\mathcal{F}_S$:} The free energy available to drive substrate flows
\begin{itemize}
\item Analogous to Gibbs or Helmholtz free energy in standard thermodynamics
\end{itemize}

\emph{Mental model:} Like a battery. $\langle E_S \rangle$ is the total charge, $\Sigma_S$ is the entropy (disorder), and $\mathcal{F}_S$ is the ``useful work'' available.

\subsection{Minimal Markov Model}

\textbf{Assumptions:}\\
Each active substrate edge $e$ is a two-state system:
\begin{itemize}
\item State 0: ``closed'' (no transfer)
\item State 1: ``open'' (allows transfer)
\end{itemize}

The edge couples to a heat bath at temperature $\Temp_{eff}$.

\textbf{Dynamics:}
\begin{itemize}
\item Pumping: $0 \to 1$ at rate $k_+$ (costs energy $\hbar\omega_0$)
\item Relaxation: $1 \to 0$ at rate $k_-$ (releases energy to bath)
\end{itemize}

\textbf{Detailed balance:}\\
$k_+/k_- = \exp[-\beta \Delta F_e]$

where $\beta = 1/(k_B \Temp_{eff})$ and $\Delta F_e$ is the free-energy difference.

\textbf{This gives:}
\begin{itemize}
\item Stationary distribution $p_e^*$
\item Entropy production rate $\dot{S}_{tot} \geq 0$ (Second Law satisfied)
\end{itemize}

\textbf{Why this matters:}\\
It connects abstract substrate coupling to concrete thermodynamic processes you can model and measure.

\subsection{The Resource Inequality}

From the Second Law (via KL divergence), we derive:

\begin{equation}
\langle W_{pump} \rangle \geq k_B \Temp \, (\Delta \Sigma_S + I_{transferred}),
\tag{8.1}
\end{equation}

\textbf{Translation:}

\textbf{$\langle W_{pump} \rangle$:} Average work supplied by the pump
\begin{itemize}
\item This is energy you put in (measurable)
\end{itemize}

\textbf{$k_B \Temp \Delta\Sigma_S$:} Thermodynamic cost of changing the substrate's entropy
\begin{itemize}
\item Maintaining a non-equilibrium configuration costs energy
\end{itemize}

\textbf{$k_B \Temp I_{transferred}$:} Cost of information transfer
\begin{itemize}
\item Information is physical (Landauer's principle)
\item Each bit of information transferred costs at least $k_B \Temp \ln(2)$ of work
\end{itemize}

\textbf{The inequality says:}\\
You must supply at least this much work to:
\begin{enumerate}
\item Drive the substrate into a special configuration (non-equilibrium)
\item Transfer $I$ bits of information
\end{enumerate}

\textbf{Proposition 8.1 (Explicit Bound for Two-State Edge):}

For each edge $e$ and measurement window $\Delta t$:

\begin{equation}
\langle W_{pump}(e)\rangle \;\ge\; k_B \Temp_{\rm eff}\,D_{\rm KL}\!\big(\mathbb{P}_{\rm driv}\Vert \mathbb{P}_{\rm eq}\big)
\;\ge\; k_B \Temp_{\rm eff}\,\ln 2\cdot I_e,
\tag{8.1'}
\end{equation}

\textbf{Components:}

\textbf{$D_{KL}$:} Kullback-Leibler divergence (relative entropy)
\begin{itemize}
\item Measures ``how different is the driven process from equilibrium?''
\end{itemize}

\textbf{$I_e$:} Information transferred through edge $e$ (in bits)

\textbf{The chain:}\\
$W_{pump} \geq$ (KL divergence) $\geq$ (information $\times k_B \Temp \ln 2$)

This makes the thermodynamic cost explicit and testable.

\textbf{Coupling to the Rate:}

\begin{equation}
\tilde{\mathcal{K}}(e,t)=\frac{P_{\rm pump}(e)}{\hbar\omega_0}\quad[\mathrm{s^{-1}}],
\tag{8.2}
\end{equation}

\textbf{$\Krate$:} The rate of pumping quanta
\begin{itemize}
\item Dimension: $s^{-1}$ (events per second)
\item Directly related to measurable pump power $P_{pump}$
\end{itemize}

The Markov model ties this to microscopic rates $k_+$ and $k_-$, making the connection to statistical mechanics explicit.

\subsection{Bitrate Bound for Experiment E1}

Combining (6.1) and (8.2), we get a fundamental limit:

\begin{equation}
R_{bit} \leq \beta \, \frac{P_{pump}}{k_B \Temp \ln 2} \, \mathcal{Q} \, e^{-d_\sigma/\lambda_\sigma}, \qquad 0 < \beta \leq 1.
\tag{8.3}
\end{equation}

\textbf{Translation:}

\textbf{$R_{bit}$:} Information transfer rate (bits per second)

\textbf{$\beta$:} Efficiency factor ($0 < \beta \leq 1$)
\begin{itemize}
\item How much of the supplied power goes into actual information transfer
\item vs. dissipation, noise, overhead
\item Realistically $\beta \sim 0.1$-0.5
\end{itemize}

\textbf{$P_{pump}/(k_B \Temp \ln 2)$:} Maximum possible rate given available power
\begin{itemize}
\item If $\Temp = 300$ K (room temperature): $k_B \Temp \approx 4 \times 10^{-21}$ J
\item If $P_{pump} = 1$ $\mu$W: Max rate $\sim 10^{15}$ bits/s (without other factors)
\end{itemize}

\textbf{$\mathcal{Q} \times \exp[-d_\sigma/\lambda_\sigma]$:} Resonance suppression
\begin{itemize}
\item This is the killer
\item $\mathcal{Q} \sim 10^{-10}$ to $10^{-2}$ (very small)
\item $\exp[-d_\sigma/\lambda_\sigma] \leq 1$ (less than 1 unless perfect match)
\item Product: $\sim 10^{-10}$ to $10^{-2}$
\end{itemize}

\textbf{Realistic estimate:}
\begin{itemize}
\item $P_{pump} = 1$ $\mu$W
\item $\Temp = 300$ K
\item $\beta = 0.1$
\item $\mathcal{Q} = 10^{-8}$
\item $\Ksig = 0.5$ (partial match)
\end{itemize}

$R_{bit} \leq 0.1 \times (10^{-6}$ W$)/(4\times10^{-21}$ J $\times 0.7) \times 10^{-8} \times 0.5$\\
$\approx 2 \times 10^{-2}$ bits/s

So under optimistic conditions, maybe \textbf{one bit per 50 seconds}.


\noindent\textbf{Worked Example: Bitrate Calculation Arithmetic}

\textbf{Given parameters:}
\begin{itemize}
\item Temperature: $T = 300$ K (room temperature)
\item Pump power: $P_{\rm pump} = 1\,\mu{\rm W} = 10^{-6}$ W
\item Efficiency: $\beta = 0.1$ (10\% of pump power goes to information transfer)
\item Quality factor: $\mathcal{Q} = 10^{-8}$
\item Kernel strength: $\Ksig = 0.5$ (partial structural match)
\end{itemize}

\textbf{Step 1: Calculate thermal energy}
\[
k_B T = (1.381 \times 10^{-23}\,{\rm J/K}) \times (300\,{\rm K}) = 4.14 \times 10^{-21}\,{\rm J}.
\]

\textbf{Step 2: Calculate maximum rate without suppression}

Landauer's principle: minimum energy per bit is $k_B T \ln(2)$:
\[
k_B T \ln(2) = 4.14 \times 10^{-21}\,{\rm J} \times 0.693 \approx 2.87 \times 10^{-21}\,{\rm J/bit}.
\]

Maximum rate from available power:
\[
R_{\rm max} = \frac{P_{\rm pump}}{k_B T \ln(2)} = \frac{10^{-6}\,{\rm W}}{2.87 \times 10^{-21}\,{\rm J}} \approx 3.5 \times 10^{14}\,{\rm bits/s}.
\]

\textbf{Step 3: Apply efficiency and resonance suppression}

Using equation (8.3):
\begin{align*}
R_{\rm bit} &\leq \beta \times R_{\rm max} \times \mathcal{Q} \times \Ksig \\
           &= 0.1 \times 3.5 \times 10^{14} \times 10^{-8} \times 0.5 \\
           &= 0.1 \times 3.5 \times 0.5 \times 10^{6} \\
           &= 1.75 \times 10^{5}\,{\rm bits/s} \times 10^{-8} \\
           &\approx 1.75 \times 10^{-3}\,{\rm bits/s}.
\end{align*}

Wait, let me recalculate more carefully:
\begin{align*}
R_{\rm bit} &= 0.1 \times \frac{10^{-6}}{4.14 \times 10^{-21} \times 0.693} \times 10^{-8} \times 0.5 \\
           &= 0.1 \times \frac{10^{-6}}{2.87 \times 10^{-21}} \times 10^{-8} \times 0.5 \\
           &= 0.1 \times 3.48 \times 10^{14} \times 10^{-8} \times 0.5 \\
           &= 0.1 \times 3.48 \times 10^{6} \times 0.5 \\
           &= 1.74 \times 10^{5} \times 10^{-1} \\
           &\approx 1.7 \times 10^{-2}\,{\rm bits/s}.
\end{align*}

\textbf{Result:} $R_{\rm bit} \approx 0.02$ bits/s, or \textbf{one bit every 50 seconds}.

\textbf{Key insight:} The $\mathcal{Q} \times \Ksig \sim 10^{-8} \times 0.5 = 5 \times 10^{-9}$ suppression factor is devastating. Even with 1 $\mu$W of power (which could theoretically support $10^{14}$ bits/s), the resonance coupling reduces this to barely detectable levels.

\medskip

\textbf{Null Result $\to$ Parameter Bound:}

If experiment E1 finds no signal above noise floor $R_{bit}^{(null)}$:

\begin{equation}
\mathcal{Q} \, e^{-d_\sigma/\lambda_\sigma} < \frac{k_B \Temp \ln 2}{\beta \, P_{pump}} \cdot R_{bit}^{(null)}.
\tag{8.4}
\end{equation}

Example:
\begin{itemize}
\item $R_{bit}^{(null)} = 10^{-6}$ bits/s (detection limit)
\item $P_{pump} = 1$ $\mu$W
\item $\beta = 0.1$
\item $\Temp = 300$ K
\end{itemize}

Then:\\
$\mathcal{Q} \times \Ksig < (4\times10^{-21} \times 0.7)/(0.1 \times 10^{-6}) \times 10^{-6} \approx 3 \times 10^{-20}$

\textbf{This directly constrains the product $\mathcal{Q}\Ksig$}, which we can then disentangle using the distance ladder (\S7).

\begin{center}\rule{0.5\linewidth}{0.5pt}\end{center}

