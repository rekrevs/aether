\section{Conclusion}

We have presented a consistent, falsifiable hypothesis for aether resonance as structure-local FTL transfer in a discrete substrate. The framework unifies:

\begin{enumerate}
  \item \textbf{Action-level formulation} (Sec.~3.1) with a well-defined $T^S_{\mu\nu}$ obtained by variation of the total action, including the $S$-mediator and interaction terms.

  \item \textbf{Covariant energy-momentum accounting} (Sec.~3.2) with split conservation laws
  \[
    \nabla_\mu T^{\mu\nu}_{\mathrm{vis}} = -J^\nu_\sigma,
    \qquad
    \nabla_\mu T^{\mu\nu}_S = +J^\nu_\sigma,
  \]
  and a light-speed causal metric response governed by standard GR.

  \item \textbf{$\alpha$-constraint} (Sec.~\ref{subsec:alpha-constraint}): a proof that the would-be gravitational coupling factor $\alpha$ multiplying $T^S_{\mu\nu}$ in the Einstein equations is fixed to $\alpha \equiv 1$ by the Bianchi identity in the presence of $J^\nu_\sigma$. There is no tunable reduced gravitational coupling; FTL resides entirely in $S$-locality.

  \item \textbf{Localization of $S$-flows} (Sec.~3.3) as compactly supported sources in $M$, preserving diffeomorphism invariance and ensuring well-defined pushforward from $S$ to $M$.

  \item \textbf{Momentum-neutrality} (Sec.~3.4) preventing reactionless drives and enforcing vanishing net spatial momentum exchange in closed systems.

  \item \textbf{Length units for $d_\sigma$ and $\lambda_\sigma$} (Secs.~2 and~7) via an embedding scale $\ell_0$ and an operationalized distance ladder.

  \item \textbf{Local $S$-mediator implementation} (Sec.~4) of structural proximity, with finite propagation speed $c_S$ and an exponentially decaying kernel $K_\sigma$.

  \item \textbf{Selection operator} (Sec.~5) that explains absence in the standard sector via degeneracy dilution and state-dependent suppression in homogeneous or thermal configurations.

  \item \textbf{Dimensionally consistent coupling law} (Sec.~6) for the substrate power $\Psig$, with clear separation between the similarity factor $K_\sigma$, the quality factor $Q$, and the pump-rate $\Gamma_{\rm pump}$.

  \item \textbf{Operationalized $d_\sigma$-metric and distance ladder} (Sec.~7), providing a practical surrogate for algorithmic similarity and a calibration procedure directly linked to experimental observables.

  \item \textbf{Thermodynamic resource bounds} (Sec.~8) with an explicit bitrate inequality that ties FTL signalling capacity to entropy production and pump power.

  \item \textbf{Modified Lieb–Robinson bound} with explicit conditions (Sec.~9, Lemma~9.1), quantifying how microcausality is softened but not destroyed, and how substrate-mediated effects are retarded and exponentially suppressed in $d_\sigma$.

  \item \textbf{Formal causality proof} (Sec.~10) via category theory, using a monotonic substrate time functor and an anti-telephone rule to exclude closed causal loops.

  \item \textbf{Anisotropy budget} (Sec.~11) with quantitative mapping to SME coefficients and explicit bounds connecting $(\varepsilon,\lambda_\sigma,Q)$ to photon- and matter-sector observations.

  \item \textbf{Numerical predictions} (Sec.~12) with concrete target scenarios (E1–E3) and end-to-end parameter maps specifying which observables constrain which parameter combinations.

  \item \textbf{No-loophole experiments} (Sec.~13) with pre-registration, cryptographic commit–reveal, spacelike separation, environmental vetoing, and multiple-test correction, together with a clear pipeline from raw data to parameter bounds.
\end{enumerate}

Taken together, these elements make the proposal simultaneously \emph{internally coherent} at the level of effective field theory and \emph{externally testable} in realistic experimental settings. The outcome of well-designed tests will be one of two informative possibilities:
\begin{equation}
  \text{either}
  \quad
  \begin{cases}
    \varepsilon \lambda_\sigma Q \;<\; 10^{-12}\,\mathrm{m},\\[4pt]
    \varepsilon \omega_0 Q \;<\; 10^{-8}\,\mathrm{Hz},\\[4pt]
    \varepsilon Q \;<\; 10^{-20},
  \end{cases}
  \qquad
  \text{or a reproducible non-local effect is observed.}
  \tag{16.1}
\end{equation}

In the first case, sharp upper bounds would significantly constrain any future substrate-local FTL models and motivate symmetry principles explaining why $\varepsilon$ and typical $Q$ values are so small. In the second case, a positive signal would open a new empirical and theoretical domain: a class of controlled, reproducible non-local effects with a clear link between microphysics (structure in $S$) and macrophysics (observables in $M$). Either way, the framework provides a well-defined target for theory, experiment, and simulation as we explore how far emergent spacetime can be pushed before it gives way to a deeper substrate.
