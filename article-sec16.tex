\section{Conclusion}

We have presented a consistent, falsifiable hypothesis for aether resonance as structure-local
FTL transfer in a discrete substrate. The framework brings together:

\begin{enumerate}
\item An action-level formulation (Sec.~3) with a well-defined $T^S_{\mu\nu}$, split conservation
   laws $\nabla_\mu T^{\mu\nu}_{\rm vis} = -J^\nu_\sigma$, $\nabla_\mu T^{\mu\nu}_S = +J^\nu_\sigma$,
   and a proof that the gravitational coupling to $T^S_{\mu\nu}$ is fixed to $\alpha\equiv1$
   by the Bianchi identity in the presence of exchange.

\item A local implementation of structural proximity via an $S$-mediator field $\chi(\sigma,T)$
   that propagates at finite speed $c_S$ and generates a positive, exponentially decaying
   kernel $K_\sigma = \exp[-d_\sigma/\lambda_\sigma]$ on pattern space (Secs.~4,~F).

\item A selection operator $O_S$ that is built from higher-dimension, gauge- and
   diffeomorphism-invariant seed operators, is technically natural in the Wilsonian sense,
   and is dynamically suppressed in homogeneous or thermal states by degeneracy dilution,
   while allowing $Q=O(1)$ in engineered, pattern-rich configurations (Secs.~5--6,~I).

\item An operationalized structural distance $d_\sigma$ and distance ladder that tie the abstract
   pattern space $(S,d_\sigma)$ to experimentally accessible feature maps and deformation
   protocols (Sec.~7).

\item A thermodynamic resource inequality linking any FTL bit rate to entropy production and
   pump power, providing a Landauer-type upper bound on signalling capacity through the
   substrate channel (Sec.~8).

\item A modified Lieb--Robinson bound and a categorical causality proof showing that, given a
   global substrate time $\tau$, the model admits superluminal transfers in $M$ but excludes
   exact instantaneous commutators and antitelephone-type paradoxes (Secs.~9--10).

\item An anisotropy budget and SME mapping that translate the parameters
   $(\varepsilon,\lambda_\sigma,Q)$ into effective photon- and matter-sector coefficients
   constrained by existing Lorentz-violation searches (Sec.~11, App.~H).

\item Concrete, ``no-loophole'' experimental designs and numerical targets (E1--E3) that specify
   which parameter combinations are probed by which observables, and how null results or
   positive signals should be reported (Secs.~12--13).
\end{enumerate}

Taken together, these elements make the proposal simultaneously internally coherent at the
level of effective field theory and externally testable in realistic experimental settings. For
the benchmark sensitivities discussed in Sec.~12, null results in all three classes of experiment
would translate, up to order-of-magnitude factors, into bounds of the schematic form
\begin{equation}
\varepsilon \lambda_\sigma Q \lesssim 10^{-12}\,{\rm m},\qquad
\varepsilon \omega_0 Q \lesssim 10^{-8}\,{\rm Hz},\qquad
\varepsilon Q \lesssim 10^{-20},
\label{eq:bound-schematic}
\tag{16.1}
\end{equation}
for the range of platforms considered. Conversely, a reproducible nonlocal effect consistent
with the pre-registered analyses and control tests would open a new empirical and theoretical
domain: a class of controlled, pattern-dependent superluminal channels with a clear link between
structure in $S$ and observables in $M$. Either outcome would be informative: strong null
bounds would constrain any future substrate-local FTL models to a very small corner of
parameter space and motivate symmetry principles explaining why $\varepsilon$ and typical
$Q$ values are so small, while a positive signal would demand a systematic extension of both
effective field theory and experimental practice in the direction outlined here.

Finally, it bears repeating that the proposal is deliberately modest in scope. We have not
attempted to explain how spacetime itself emerges from a discrete substrate, nor to derive
general relativity or the standard model from microscopic rules. Instead, we have shown that,
if such an emergent background exists and if it is coupled in the very specific way analysed here
to a discrete pattern space with a global time ordering, then there is a well-defined regime in
which pattern-local, apparently superluminal transfer is compatible with effective field theory,
with thermodynamics, and with a sharpened notion of causality, and in which the resulting
phenomenology can be probed concretely in the laboratory. Whether nature in fact uses anything
like this mechanism is an empirical question; the point of this work is to make that question as
precise and answerable as possible.

\section*{Acknowledgements of AI assistance}
This manuscript was prepared with the assistance of AI tools (notably OpenAI’s ChatGPT (versions 5 and 5.1 in Pro mode), Google's Gemini (version 3 in Pro mode), and Anthropic’s Claude Code). The human author provided the fundamental idea for the paper, which the AI tools helped concretize into physical theory and the present text in an iterative process of generation and reviewing. The human author is not a physicist.
