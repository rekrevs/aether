\section{Introduction}
Our starting point in this paper is deliberately modest. We take as given the empirical success
of the standard effective field theory description of nature: quantum fields propagating on an
emergent Lorentzian manifold $(M,g)$ that to very high precision obeys general relativity at
macroscopic scales. We do \emph{not} attempt to derive this continuum description from a
microscopic model, nor to unify quantum field theory and gravity. Instead, we ask a narrower,
conditional question:
\begin{quote}
Given such an emergent relativistic background, is there a consistent way to couple it to an
underlying discrete substrate with a global time ordering, such that the substrate can mediate
pattern-dependent, apparently superluminal transfer of information and energy, without
introducing causal paradoxes or obvious conflicts with existing bounds on Lorentz violation?
\end{quote}

The rest of the paper answers this question in the affirmative for a particular class of
effective models, and develops their thermodynamic and experimental consequences. Our aim
is not to replace general relativity or the standard model, but to exhibit an \emph{existence
proof} at the level of effective field theory: a concrete, falsifiable class of models in which
(i) all observed low-energy physics remains approximately local and Lorentz invariant in $M$,
(ii) a discrete substrate with global time underlies the dynamics, and (iii) certain highly
structured configurations can excite a ``pattern-local'' channel in an abstract pattern space $S$
that projects as faster-than-light transfer in $M$. The construction is conservative in that it is
formulated as an ordinary action, respects the Bianchi identity and stress--energy conservation,
and can be parametrised in the same language as the Standard-Model Extension (SME).

\subsection{Relation to existing approaches}
\label{subsec:related-work}

Several existing frameworks introduce preferred structures, small departures from exact local Lorentz invariance, or an underlying discrete substrate from which spacetime emerges. Here we position the present proposal with respect to the most relevant lines of work and highlight what is genuinely new.

\paragraph{Discrete and emergent spacetime.}
Approaches such as causal set theory, causal dynamical triangulations (CDT), and quantum graphity start from a fundamentally discrete structure---a locally finite partial order, ensembles of triangulations, or dynamical graphs---from which continuum spacetime and locality emerge in a coarse-grained limit~\cite{surya2019_cst,loll2019_cdt,konopka2008_graphity}. These models show that discreteness can coexist with approximate Lorentz invariance and with nonlocal microscopic structure. Our framework is agnostic about the microscopic details, but shares the basic premise that the observed manifold $(M,g)$ need not be fundamental. The distinctive feature here is the introduction of a \emph{second} proximity structure $(S,d_\sigma)$, together with a weak, pattern-dependent coupling that is strictly local in $S$ yet can redistribute energy and information superluminally in $M$ while remaining compatible with a globally well-defined substrate time.

\paragraph{Cellular and quantum cellular automata substrates.}
Discrete-time dynamics with a global update ordering are familiar from cellular automata and from quantum cellular automata (QCA), which provide general frameworks for local deterministic and local unitary evolution on discrete lattices and graphs~\cite{thooft2016_ca,farrelly2020_qca}. Such models can reproduce relativistic wave equations and even full quantum field theories in appropriate continuum limits. Our postulate P1 (discrete dynamics and global ordering) is in this spirit: we assume that there \emph{exists} some CA/QCA-like substrate that gives rise to standard QFT on $(M,g)$ at low energies, but we do not attempt to specify it. Instead, we add an additional structure---the pattern space $S$, a computable metric $d_\sigma$, and a substrate-local coupling in $S$---and explore the consequences of this extra channel. In this sense aether resonance is ``QCA + an additional, structure-selective channel'', rather than a new microscopic model of the substrate itself.

\paragraph{Einstein--\ae ther and khronometric gravity.}
Einstein--\ae ther and khronometric models introduce a unit timelike vector field $u^\mu$ that picks out a preferred foliation while keeping the gravitational sector close to GR~\cite{jacobson2004_einstein_aether,jacobson2008_einstein_aether_status,horava2009_qg_lifshitz,blas2011_horava_models}. These theories propagate additional modes and are tightly constrained by post-Newtonian tests, binary pulsars, and by the near-equality of the speed of gravity and the speed of light inferred from GW170817 and GRB 170817A~\cite{yagi2014_einstein_aether_pulsars,oost2018_einstein_aether_gw170817,baker2017_gw170817_mg,creminelli2017_dark_energy_gw170817,will2014_confrontation_gr}. Our $L_S$ in Eqs.~(3.1A)–(3.1C) is chosen from this well-studied class, but we adopt a \emph{constraint-only, minimal khronon} regime: the unit timelike vector is enforced by a Lagrange multiplier, we set $c_T=c$, and we do not allow new propagating metric degrees of freedom. As a result, the strong constraints on modified gravity are automatically respected. The preferred structure that is phenomenologically relevant in our framework lives primarily in the pattern space $S$ and in the selection operator $O_S$, which together generate an additional channel for energy/information transfer rather than a simple modification of dispersion relations.

\paragraph{Standard-Model Extension and Lorentz-violation searches.}
The Standard-Model Extension (SME) provides a systematic parametrization of small Lorentz-violating coefficients in photon and matter sectors~\cite{kostelecky2002_photon_sme,kostelecky2009_photon_nonminimal}. Modern resonant-cavity and clock-comparison experiments constrain these coefficients at the level of $\Delta\nu/\nu \sim 10^{-18}$ and below~\cite{eisele2009_light_isotropy,nagel2015_lorentz10minus18,kostelecky2018_clock_tests}. Sections~11 and Appendix~\ref{app:sme-coupling} explicitly borrow the SME language: we define an ``anisotropy budget'' and translate aether-resonance effects into effective photon-sector coefficients $\tilde{\kappa}^{e-}_{JK}$. In contrast to a generic SME analysis, however, these coefficients are not free parameters: in our model they are tied to a specific mechanism (substrate locality in $S$ combined with $O_S$ and the kernel $K_\sigma$), and to concrete experiments (E1–E3) with pre-specified parameter maps. Null results in those experiments therefore become bounds on tightly constrained combinations of SME coefficients rather than isolated numbers.

\paragraph{Locality bounds, Bell nonlocality, and fast-light analogues.}
In standard local quantum lattice systems, the Lieb–Robinson bound quantifies an emergent light cone for the propagation of operator commutators~\cite{lieb1972_lr}. For systems with long-range interactions this is generalized to ``soft'' or distorted light cones whose shape depends on the interaction tail~\cite{fossfeig2015_longrange_lr,tran2020_hierarchy_lightcones}. In quantum information, Bell nonlocality and related no-signalling theorems show how nonlocal correlations can coexist with relativistic causality~\cite{brunner2014_bell_nonlocality}, while fast-light experiments demonstrate that superluminal group velocities do not imply superluminal information transfer~\cite{stenner2003_fast_light}. Our modified Lieb–Robinson bound (Sec.~9) and the categorical causality proof (Sec.~10) are best viewed as substrate-level analogues of these results: individual influence steps can be superluminal in the emergent spacetime $M$, but every allowed influence chain is monotone in the substrate time $\tau$, and the effective commutators outside the usual light cone are exponentially suppressed in $d_\sigma$ and retarded at finite mediator speed $c_S$.

\paragraph{Thermodynamics of information and pattern-space methods.}
The thermodynamic resource inequality in Sec.~8 rests on Landauer's principle---the $k_B T \ln 2$ cost of reliably erasing a bit---and on its experimental confirmations in single-bit systems~\cite{landauer1961_irreversibility,berut2012_landauer_exp,jun2014_landauer_precision}. We do not propose any violation of this principle; instead, we derive a bound that says that any sustained FTL bitrate through the $S$-channel must pay a conventional Landauer cost multiplied by an efficiency factor involving the pattern quality factor $Q$ and the similarity kernel $K_\sigma$. The operational pattern metric $d_\sigma$ is in turn built from gauge- and diffeomorphism-invariant features, including topological signatures (persistent homology) and dynamical response features that can be extracted using techniques from topological data analysis and physical reservoir computing~\cite{carlsson2009_topology_data,tanaka2019_physical_reservoir,vandersande2017_photonic_reservoir}. In this way the abstract notion of ``algorithmic similarity'' is connected to concrete, calibration-ready tools.

\paragraph{Conventional FTL geometries in general relativity.}
There is a long-standing literature on effective FTL travel within GR based on exotic spacetime geometries such as Alcubierre's warp drive and traversable wormholes~\cite{alcubierre1994_warp,morris1988_wormholes}. These scenarios keep local microphysics unchanged but rely on large curvature and violations or saturation of energy conditions. By contrast, the present proposal keeps GR unchanged in the metric sector and instead introduces a new, pattern-selective channel that is local in $S$ and nonlocal in $M$. From the point of view of observational tests, this means that our framework should be constrained alongside SME coefficients and long-baseline Lorentz tests, rather than through the viability of exotic stress–energy tensors.

The present framework differs from these approaches in three conceptually distinct ways. First, we introduce an explicit pattern space $S$ with metric $d_\sigma$ that controls a substrate-local but spacetime-nonlocal communication channel; this goes beyond the usual implementations of a preferred frame or modified dispersion relations. Second, we derive a resource inequality and a modified Lieb--Robinson-type bound for this channel, thereby quantifying both its thermodynamic cost and its effective causal cone. Third, we connect the formalism to concrete experimental protocols (E1--E3) and give explicit parameter maps from their observable sensitivities to bounds on the underlying pattern-coupling parameters $(\varepsilon,\lambda_\sigma,Q,\omega_0)$.

A structural consequence of our assumptions is that the $S$-sector must couple to gravity with the same strength as visible matter. In particular, Sec.~\ref{subsec:alpha-constraint} shows that, given our split conservation law and the Bianchi identity, the relative coupling parameter $\alpha$ is not a tunable free parameter but is fixed to $\alpha \equiv 1$ whenever $J^\nu_\sigma \neq 0$.

\subsection{Notation and key parameters (quick reference)}

For convenience we summarize the most frequently used symbols; Appendix~A gives a more
complete list.

\begin{itemize}
  \item $M$: emergent spacetime manifold with metric $g_{\mu\nu}$ and light speed $c$.
  \item $S$: pattern space; $d_\sigma(\sigma,\sigma')$ is a structural distance with units of
        length, and $\lambda_\sigma$ is the associated structural range.
  \item $\tau$: coarse-grained substrate time function; strictly increasing along all allowed
        microscopic transitions.
  \item $O_S$: pattern-selection operator constructed from higher-dimension seed operators
        with $[\!O_S\!]=4$ after normalization; couples the visible sector to the substrate.
  \item $\varepsilon$: dimensionless microscopic coupling between visible fields and the
        $S$-sector.
  \item $T^{\mu\nu}_{\rm vis}$, $T^{\mu\nu}_S$: stress--energy tensors for visible and
        $S$-sector degrees of freedom, with split conservation
        $\nabla_\mu T^{\mu\nu}_{\rm vis} = -J^\nu_\sigma$ and
        $\nabla_\mu T^{\mu\nu}_S = +J^\nu_\sigma$.
  \item $Q$: pattern quality factor $0\le Q\le 1$, measuring how well a configuration
        matches the pattern selected by $O_S$; $Q\to 0$ in homogeneous or thermal states
        by degeneracy dilution.
  \item $K_\sigma(\sigma,\sigma') = \exp[-d_\sigma(\sigma,\sigma')/\lambda_\sigma]$:
        positive, exponentially decaying similarity kernel on $S$ induced by the mediator.
  \item $P_\sigma$: net power into the substrate channel along an active edge, parametrized
        phenomenologically via Eq.~(6.1).
\end{itemize}

\subsection{Scope of this work and what we do not attempt}
\label{subsec:scope}

It is important to be explicit about the domain of validity of the framework and about what
problems we do not claim to solve.

First, we assume from the outset that there already exists an emergent, approximately continuous
spacetime $(M,g)$ on which low-energy physics is well described by local quantum fields and
general relativity. In other words, the continuum description is taken as an empirical input and
as the background of our construction. We do not attempt in this paper to derive $(M,g)$ from a
specific discrete microdynamics, nor to explain why the continuum description works as well as
it does over many orders of magnitude in scale.

Second, we do not propose a new route to unification of quantum field theory and gravity, nor do
we modify the Einstein--Hilbert term in the action. The gravitational sector is the usual one, and
the consistency result $\alpha\equiv 1$ is precisely the statement that once energy--momentum
exchange with the substrate is allowed, the substrate stress--energy must couple to gravity in
the same way as ordinary matter.

Within this restricted scope, our focus is the following conditional question:
\begin{quote}
If there exists a discrete substrate endowed with a global time function $\tau$ and local dynamics
in a pattern space $(S,d_\sigma)$ satisfying the assumptions listed in Appendix~\ref{app:assumptions},
what are the generic effective consequences for low-energy physics on $(M,g)$, and how can
they be tested?
\end{quote}

This leads naturally to three more concrete questions that often arise when one first encounters
the framework:

\begin{enumerate}
\item \emph{Why does the world look continuous if there is a discrete substrate?}\\
      In our treatment this is not derived but assumed: we work ``downstream'' of whatever
      mechanism produces an emergent Lorentzian manifold, and we only analyse additional
      effects coming from a weak coupling to $S$. All constructions in the paper are formulated
      as deformations of an already approximately continuous effective field theory on $(M,g)$.

\item \emph{At what scales does the model predict departures from a purely continuous description?}\\
      In homogeneous or near-thermal states the expectation value of the selection operator
      $O_S$ is driven to zero by degeneracy dilution, so the substrate channel is effectively
      inactive and the model reduces to ordinary local EFT, up to SME-suppressed corrections.
      Observable departures are expected only in carefully engineered, pattern-rich configurations
      where the quality factor $Q$ approaches order one, and even there the strength of the
      effect is controlled by the small dimensionless coupling $\varepsilon$ and the structural
      range $\lambda_\sigma$.

\item \emph{What would actually be different or observable compared with standard local EFT?}\\
      At the level of coarse-grained observables the framework predicts, in principle, (i)
      pattern-dependent, apparently superluminal transfer between suitably prepared systems,
      subject to the thermodynamic and causality bounds derived in Secs.~8 and~10; and (ii)
      small anisotropies and dispersion effects that can be parameterised by photon- and
      matter-sector coefficients in the Standard-Model Extension, as discussed in Sec.~11. The
      experimental designs in Secs.~12--13 are built precisely to look for such departures.
\end{enumerate}

By making these restrictions and questions explicit we hope to avoid over-claiming: the proposal
should be read as an effective and testable ``existence proof'' for one narrow class of
substrate-local FTL models, not as a general theory of quantum gravity or of continuum
emergence. At the same time, the framework is not a single fine-tuned model. In the body of the paper we alternate between making specific, simple choices (for the aether/khronon sector, the mediator dynamics on $S$, the selection operator $O_S$, and the operational metric $d_\sigma$) and identifying the structural properties that any such choice must satisfy if it is to remain compatible with general relativity, local EFT on $(M,g)$, and a globally monotone substrate time.
