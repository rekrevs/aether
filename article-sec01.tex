\section{Introduction and Motivation}

Relativity and quantum mechanics provide a consistent, locally causal description of all known phenomena over many orders of magnitude. At the same time, they leave open the question of whether spacetime and its light-cone structure are fundamental, or whether both might instead emerge from some deeper, discrete substrate. In this work we investigate the speculative but internally consistent hypothesis that:

\begin{enumerate}
\item Observable spacetime $(M)$ with light speed $(c)$ arises as an effective, coarse-grained description of a discrete substrate with global update ordering $(T=0,1,2,\ldots)$.
\item There exists a second notion of distance --- \textbf{structural proximity} in a pattern space $(\Sspace,\Dsig)$ --- in which two local substrate configurations are ``close'' if they are algorithmically isomorphic.
\item A weak, substrate-local coupling in $(\Sspace)$ --- \textbf{aether resonance} --- allows energy and information to be redistributed ``in place'' in $(\Sspace)$, which is experienced as FTL in $(M)$.
\end{enumerate}

The question is whether this can be made \textbf{physically coherent}: compatible with conservation laws, with the Equivalence Principle and observed Lorentz symmetry, with quantum mechanics and Bell tests, with known high-energy and astrophysical constraints, and with existing no-go theorems on FTL signalling. We aim to show that this is possible in a tightly constrained way that yields concrete, conservative experimental results, and \textbf{falsifiable consequences}.

Throughout this work we treat the visible sector as standard quantum
field theory on $(M,g)$, with local Hamiltonian dynamics and the usual
Born-rule interpretation.  The substrate and the $S$-mediator are
introduced at an effective level as additional degrees of freedom that
mediate a pattern-dependent channel for energy and information
transfer.  We do not attempt to quantize the substrate explicitly; for
the purposes of the modified Lieb--Robinson bound and the causality
proof it is sufficient that the combined dynamics be generated by a
Hamiltonian on a Hilbert space and that the $S$-sector obey the
retardation, sparsity, and cost assumptions spelled out in
Appendix~\ref{app:assumptions}.  The proposed experiments are therefore tests of an
\emph{extra} channel on top of standard quantum theory in the visible
sector, rather than modifications of the linear structure of quantum
mechanics itself.

\subsection{Relation to existing approaches}
\label{subsec:related-work}

Several existing frameworks introduce preferred structures or small
departures from exact local Lorentz invariance.  Here we briefly
position the present proposal with respect to some of the most
relevant lines of work.

\paragraph{Einstein--\ae ther and khronometric gravity.}
Einstein--\ae ther and khronometric models introduce a unit timelike
vector field $u^\mu$ that picks out a preferred foliation while
keeping the gravitational sector close to GR.  Our $L_S$ in
Eqs.~(3.1A)--(3.1C) is chosen precisely from this well-studied class,
and we adopt a conservative parameter regime in which $c_T=c$ and
existing bounds on preferred-frame effects are respected.  However,
aether resonance differs qualitatively from these models in that the
preferred structure lives primarily in the pattern space $S$ and in
the selection operator $O_S$, which generate an additional channel for
energy/information transfer rather than a simple modification of
dispersion relations.

\paragraph{Standard-Model Extension and Lorentz-violation searches.}
The Standard-Model Extension (SME) provides a systematic parametriza-
tion of small Lorentz-violating coefficients in particle and photon
sectors.  Section~11 and Appendix~\ref{app:sme-coupling} borrow this language to define an
``anisotropy budget'' and to translate aether-resonance effects into
effective photon-sector coefficients $\tilde\kappa^{e-}_{JK}$.  In
contrast to a generic SME analysis, however, the present framework
ties those effective coefficients to a specific mechanism---substrate
locality in $S$ combined with the pattern operator $O_S$ and the
kernel $K_\sigma$---and to concrete experiments (E1--E3) with
pre-specified parameter maps.

\paragraph{Nonlocal and emergent-gravity models.}
There is a broad literature on nonlocal field theories, analogue
gravity, and emergent gravity scenarios in which spacetime and its
metric arise as coarse-grained descriptors of more microscopic
dynamics.  The present work is agnostic about the detailed
microphysics, but shares with these approaches the idea that $(M,g)$
is not fundamental.  The distinctive feature here is the introduction
of a second proximity structure $(S,d_\sigma)$ and a weak,
pattern-dependent coupling that is \emph{local} in~$S$ yet can
redistribute energy and information superluminally in~$M$ while
remaining compatible with a globally well-defined substrate time.

\paragraph{Preferred-frame collapse and superluminal signalling
proposals.}
Various proposals have explored faster-than-light influences in
quantum mechanics by modifying the collapse rules or by postulating
hidden superluminal channels.  Many of these run into the standard
no-go theorems for frame-independent signalling or require highly
nonlinear dynamics in Hilbert space.  By contrast, the present
framework keeps the visible-sector quantum dynamics linear and
local on~$M$, introduces a preferred substrate frame explicitly, and
enforces a monotonic substrate time~$\tau$.  The combination of the
modified Lieb--Robinson bound and the categorical causality proof
(Secs.~9--10) then ensures that no antitelephone protocols are
possible even though individual influence steps can be superluminal in
the emergent spacetime.
