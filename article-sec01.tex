\section{Introduction and Motivation}

Relativity and quantum mechanics provide a consistent, locally causal description of all known phenomena over many orders of magnitude. At the same time, they leave open the question of whether spacetime and its light-cone structure are fundamental, or whether both might instead emerge from some deeper, discrete substrate. In this work we investigate the speculative but internally consistent hypothesis that:

\begin{enumerate}
\item Observable spacetime $(M)$ with light speed $(c)$ arises as an effective, coarse-grained description of a discrete substrate with global update ordering $(T=0,1,2,\ldots)$.
\item There exists a second notion of distance --- \textbf{structural proximity} in a pattern space $(\Sspace,\Dsig)$ --- in which two local substrate configurations are ``close'' if they are algorithmically isomorphic.
\item A weak, substrate-local coupling in $(\Sspace)$ --- \textbf{aether resonance} --- allows energy and information to be redistributed ``in place'' in $(\Sspace)$, which is experienced as FTL in $(M)$.
\end{enumerate}

The question is whether this can be made \textbf{physically coherent}: compatible with conservation laws, with the Equivalence Principle and observed Lorentz symmetry, with quantum mechanics and Bell tests, with known high-energy and astrophysical constraints, and with existing no-go theorems on FTL signalling. We aim to show that this is possible in a tightly constrained way that yields concrete, conservative experimental results, and \textbf{falsifiable consequences}.
