\section{Compatibility with Experiments: Why Haven't We Seen This Yet?}

\begin{quote}
\textbf{Core Concept:} If this exists, why don't we see it everywhere? Answer: The preferred frame creates incredibly tiny anisotropies - so small they're at the edge of detectability.
\end{quote}

\subsection{The Preferred Frame Problem}

\textbf{The issue:}\\
We've introduced:
\begin{itemize}
\item A foliation scalar $\tau$ (universal clock)
\item A preferred time direction $u^\mu$
\item Substrate structure with its own geometry
\end{itemize}

This \textbf{breaks Lorentz invariance} - the principle that physics looks the same in all reference frames moving at constant velocity.

\textbf{But wait!}\\
Lorentz invariance has been tested to incredible precision:
\begin{itemize}
\item Michelson-Morley experiment (1887): $\Delta c/c < 10^{-8}$
\item Modern versions: $\Delta c/c < 10^{-18}$
\end{itemize}

How can we have a preferred frame without violating these tests?

\textbf{Answer:} The preferred frame causes \textbf{anisotropy} (direction-dependence) so weak it's barely detectable.

\subsection{Derivation of Anisotropy}

From our Lagrangian (5.1), with a preferred frame $\xi^\mu = (1, 0, 0, 0)$ in substrate rest, we get modifications to the \textbf{dispersion relation}:

\begin{equation}
    E^2 = p^2 c^2 + m^2 c^4\,\big[1+\Delta(E,\hat p\!\cdot\!\hat\xi)\big],
    \tag{11.1}
\end{equation}

where $\Delta$ is a \textbf{dimensionless modifier}:

\begin{equation}
    \Delta \;\sim\; \varepsilon\left(\frac{\lambda_\sigma}{\lambda_C}\right)\mathcal Q
    \left(\frac{E}{mc^2}\right)\,(\hat p\!\cdot\!\hat\xi)^2.
    \tag{11.2}
\end{equation}

\textbf{Translation:}

\textbf{Dispersion relation:} The equation linking energy $E$, momentum $p$, and mass $m$
\begin{itemize}
\item Standard relativity: $E^2 = p^2c^2 + m^2c^4$
\item Modified: $E^2 = p^2c^2 + m^2c^4[1 + \Delta]$ where $\Delta$ is a tiny dimensionless correction
\end{itemize}

\textbf{The dimensionless modifier $\Delta$:}
\begin{itemize}
\item Depends on the \textbf{direction} of momentum $\hat{p}$ relative to preferred frame $\hat{\xi}$
\item Proportional to $\varepsilon$ (coupling), $(\lambda_\sigma/\lambda_C)$ (length scale ratio), $\mathcal{Q}$ (coherence)
\item Goes as $(E/mc^2)$ (energy dependence) and $(\hat{p}\cdot\hat{\xi})^2$ (direction dependence)
\item Being dimensionless, it multiplies the mass term $m^2c^4$ correctly
\end{itemize}

\textbf{For photons ($m = 0$):}\\
The correction causes an effective \textbf{velocity variation}:

\begin{equation}
    \frac{\Delta c}{c} \;\sim\; \epsgam\left(\frac{\lsig}{\lambda_C}\right)\Qgam.
    \tag{11.3}
\end{equation}

where $\epsgam$ is the photon-sector coupling, $\lsig$ is the substrate coherence length, and $\Qgam$ is the photon-sector quality factor.

\textbf{Plug in numbers:}
\begin{itemize}
\item $\epsgam \sim 10^{-15}$
\item $\lsig \sim 1$ $\mu$m $= 10^{-6}$ m
\item $\lambda_C \sim 10^{-12}$ m (Compton wavelength)
\item $\lsig/\lambda_C \sim 10^6$
\item $\Qgam \sim 10^{-24}$ to $10^{-18}$ (depending on scenario)
\end{itemize>

Result:\\
$\Delta c/c \sim 10^{-15} \times 10^6 \times 10^{-24}$ to $10^{-18}$\\
$\sim 10^{-33}$ to $10^{-27}$

\textbf{This is way below current detection limits!}

\textbf{Current best bounds:}
\begin{itemize}
\item Michelson-Morley type: $\Delta c/c \lesssim 10^{-18}$
\item Hughes-Drever type: $\Delta c/c \lesssim 10^{-27}$ (nuclear clock experiments)
\end{itemize}

\textbf{Our constraint:}

\begin{equation}
    \epsgam\left(\frac{\lsig}{\lambda_C}\right)\Qgam \;\lesssim\; 10^{-18}.
    \tag{11.4}
\end{equation}

With $\lsig/\lambda_C \sim 10^6$, this gives:

$\epsgam \cdot \Qgam \lesssim 10^{-24}$

\textbf{This is why we don't see it in normal experiments!}

The photon-sector coupling $\epsgam$ is already tiny ($\sim 10^{-15}$). The coherence $\Qgam$ is tinier still ($10^{-10}$ to $10^{-2}$ at best). Their product must satisfy (11.4), which is incredibly restrictive.

\subsection{Sidereal and Annual Modulation}

\textbf{The smoking gun:}\\
If there's a preferred frame, Earth is moving through it. As Earth rotates daily and orbits the Sun annually, our velocity relative to the preferred frame changes.

This should cause \textbf{modulation} - a rhythmic variation with:
\begin{itemize}
\item \textbf{Sidereal period:} 23 hours, 56 minutes, 4 seconds (one rotation relative to distant stars)
\item \textbf{Annual period:} 365.25 days (one orbit around the Sun)
\end{itemize}

\textbf{The formula (detailed in phenomenology section, Eq. (12.4)):}

For the \textbf{matter sector}:

\[
A_{sid}^{(\rm mat)} \simeq \epsmat \left( \frac{\lsig}{L_{exp}} \right) \Qmat \, \Xi
\]

\textbf{Components:}

\textbf{$A_{sid}^{(\rm mat)}$:} Modulation amplitude (matter sector)
\begin{itemize}
\item Dimensionless fractional variation
\item Example: $A_{sid} = 10^{-20}$ means a $10^{-20}$ fractional change
\end{itemize}

\textbf{$\epsmat$, $\Qmat$:} Matter-sector coupling and quality factor
\begin{itemize}
\item Distinct from photon-sector values $\epsgam$, $\Qgam$
\item Matter experiments probe different physics than optical tests
\end{itemize}

\textbf{$L_{exp}$:} Apparatus scale
\begin{itemize}
\item Length of the experimental setup
\item Larger apparatus $\to$ smaller fractional effect
\end{itemize}

\textbf{$\Xi$:} Geometry factor
\begin{itemize}
\item Order 1 ($\Xi \sim 1$)
\item Depends on apparatus orientation and configuration
\end{itemize}

\textbf{Important distinction:}\\
The photon-sector bounds (11.4) constrain $\epsgam \cdot \Qgam$. The matter-sector modulation (above) involves $\epsmat \cdot \Qmat$. These are \textbf{different parameters} - photon and matter sectors can have different coupling strengths.

\textbf{Numerical target:}

For E2 (rotation test in \S13):

\[
A_{\rm sid}\gtrsim 10^{-20}\ \text{(3$\sigma$ significance over }10^7\text{ seconds)}
\]

\textbf{What this means:}\\
After integrating data for $\sim$115 days ($10^7$ seconds), we aim to detect a modulation with amplitude $\sim 10^{-20}$ at 3-sigma confidence (99.7\% certainty).

\textbf{Example:}\\
If measuring energy transfer between two cavities:
\begin{itemize}
\item Baseline transfer: $J_0 = 1 \times 10^{-30}$ W
\item Modulated transfer: $J(t) = J_0 (1 + 10^{-20} \cos(2\pi t/T_{sid}))$
\item Peak-to-peak variation: $2 \times 10^{-50}$ W
\end{itemize}

This is absurdly small - but potentially measurable with:
\begin{itemize}
\item Cryo-calorimetry (mK temperatures)
\item Long integration times
\item Careful systematic control
\end{itemize}

\textbf{Why this is testable:}
\begin{itemize}
\item Sidereal period is precisely known (astronomy)
\item Distinct from solar day (24 hours) or monthly/annual cycles
\item Hard to fake with systematic errors (which usually follow solar time or lab rhythms)
\end{itemize}

\textbf{Null result $\to$ Parameter bound:}\\
If we don't see modulation above $A_{sid}^{(\rm mat)} < 10^{-20}$, then:

$\epsmat \cdot (\lsig/L_{exp}) \cdot \Qmat < 10^{-20}$

With $L_{exp} \sim 1$ meter and $\lsig \sim 10^{-6}$ m:

$\epsmat \cdot \Qmat < 10^{-14}$

This would constrain the matter-sector parameter space significantly.

\subsection{SME Parametrization (For Comparison)}

\textbf{What is SME?}\\
The \textbf{Standard Model Extension} (SME) is a systematic framework for testing Lorentz invariance violations. It parametrizes all possible small violations using coefficients.

We report photon-sector coefficients $\tilde\kappa_{e-}^{JK}$ in the standard Sun-centered celestial-equatorial frame (SCCEF, $J,K\in\{X,Y,Z\}$), where $\tilde\kappa_{e-}^{JK}$ is real, symmetric, and traceless, and $(\Delta c/c)(\hat n)\simeq \tfrac12\,\hat n_J\hat n_K\,\tilde\kappa_{e-}^{JK}$.

For \textbf{photons}, the minimal SME uses coefficients:
\begin{itemize}
\item $\tilde{\kappa}_{e-}^{JK}$ (even parity, electric type)
\item $\tilde{\kappa}_{o+}^{JK}$ (odd parity, magnetic type)
\item $\kappa_{tr}$ (trace)
\end{itemize}

\textbf{Our model predicts:}

\begin{equation}
\tilde{\kappa}_{e-}^{JK}\ \sim\ \epsgam\,\Qgam\,
\Big(\frac{\lsig}{L_{\rm exp}}\Big)\,\Xi^{JK}.
\tag{11.5}
\end{equation}

\textbf{Translation:}\\
The SME coefficients, which are directly measured in precision tests (optical resonators, atomic clocks), are related to our parameters.

\textbf{Why this matters:}
\begin{itemize}
\item It lets us compare to existing experimental bounds
\item Provides a common language with the Lorentz-violation community
\item Allows model-independent reporting
\end{itemize}

\textbf{Practical advice:}\\
When reporting sidereal modulation results, also estimate $|\tilde{\kappa}_{e-}^{JK}|$ with error bars and link to (11.4). This enables direct comparison with resonator and atomic clock studies.

\begin{tcolorbox}[colback=yellow!5!white,colframe=yellow!75!black,title=\textbf{SME Connection: Why This Matters for Experiments}]
\textbf{What is the Standard Model Extension (SME)?}

The SME is a comprehensive framework for testing violations of Lorentz invariance (the principle that physics looks the same in all reference frames). It's used by experimentalists worldwide to:
\begin{itemize}[leftmargin=*,noitemsep,topsep=3pt]
\item Report bounds from precision tests (atomic clocks, resonators, particle physics)
\item Compare results across different experimental techniques
\item Provide a common language for Lorentz violation searches
\end{itemize}

\textbf{How our theory connects:}

The preferred frame in our substrate creates \textbf{direction-dependent effects}:
\begin{itemize}[leftmargin=*,noitemsep,topsep=3pt]
\item Light traveling parallel vs. perpendicular to the preferred direction has slightly different speeds
\item The difference goes as $(\hat{p} \cdot \hat{\xi})^2$ - quadratic in the angle
\item This shows up as anisotropy in the speed of light: $\Delta c/c \sim (\text{direction})^2$
\end{itemize}

\textbf{The SME parameters:}

For photons, the minimal SME uses coefficients $\tilde{\kappa}_{e-}^{JK}$ (a 3$\times$3 symmetric, traceless tensor) that describe direction-dependent corrections to the speed of light.

Our model predicts:
\[
\tilde{\kappa}_{e-}^{JK} \sim \varepsilon_\gamma \mathcal{Q}_\gamma \left(\frac{\lambda_\sigma}{L_{\rm exp}}\right) \Xi^{JK}
\]

where:
\begin{itemize}[leftmargin=*,noitemsep,topsep=3pt]
\item $\varepsilon_\gamma$ = photon-sector coupling (dimensionless)
\item $\mathcal{Q}_\gamma$ = photon-sector quality factor (dimensionless)
\item $\lambda_\sigma/L_{\rm exp}$ = ratio of coherence length to apparatus size
\item $\Xi^{JK}$ = geometry factor (depends on apparatus orientation)
\end{itemize}

\textbf{Current experimental bounds:}

The best constraints on $\tilde{\kappa}_{e-}^{JK}$ come from:
\begin{itemize}[leftmargin=*,noitemsep,topsep=3pt]
\item Optical resonators (cavities that measure speed of light): $|\tilde{\kappa}| \lesssim 10^{-18}$
\item Atomic clocks: $|\tilde{\kappa}| \lesssim 10^{-20}$
\item Astrophysical observations: $|\tilde{\kappa}| \lesssim 10^{-32}$ (model-dependent)
\end{itemize}

These bounds constrain: $\varepsilon_\gamma \mathcal{Q}_\gamma (\lambda_\sigma/L_{\rm exp}) \lesssim 10^{-18}$ to $10^{-20}$.

\textbf{Why this is useful:}

\begin{enumerate}[leftmargin=*,noitemsep,topsep=3pt]
\item \textbf{Connects to established literature:} Our parameters map directly to SME coefficients that thousands of experiments have measured.
\item \textbf{Cross-checks our parameter space:} Existing bounds immediately constrain our model.
\item \textbf{Enables predictions:} We can predict what SME parameters should be measured in future experiments.
\item \textbf{Sector separation:} Different particles (photons, electrons, nucleons) couple differently, allowing independent tests.
\end{enumerate}

\textbf{Bottom line:} Our theory isn't isolated. It makes contact with a vast experimental program searching for Lorentz violation. The SME provides the bridge.
\end{tcolorbox}

\begin{center}\rule{0.5\linewidth}{0.5pt}\end{center}

