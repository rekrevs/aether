\section{Action and Field Content}

\subsection{Variational Principle}

We derive the framework from a \textbf{total action}:

\begin{equation}
\begin{split}
S_{tot} = \int d^4x \, \sqrt{-g} \left[ \frac{1}{16\pi G} R
+ \mathcal{L}_{vis}[\phi, g] + \mathcal{L}_S[\tau, u^\mu, g] \right] \\
 + \varepsilon \int d^4x \sqrt{-g}\!\int\! d\mu(\sigma)\,d\mu(\sigma')\,
\frac{O_S(x,\sigma)\,\Kern(\sigma,\sigma')\,O_S(x,\sigma')}{\LamUV^{4}},
\end{split}
\tag{3.1}
\end{equation}

where:

\begin{itemize}
\item $\mathcal{L}_{vis}$ collects all visible matter and gauge fields.
\item $\mathcal{L}_S$ is a minimal aether/khronon sector that allows for a preferred foliation.
\item The interaction term (last line) encodes the substrate-local coupling in $(\Sspace)$ via a \textbf{bilocal} operator product $O_S(x,\sigma) \cdot \Kern(\sigma,\sigma') \cdot O_S(x,\sigma')$ with a structural kernel $\Kern$ on $(\Sspace,\Dsig)$ and a selection operator $O_S$. The measure $d\mu(\sigma)$ is dimensionless; all mass dimensions are carried by $O_S$ and the explicit scale $\LamUV$.
\end{itemize}

\subsection{Aether / Khronon Sector}

We choose a simple baseline aether/khronon sector that is compatible with existing constraints. The general structure of such preferred-frame sectors is well known from Einstein--\ae ther and Ho\v{r}ava/khronometric gravity, where a unit timelike vector field selects a foliation and leads to additional propagating modes and preferred-frame effects~\cite{jacobson2004_einstein_aether,jacobson2008_einstein_aether_status,horava2009_qg_lifshitz,blas2011_horava_models}. A conservative option is to take a unit timelike vector field $u^\mu$ enforced by a Lagrange multiplier:
\begin{equation}
\mathcal{L}_S^{(A)} = - \frac{M_S^2}{2}\,\lambda_L(x)\,\big(u^\mu u_\mu + 1\big),
\tag{3.1A}
\end{equation}
where the unit timelike vector $u^\mu$ is introduced via a Lagrange multiplier $\lambda_L(x)$. This defines a preferred foliation without adding new propagating degrees of freedom.

A more general option, closer to Einstein--æther / khronometric gravity, is
\begin{equation}
\begin{split}
\mathcal{L}_S^{(B)} = -\frac{M_S^2}{2} \Big[
 & c_1 (\nabla_\mu u_\nu)(\nabla^\mu u^\nu)
 + c_2 (\nabla_\mu u^\mu)^2
 + c_3 (\nabla_\mu u_\nu)(\nabla^\nu u^\mu) \\
 & + c_4 u^\mu u^\nu (\nabla_\mu u_\alpha)(\nabla_\nu u^\alpha)
 + \lambda_L(x)\,(u^\mu u_\mu+1).
\Big]
\end{split}
\tag{3.1B}
\end{equation}

We choose the parameter regime
\begin{equation}
c_{13}:=c_1+c_3=0\quad(\Rightarrow c_T=c),
\qquad
c_2=0,
\qquad
c_4\ll 1,
\tag{3.1C}
\end{equation}
so that gravitational-wave speed $c_T$ matches $c$ to high precision and the preferred frame effects are within current bounds. This parameter choice lies within the region allowed by binary pulsar timing, post-Newtonian constraints, and the near-equality of the speed of gravity and the speed of light inferred from GW170817 and GRB 170817A~\cite{yagi2014_einstein_aether_pulsars,oost2018_einstein_aether_gw170817,baker2017_gw170817_mg,creminelli2017_dark_energy_gw170817,will2014_confrontation_gr}. For the bulk of this paper, option (A) will suffice; (B) simply provides a conservative embedding into the broader Einstein--æther literature and leaves room to maneuver if one later wishes to let $u^\mu$ carry weak, long-wavelength structure. In the bulk of this work we further specialise to the constraint-only option (3.1A), so that no new propagating metric degrees of freedom appear and all observable effects arise from the $M\times S$ interaction.

\subsection{Field Equations and Energy-Momentum Accounting}

Varying with respect to $g_{\mu\nu}$, the visible fields, and the aether/khronon sector yields:

\begin{enumerate}
\item Einstein equations with source
\begin{equation}
G_{\mu\nu} = \frac{8\pi G}{c^4}
\Bigl( T^{vis}_{\mu\nu} + \TS \Bigr),
\tag{3.2}
\end{equation}
where $\TS$ includes the aether/khronon contribution and any effective stress-energy induced by the $S$-mediator.

\item A split conservation law
\begin{equation}
\nabla_\mu T^{\mu\nu}_{vis} = -J^\nu_\sigma,
\qquad
\nabla_\mu T^{\mu\nu}_S = +J^\nu_\sigma,
\tag{3.4}
\end{equation}
so that
\begin{equation}
\nabla_\mu (T^{\mu\nu}_{vis} + T^{\mu\nu}_S) = 0.
\tag{3.5}
\end{equation}
\end{enumerate}

\subsubsection{Why $\alpha=1$ is Required by Consistency}
\label{subsec:alpha-constraint}

In a more general parametrization one might try to write
\[
G_{\mu\nu} = \frac{8\pi G}{c^4}\bigl(T^{vis}_{\mu\nu} + \alpha\,T^S_{\mu\nu}\bigr),
\]
with a dimensionless factor $\alpha$ controlling the gravitational response to the $S$-sector. However, this freedom is illusory in the presence of the exchange current $J^\nu_\sigma$.

By metric compatibility, the Einstein tensor satisfies
\[
\nabla_\mu G^{\mu\nu} = 0.
\]

If we write the Einstein equations as
\[
G_{\mu\nu} = \frac{8\pi G}{c^4}\bigl(T^{vis}_{\mu\nu} + \alpha\,T^S_{\mu\nu}\bigr),
\]
then taking the covariant divergence gives
\[
0 = \nabla_\mu G^{\mu\nu}
 = \frac{8\pi G}{c^4}
\bigl(\nabla_\mu T^{\mu\nu}_{vis} + \alpha\,\nabla_\mu T^{\mu\nu}_S\bigr).
\]

Using the split conservation equations (3.4), we have $\nabla_\mu T^{\mu\nu}_{vis} = -J^\nu_\sigma$ and $\nabla_\mu T^{\mu\nu}_S = +J^\nu_\sigma$. Substituting, one finds
\[
0 = \frac{8\pi G}{c^4}(\alpha-1) J^\nu_\sigma.
\]

For generic configurations where $J^\nu_\sigma \neq 0$, this forces $\boxed{\alpha = 1}$ exactly. Thus any apparent freedom to choose a different gravitational coupling for the $S$-sector is removed once we demand consistency with the Bianchi identity and the explicit exchange current: the metric responds to \emph{all} energy-momentum with the same gravitational strength. This is a structural consistency condition, not a tunable parameter, and should be verified experimentally in E2 (see Assumption~A8 in Appendix~\ref{app:assumptions}).

\subsection{Gravitational Signature and Bounds}

\textbf{Gravitational coupling $\alpha$.} As shown in the preceding subsection (``Why $\alpha=1$ is required by consistency''), once we split the conservation laws as
\[
\nabla_\mu T^{\mu\nu}_{\rm vis} = -J^\nu_\sigma,
\qquad
\nabla_\mu T^{\mu\nu}_S = +J^\nu_\sigma,
\]
the Bianchi identity $\nabla_\mu G^{\mu\nu}=0$ forces the Einstein equations written as
\[
G_{\mu\nu} = \frac{8\pi G}{c^4}\bigl(T^{\rm vis}_{\mu\nu} + \alpha\,T^S_{\mu\nu}\bigr)
\]
to have a \emph{unique} universal coupling:
\begin{equation}
\boxed{\;\alpha\equiv 1\ \text{(exact)}\;}
\end{equation}
whenever $J^\nu_\sigma\neq 0$. In other words, the metric responds to \emph{all} stress--energy with the same strength; there is no consistent way to tune a ``reduced gravitational coupling'' for the $S$-sector.

This has two direct implications for the phenomenology:
\begin{enumerate}
\item The causal structure of $(M,g)$ remains light-speed: null cones are defined by $c$, and gravitational disturbances propagate at $c$ (up to the small adjustments already allowed by the chosen aether/khronon sector). Any apparent superluminal behaviour arises entirely from substrate-local propagation in $S$ combined with the projection $\pi:S\to M$, not from a modified gravitational constant.
\item Any observable gravitational signature of aether resonance must therefore come from the \emph{amount} and \emph{pattern} of $T^S_{\mu\nu}$ excited in resonant configurations, not from changing $\alpha$. In the rest of the paper all constraints and experimental targets are expressed in terms of bounds on the coupling $\varepsilon$, the range $\lambda_\sigma$, the quality factor $Q$, and the effective $T^S_{\mu\nu}$ generated by $O_S$, with $\alpha$ fixed to~1.
\end{enumerate}
