\section{Action and Field Content}
\label{sec:action}

\subsection{Variational Principle}
\label{sec:action:variational}

From this point on we work with one explicit effective-field-theory realization of the postulates in Sec.~2. The goal is not to single out a unique microphysics, but to have a concrete, internally consistent model that makes all assumptions visible and calculable. The specific choices made below for the aether/khronon sector, for the interaction term, and for the mediator and selection operators are representative of a broader family. Whenever a later argument relies only on the structural content of P1--P4 and the assumptions (A1)--(A9), rather than on the detailed form of these choices, we will say so explicitly.

We derive the framework from a total action defined on the extended configuration space $M \times S$. We postulate the existence of a dynamical mediator field $\chi(\sigma, \tau)$ living on the substrate, which couples locally to the visible sector. The total action is:

\begin{equation}
S_{tot} = S_{grav} + S_{vis} + S_{aether} + S_{med} + S_{int}
\tag{3.1}
\end{equation}

where the standard sectors are:

\begin{equation}
S_{grav} + S_{vis} + S_{aether} = \int d^4x \sqrt{-g} \left( \frac{R}{16\pi G} + \mathcal{L}_{vis} + \mathcal{L}_{S}[\tau, u^\mu] \right)
\end{equation}

The substrate-mediator sector $S_{med}$ governs the dynamics of the scalar field $\chi$ on the pattern space $S$ with respect to substrate time $T$ (or its smooth approximation $\tau$):

\begin{equation}
S_{med} = -\frac{1}{2} \int dT \int d\mu(\sigma) \left[ (\partial_T \chi)^2 + \chi (-\hat{\mathcal{O}}_\sigma) \chi \right]
\end{equation}

where $\hat{\mathcal{O}}_\sigma = -c_S^2 \nabla_\sigma^2 + m_\chi^2$ is the elliptic operator defining the structural metric and range (see Sec.~4).

Crucially, the interaction is \textbf{local} in the extended bundle $M \times S$. The visible sector couples to the mediator via a source current $J_S$:

\begin{equation}
S_{int} = \int d^4x \sqrt{-g} \, Q_*^2(x) \int d\mu(\sigma) J_S(x, \sigma) \chi(\sigma, \tau(x))
\end{equation}

Here, $Q_*(x)$ is a dimensionless, coarse-grained background scalar field (a "spurion") that encodes pattern activation. This spurion acts as a switch: $Q_*(x) \approx 0$ in homogeneous vacuum or thermal states, and $Q_*(x) \sim \mathcal{Q}$ (the pattern quality factor) in engineered, pattern-rich configurations. The quadratic dependence $Q_*^2$ ensures that single-spurion insertions are forbidden, providing a structural sequestering of the S-sector from the vacuum (see Sec.~5.6).

\emph{Interpretation of $Q_*$.}---Throughout we treat $Q_*(x)$ as a spurion in the Wilsonian sense: a coarse-grained, non-dynamical field that tracks the degree of pattern activation in a given region, rather than as an independent propagating degree of freedom. Microscopic substrate variables that actually carry pattern information are assumed to average, under coarse-graining, to an effective scalar $Q_*$ that vanishes in homogeneous or thermal states and approaches the quality factor $Q$ in deliberately engineered configurations. The quadratic dependence in Eq.~(3) implements the ``pattern parity'' assumption made explicit in Sec.~5.6: single insertions of $Q_*$ are forbidden, so that any Lorentz-violating operator induced in the visible sector carries at least two powers of $Q_*$ and therefore disappears continuously as $Q_* \to 0$. A UV completion would have to explain this structure dynamically; here we use it as an effective, symmetry-based parametrization of pattern sequestering.

Varying this action with respect to the metric yields field equations involving the stress--energy tensor $T^{\mu\nu}_S$. Here $T^{\mu\nu}_S$ includes both the stress--energy associated with the aether/khronon sector defined by $\mathcal{L}_S$ and the effective stress--energy of the $S$-mediator and its interaction terms; the detailed decomposition is discussed in Appendix~\ref{app:bilocal-kernel}.

\subsection{Aether / Khronon Sector}
\label{sec:action:aether}

We choose a simple baseline aether/khronon sector that is compatible with existing constraints. The general structure of such preferred-frame sectors is well known from Einstein--\ae ther and Ho\v{r}ava/khronometric gravity, where a unit timelike vector field selects a foliation and leads to additional propagating modes and preferred-frame effects~\cite{jacobson2004_einstein_aether,jacobson2008_einstein_aether_status,horava2009_qg_lifshitz,blas2011_horava_models}. A conservative option is to take a unit timelike vector field $u^\mu$ enforced by a Lagrange multiplier:
\begin{equation}
\mathcal{L}_S^{(A)} = - \frac{M_S^2}{2}\,\lambda_L(x)\,\big(u^\mu u_\mu + 1\big),
\tag{3.1A}
\end{equation}
where the unit timelike vector $u^\mu$ is introduced via a Lagrange multiplier $\lambda_L(x)$. This defines a preferred foliation without adding new propagating degrees of freedom.

A more general option, closer to Einstein--æther / khronometric gravity, is
\begin{equation}
\begin{split}
\mathcal{L}_S^{(B)} = -\frac{M_S^2}{2} \Big[
 & c_1 (\nabla_\mu u_\nu)(\nabla^\mu u^\nu)
 + c_2 (\nabla_\mu u^\mu)^2
 + c_3 (\nabla_\mu u_\nu)(\nabla^\nu u^\mu) \\
 & + c_4 u^\mu u^\nu (\nabla_\mu u_\alpha)(\nabla_\nu u^\alpha)
 + \lambda_L(x)\,(u^\mu u_\mu+1).
\Big]
\end{split}
\tag{3.1B}
\end{equation}

We choose the parameter regime
\begin{equation}
c_{13}:=c_1+c_3=0\quad(\Rightarrow c_T=c),
\qquad
c_2=0,
\qquad
c_4\ll 1,
\tag{3.1C}
\end{equation}
so that gravitational-wave speed $c_T$ matches $c$ to high precision and the preferred frame effects are within current bounds. This parameter choice lies within the region allowed by binary pulsar timing, post-Newtonian constraints, and the near-equality of the speed of gravity and the speed of light inferred from GW170817 and GRB 170817A~\cite{yagi2014_einstein_aether_pulsars,oost2018_einstein_aether_gw170817,baker2017_gw170817_mg,creminelli2017_dark_energy_gw170817,will2014_confrontation_gr}. For the bulk of this paper, option (A) will suffice; (B) simply provides a conservative embedding into the broader Einstein--æther literature and leaves room to maneuver if one later wishes to let $u^\mu$ carry weak, long-wavelength structure. In the bulk of this work we further specialise to the constraint-only option (3.1A), so that no new propagating metric degrees of freedom appear and all observable effects arise from the $M\times S$ interaction. None of the structural consistency results in Secs.~\ref{sec:action:field-eqs}, \ref{sec:lieb-robinson}, and~\ref{sec:causality} depend on the detailed form of $L_S$; they use only the existence of a preferred foliation compatible with P1 and (A6), so the analysis applies equally to any aether/khronon sector in that parameter regime.

\subsection{Field Equations and Energy-Momentum Accounting}
\label{sec:action:field-eqs}

Varying with respect to $g_{\mu\nu}$, the visible fields, and the aether/khronon sector yields:

\begin{enumerate}
\item Einstein equations with source
\begin{equation}
G_{\mu\nu} = \frac{8\pi G}{c^4}
\Bigl( T^{vis}_{\mu\nu} + \TS \Bigr),
\tag{3.2}
\end{equation}
where $\TS$ represents the total stress-energy of the substrate sector, including the aether/khronon contribution $T_{aether}^{\mu\nu}$ and the contribution from the mediator field and interaction $T_{\chi}^{\mu\nu}$.

\item The interaction term $S_{int}$ involves both metric and substrate variables. Variation with respect to $g_{\mu\nu}$ generates the necessary exchange terms. The split conservation laws (3.4-3.5) follow from the diffeomorphism invariance of the coupled local action:
\begin{equation}
\nabla_\mu T^{\mu\nu}_{vis} = -J^\nu_\sigma,
\qquad
\nabla_\mu T^{\mu\nu}_S = +J^\nu_\sigma,
\tag{3.4}
\end{equation}
where the exchange current $J^\nu_\sigma$ arises from the energy-momentum transfer between the visible fields and the mediator $\chi$.
so that
\begin{equation}
\nabla_\mu (T^{\mu\nu}_{vis} + T^{\mu\nu}_S) = 0.
\tag{3.5}
\end{equation}
\end{enumerate}

\subsubsection{Why $\alpha=1$ is Required by Consistency}
\label{subsec:alpha-constraint}

In a more general parametrization one might try to write
\[
G_{\mu\nu} = \frac{8\pi G}{c^4}\bigl(T^{vis}_{\mu\nu} + \alpha\,T^S_{\mu\nu}\bigr),
\]
with a dimensionless factor $\alpha$ controlling the gravitational response to the $S$-sector. However, this freedom is illusory in the presence of the exchange current $J^\nu_\sigma$.

By metric compatibility, the Einstein tensor satisfies
\[
\nabla_\mu G^{\mu\nu} = 0.
\]

If we write the Einstein equations as
\[
G_{\mu\nu} = \frac{8\pi G}{c^4}\bigl(T^{vis}_{\mu\nu} + \alpha\,T^S_{\mu\nu}\bigr),
\]
then taking the covariant divergence gives
\[
0 = \nabla_\mu G^{\mu\nu}
 = \frac{8\pi G}{c^4}
\bigl(\nabla_\mu T^{\mu\nu}_{vis} + \alpha\,\nabla_\mu T^{\mu\nu}_S\bigr).
\]

Using the split conservation equations (3.4), we have $\nabla_\mu T^{\mu\nu}_{vis} = -J^\nu_\sigma$ and $\nabla_\mu T^{\mu\nu}_S = +J^\nu_\sigma$. Substituting, one finds
\[
0 = \frac{8\pi G}{c^4}(\alpha-1) J^\nu_\sigma.
\]

For generic configurations where $J^\nu_\sigma \neq 0$, this forces $\boxed{\alpha = 1}$ exactly. Thus any apparent freedom to choose a different gravitational coupling for the $S$-sector is removed once we demand consistency with the Bianchi identity and the explicit exchange current: the metric responds to \emph{all} energy-momentum with the same gravitational strength. This is a structural consistency condition, not a tunable parameter, and should be verified experimentally in E2 (see Assumption~A8 in Appendix~\ref{app:assumptions}). Therefore, any viable completion of this scenario at the substrate level must reproduce $\alpha \equiv 1$ in the low-energy effective action: $\alpha$ cannot be treated as a tunable parameter that could be adjusted to ``hide'' the gravitational effects of the $S$-sector whenever $J^\nu_\sigma$ is nonzero.

\subsection{Gravitational Signature and Bounds}
\label{sec:action:grav-signature}

\textbf{Gravitational coupling $\alpha$.} As shown in the preceding subsection (``Why $\alpha=1$ is required by consistency''), once we split the conservation laws as
\[
\nabla_\mu T^{\mu\nu}_{\rm vis} = -J^\nu_\sigma,
\qquad
\nabla_\mu T^{\mu\nu}_S = +J^\nu_\sigma,
\]
the Bianchi identity $\nabla_\mu G^{\mu\nu}=0$ forces the Einstein equations written as
\[
G_{\mu\nu} = \frac{8\pi G}{c^4}\bigl(T^{\rm vis}_{\mu\nu} + \alpha\,T^S_{\mu\nu}\bigr)
\]
to have a \emph{unique} universal coupling:
\begin{equation}
\boxed{\;\alpha\equiv 1\ \text{(exact)}\;}
\end{equation}
whenever $J^\nu_\sigma\neq 0$. In other words, the metric responds to \emph{all} stress--energy with the same strength; there is no consistent way to tune a ``reduced gravitational coupling'' for the $S$-sector.

This has two direct implications for the phenomenology:
\begin{enumerate}
\item The causal structure of $(M,g)$ remains light-speed: null cones are defined by $c$, and gravitational disturbances propagate at $c$ (up to the small adjustments already allowed by the chosen aether/khronon sector). Any apparent superluminal behaviour arises entirely from substrate-local propagation in $S$ combined with the projection $\pi:S\to M$, not from a modified gravitational constant.
\item Any observable gravitational signature of aether resonance must therefore come from the \emph{amount} and \emph{pattern} of $T^S_{\mu\nu}$ excited in resonant configurations, not from changing $\alpha$. In the rest of the paper all constraints and experimental targets are expressed in terms of bounds on the coupling $\varepsilon$, the range $\lambda_\sigma$, the quality factor $Q$, and the effective $T^S_{\mu\nu}$ generated by $O_S$, with $\alpha$ fixed to~1.
\end{enumerate}
