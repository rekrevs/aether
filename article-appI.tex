\section{Toy Model for Degeneracy Dilution}
\label{app:degeneracy-toy-model}

We model the degeneracy-dilution mechanism using a discrete Hilbert space of pattern embeddings to demonstrate the $1/N$ scaling of $\mathcal{Q}$ in homogeneous states.

\subsection{Setup}
Consider an emergent region that can host a specific mesoscopic pattern $\mathcal{P}$ in $N$ macroscopically equivalent ways (embeddings), labeled by $j = 1, \dots, N$. Let $|j\rangle$ denote the coarse-grained state where the pattern is located at embedding $j$. We treat these as orthonormal, $\langle j | k \rangle = \delta_{jk}$.

We define the selection operator $\hat{O}_S$ to "reward" a specific target embedding $j_0$:
\begin{equation}
\hat{O}_S = O_0 |j_0\rangle \langle j_0|
\end{equation}

\subsection{Structured (Engineered) State}
In a perfectly engineered configuration where the pattern is deliberately pinned at $j_0$, the state is $\rho_{str} = |j_0\rangle \langle j_0|$.
\begin{equation}
\mathcal{Q}[\rho_{str}] = \frac{|\mathrm{Tr}(\rho_{str} \hat{O}_S)|}{O_0} = \frac{|\langle j_0 | O_0 | j_0 \rangle|}{O_0} = 1
\end{equation}
This confirms that $\mathcal{Q} \sim 1$ for engineered states.

\subsection{Homogeneous (Mixed) State}
A maximally homogeneous state has no preference for any specific embedding and is modeled as a uniform classical mixture:
\begin{equation}
\rho_{hom} = \frac{1}{N} \sum_{j=1}^N |j\rangle \langle j|
\end{equation}
The expectation value is:
\begin{equation}
\mathrm{Tr}(\rho_{hom} \hat{O}_S) = \frac{1}{N} \sum_{j=1}^N \langle j | \hat{O}_S | j \rangle = \frac{1}{N} \langle j_0 | \hat{O}_S | j_0 \rangle = \frac{O_0}{N}
\end{equation}
The quality factor is:
\begin{equation}
\mathcal{Q}[\rho_{hom}] = \frac{1}{N} \xrightarrow{N \to \infty} 0
\end{equation}

\subsection{Random-Phase Superposition}
Alternatively, if the homogeneous state is a quantum superposition with random phases $\phi_j$:
\begin{equation}
|\psi_{rand}\rangle = \frac{1}{\sqrt{N}} \sum_{j=1}^N e^{i\phi_j} |j\rangle
\end{equation}
Then $\langle \psi_{rand} | \hat{O}_S | \psi_{rand} \rangle = \frac{1}{N} \langle j_0 | \hat{O}_S | j_0 \rangle = \frac{O_0}{N}$.
Again, $\mathcal{Q} = 1/N$.

\subsection{Conclusion}
This model demonstrates that if a pattern has $N$ equivalent embeddings in a volume, the signal detected by a local selection operator scales as $1/N$. In the thermodynamic limit ($N \to \infty$), the S-channel is effectively closed ($\mathcal{Q} \to 0$) for homogeneous matter, while remaining open ($\mathcal{Q} \sim 1$) for engineered states.

\subsection{Example of a local pattern operator in a spin model}
To illustrate degeneracy dilution in a more physical setting, consider a one-dimensional spin-$1/2$ chain of length $N$ with local Pauli operators $\sigma^z_j$. Let the ``pattern'' be a specific length-3 domain $\uparrow\downarrow\uparrow$ and define a local projector at site $j$
\[
\Pi_j := \frac{1}{8}(1+\sigma^z_j)(1-\sigma^z_{j+1})(1+\sigma^z_{j+2}),
\]
which is 1 if the spins at $(j,j+1,j+2)$ form $\uparrow\downarrow\uparrow$ and 0 otherwise. A simple instance of $O_S$ in this toy model is
\[
\hat O_S = O_0\,\Pi_{j_0},
\]
where $j_0$ is a designated embedding and $O_0$ sets the normalization. In an engineered state where the pattern is deliberately pinned at $j_0$, the density matrix is $\rho_{\rm str}=|{\uparrow\downarrow\uparrow}\rangle_{j_0}\langle{\uparrow\downarrow\uparrow}|\otimes\rho_{\rm rest}$ and $Q[\rho_{\rm str}]=1$.

In contrast, consider a homogeneous thermal state at inverse temperature $\beta$ in a translation-invariant nearest-neighbour Hamiltonian. The coarse-grained state can be approximated by
\[
\rho_{\rm hom} \simeq \frac{1}{N}\sum_{j=1}^N \rho_j,
\]
where $\rho_j$ is the state with the pattern centred at $j$ and the rest of the chain in local equilibrium. By the same counting as in Sec.~5.2 and Appendix I.1--I.4, one finds
\[
\mathrm{Tr}(\rho_{\rm hom}\hat O_S) = \frac{O_0}{N},\qquad Q[\rho_{\rm hom}] \sim \frac{1}{N}\to 0\quad(N\to\infty).
\]
Thus a completely local operator built from a small cluster of spins can exhibit the same $1/N$ suppression in homogeneous states as the abstract embedding model. The essential ingredient is not the particular Hilbert space, but the large combinatorial degeneracy of macroscopically equivalent placements of the pattern. Similar constructions can be carried out in higher-dimensional spin systems or lattice boson models, and a key task for future work is to identify microscopic substrates where high-$Q$ engineered states and strong degeneracy dilution can coexist for realistic values of $N$.