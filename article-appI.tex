\section{Toy model for degeneracy dilution and $Q\to 0$ scaling}
\label{app:degeneracy-toy-model}

In this appendix we work with a discrete toy Hilbert-space model and therefore denote the selection operator by $\hat{O}_S$ to emphasize its operator character. This is the same object that appears as $O_S$ in the main text; the hat is purely a notational reminder in this finite-dimensional setting.

To make the degeneracy-dilution mechanism in Sec.~5 more concrete, we
give a simple toy model in which the quality factor $Q$ can be computed
explicitly and shown to vanish as the system size grows.

Consider a one-dimensional ring of $N$ sites labeled by $i=1,\dots,N$,
with periodic boundary conditions.  At each site we have a local degree
of freedom $s_i$ taking values in a finite set~$\mathcal{A}$ (for
example, spin configurations or coarse-grained field values).  Let
$\mathcal{P}$ denote a fixed pattern of length~$\ell\ll N$, i.e.\ a
specified sequence $(p_1,\dots,p_\ell)\in\mathcal{A}^\ell$.

Define a local pattern-detection operator
\[
\hat o_j :=
\begin{cases}
+1, & \text{if } (s_j,\dots,s_{j+\ell-1}) = \mathcal{P},\\
0,  & \text{otherwise,}
\end{cases}
\]
with indices understood modulo~$N$, and let
\[
\hat O_S := \frac{1}{\sqrt{N}} \sum_{j=1}^N \hat o_j.
\]
This is a simple discrete analogue of an $O_S$ that is sensitive to a
mesoscopic pattern of size~$\ell$ but insensitive to global translations.

\paragraph{Homogeneous / thermal states.}
Consider first a homogeneous or high-temperature state in which each
site is independently and identically distributed according to some
probability measure $\mu$ on~$\mathcal{A}$, with no long-range order.
In such states the random variables $\hat o_j$ have mean
$\mathbb{E}[\hat o_j]=p$, where $p$ is the probability that a random
length-$\ell$ window matches $\mathcal{P}$, and correlations decay
rapidly with separation.  By the central limit theorem,
\[
\mathbb{E}[\hat O_S] = \frac{1}{\sqrt{N}}\sum_{j=1}^N \mathbb{E}[\hat o_j]
  = \sqrt{N}\,p,
\]
while the variance scales as
$\mathrm{Var}(\hat O_S) = \mathcal{O}(1)$ for $N\to\infty$.  If we
define the quality factor as
\[
Q := \frac{|\mathbb{E}[\hat O_S]|}{\sqrt{\mathrm{Var}(\hat O_S)}},
\]
then in homogeneous states we obtain
\[
Q_{\rm hom} \sim \sqrt{N}\,p.
\]
For patterns that are rare in a random background, $p$ typically scales
as $N^{-1}$ or faster (for instance, if one demands that $\mathcal{P}$
match exactly at a specific amplitude threshold), so that
$Q_{\rm hom}\to 0$ as $N\to\infty$.  This is the discrete analogue of
degeneracy dilution: the number of ways to place the pattern grows
linearly with $N$, while the effective overlap per placement shrinks.

\paragraph{Engineered mesoscopic patterns.}
Now consider a state in which the pattern $\mathcal{P}$ is imprinted
deliberately at a small number $k\ll N$ of locations
$\{j_1,\dots,j_k\}$ with aligned phases, while the rest of the system
remains disordered.  In the idealized limit where the pattern is
perfect at those $k$ locations and absent elsewhere, we have
$\hat o_{j_a}=1$ for $a=1,\dots,k$ and $\hat o_j=0$ otherwise, so that
\[
\mathbb{E}[\hat O_S] = \frac{k}{\sqrt{N}},\qquad
\mathrm{Var}(\hat O_S)=0
\]
for that pure configuration.  In the presence of noise and
fluctuations, $\mathrm{Var}(\hat O_S)$ becomes small but nonzero, and
the resulting quality factor $Q$ can approach order unity provided $k$
does not grow with $N$.

\paragraph{Implication.}
This toy model captures in a minimal setting the two regimes exploited
in Sec.~5:
\begin{itemize}
  \item In homogeneous or high-entropy states with many equivalent
        embeddings of the pattern, degeneracy dilution drives
        $Q\to 0$ as $N$ grows, so that contributions of the form
        $\langle O_S\rangle_{\rm in}\langle O_S\rangle_{\rm out}$ to
        collider amplitudes vanish in the thermodynamic limit.
  \item In carefully engineered mesoscopic states with a small number
        of well-aligned pattern embeddings, $Q$ can be of order unity
        even for large~$N$, making such platforms sensitive probes of
        aether resonance.
\end{itemize}
Although highly simplified, the scaling structure of this model is
generic: any observable that averages over $\mathcal{O}(N)$ equivalent
pattern placements while the signal is concentrated in $\mathcal{O}(1)$
of them will display the same $Q\to 0$ behaviour in the homogeneous
limit.
