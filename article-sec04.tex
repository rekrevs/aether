\section{From Local Mediator to Effective Bilocal Action}
\label{sec:mediator}

The fundamental action (3.1) describes a local coupling between visible matter and the mediator $\chi$. To analyse the phenomenology in spacetime $M$, we integrate out the mediator field $\chi$ to obtain an effective action for the visible sector.

\subsection{Integrating out $\chi$}

The equation of motion for the mediator derived from $S_{med} + S_{int}$ is:
\begin{equation}
(\partial_T^2 - c_S^2 \nabla_\sigma^2 + m_\chi^2) \chi(\sigma, T) = J_S(\sigma, T)
\end{equation}

This can be solved using the retarded Green's function $G_{ret}$ on $S$. In the path integral formalism, since the action is quadratic in $\chi$, integrating out $\chi$ yields an effective interaction term for the source current $J_S$:

\begin{equation}
S_{eff}[J_S] = \frac{1}{2} \int dT dT' d\mu(\sigma) d\mu(\sigma') J_S(\sigma, T) G_{ret}(\sigma, T; \sigma', T') J_S(\sigma', T')
\end{equation}

\subsection{The Effective Bilocal Kernel}

In the quasistatic limit where the internal dynamics of $O_S$ are slow compared to the mediator relaxation time, we time-integrate the propagator to obtain the static structural kernel $K_\sigma(\sigma, \sigma') = \int dT G_{ret}$.

Substituting $J_S(x, \sigma) \propto \sqrt{\epsilon} O_S(x) \delta(\pi(\sigma)-x)$, we recover the \textbf{effective bilocal action} used for phenomenology:

\begin{equation}
S_{int}^{eff} \approx \epsilon \int d^4x \sqrt{-g(x)} \int d^4x' \sqrt{-g(x')} O_S(x) \mathcal{K}_{eff}(x, x') O_S(x')
\end{equation}

where $\mathcal{K}_{eff}$ is the pushforward of the structural kernel $K_\sigma$ discussed in Appendix F.

\textbf{Key Consequence:} The apparent non-locality in $M$ (coupling $x$ to $x'$) is physically generated by the propagation of $\chi$ through the pattern space $S$. Because the fundamental propagator $G_{ret}$ is causal in substrate time $\tau$, the effective non-local kernel $\mathcal{K}_{eff}$ is strictly retarded in $\tau$, preventing causal paradoxes (see Sec.~10).

\subsection{Regime of validity and analyticity}
Integrating out the mediator $\chi$ produces an explicitly bilocal term in the emergent spacetime action, Eq.~(7), which is nonlocal in the time coordinate used in the visible sector. At the microscopic level, however, the dynamics is generated by a local action $S_{\rm med}[\chi] + S_{\rm int}[\chi,O_S]$ on $(\sigma,\tau)$ with a standard retarded Green function $G_{\rm ret}$ and a positive kernel $K_\sigma$ on pattern space. As shown in Appendix F, the pushed-forward kernel $\mathcal{K}_{\rm eff}(x,x')$ is positive semidefinite, diffeomorphism-invariant, and strictly retarded in the scalar time $\tau$.

This structure allows one to think of the $\chi$ sector as admitting a Källén--Lehmann--type spectral representation with a non-negative spectral density and support only for $\tau'\ge \tau$. In the present paper we use this representation only at the level of linear response and low-order perturbation theory: the bilocal term is treated as a coarse-grained, retarded interaction valid below a cutoff $\Lambda_S$ set by the mediator mass $m_\chi$ and by the breakdown of the Markov approximation on S-edges. We do not attempt to construct a full high-energy S-matrix or to prove standard analyticity and positivity bounds for scattering amplitudes in this nonlocal EFT. Any UV completion of the substrate dynamics would need to address these questions explicitly; here we restrict attention to observables (Secs.~8 and 12) whose characteristic frequencies and time scales lie well below $\Lambda_S$, where the retarded, positive-definite kernel suffices to ensure stability of the effective description.
