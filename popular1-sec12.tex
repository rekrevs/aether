\section{Predictions: What We Expect to See (Or Not See)}

\begin{quote}
\textbf{Core Concept:} A good theory makes predictions - things that should happen and things that shouldn't.
\end{quote}

\subsection{Negative Predictions (Should NOT Be Seen)}

These are crucial - ways the theory could be falsified:

\textbf{1. No deviations in gravitational laws}
\begin{itemize}
\item Einstein's equations with $\alpha = 1$ (\S3.5)
\item Gravitational waves travel at $c$ (confirmed by LIGO/Virgo + Fermi)
\item Binary pulsar timing, solar system tests - all standard
\end{itemize}

\emph{Test:} Gravitational wave observations, precision tests of GR\\
\emph{Status:} All consistent so far \checkmark

\textbf{2. No robust effects in homogeneous crystals}
\begin{itemize}
\item Degeneracy dilution (\S5.2) suppresses $O_S$ in periodic systems
\item $N \to \infty \Rightarrow J_\sigma \to 0$
\end{itemize}

\emph{Test:} Look for anomalous heat transport or correlations in perfect crystals\\
\emph{Prediction:} None (within noise)

\textbf{3. No signals in accelerator experiments}
\begin{itemize}
\item $O_S$ is RG-irrelevant ($\Delta > 4$) $\Rightarrow$ suppressed at high energies
\item $\langle O_S(E) \rangle \sim (E/\Lambda)^{-n}$, $n > 0$
\end{itemize}

\emph{Test:} Collider searches, rare-decay experiments\\
\emph{Prediction:} Nothing (well below sensitivity)

\textbf{4. No everyday signaling without special conditions}
\begin{itemize}
\item Need: structural similarity (small $d_\sigma$) + coherence (high $\mathcal{Q}$) + pump (active drive)
\item Missing any one $\Rightarrow$ no effect
\end{itemize}

\emph{Test:} Can random objects communicate via aether resonance?\\
\emph{Prediction:} No

These negative predictions are as important as positive ones. They show the theory isn't ``anything goes'' - it's tightly constrained.

\subsection{Positive Predictions (SHOULD Be Seen If Theory Is Correct)}

Now the exciting part - three experiments we can actually build:

\begin{center}\rule{0.5\linewidth}{0.5pt}\end{center}

\subsubsection*{Prediction 1: Twin-Reservoir Correlations (E1)}

\textbf{The Setup:}\\
Two identical ``reservoir computing'' networks (interconnected nodes with nonlinear dynamics):
\begin{itemize}
\item Trained on the same dataset (e.g., handwritten digits, audio clips, video frames)
\item Separated by $> 1$ km (spacelike separation)
\item Optically isolated (no light paths between them)
\item One network (``sender'') is shown a test input
\item Other network (``receiver'') tries to guess what the sender saw
\end{itemize}

\textbf{The Prediction:}\\
Bit error rate (BER) should scale as:

\begin{equation}
BER = \frac{1}{2} \left( 1 - \eta_{\rm BER} \, e^{-d_\sigma/\lambda_\sigma} \right).
\tag{12.1}
\end{equation}

\textbf{What this means:}

\textbf{BER = 0.5:} Pure chance (50\% error rate - flipping coins)\\
\textbf{BER < 0.5:} Better than chance (some information is getting through)

\textbf{For perfect match ($d_\sigma = 0$):}\\
$BER_{min} = (1 - \eta_{\rm BER})/2 \approx 0.25$ (for $\eta_{\rm BER} \sim 0.5$)

\textbf{For strong mismatch ($d_\sigma \gg \lambda_\sigma$):}\\
$BER \to 0.5$ (back to chance)

\textbf{Numerical target:}
\begin{itemize}
\item Report $\Delta BER \sim 10^{-3}$ (difference between match and mismatch)
\item With $10^9$ bits tested
\item Primary test statistic: \textbf{cross-correlation} or \textbf{coherence} (more sensitive than raw BER)
\item Analysis: Sequential Probability Ratio Test (SPRT), permutation tests, Holm-Bonferroni correction
\item \textbf{Delayed choice:} Use quantum random number generator (QRNG) to select test inputs after commit
\item \textbf{Spacelike separation:} Ensure light-travel time $> 3$ $\mu$s between sender and receiver
\end{itemize}

\textbf{Expected signal:}\\
Not ``telepathic images'' but subtle statistical correlations:
\begin{itemize}
\item Receiver's internal state becomes slightly more correlated with sender's
\item Cross-correlation function shows peak at zero lag
\item Coherence measure (frequency-domain correlation) shows coupling
\end{itemize}

\textbf{Mental model:}\\
Like two tuning forks. When one is struck (sender shown input), the other starts to hum faintly (receiver's state shifts). The effect is tiny but measurable statistically over many trials.

\textbf{Null bound:}\\
If $BER \geq 0.49$ for all configurations (no better than chance):

$\varepsilon \lambda_\sigma \mathcal{Q} < 10^{-12}$ m

This directly constrains the product of coupling, coherence length, and quality factor.

\begin{center}\rule{0.5\linewidth}{0.5pt}\end{center}

\subsubsection*{Prediction 2: Energy Tunnel (E2)}

\textbf{The Setup:}\\
Two identical systems (superconducting cavities or metamaterial resonators):
\begin{itemize}
\item Separated by $> 1$ km
\item Cavity A is pumped with microwave power ($P_{pump} \sim 1$ $\mu$W)
\item Cavity B is below threshold (not pumped)
\item Measure energy balance with cryo-calorimetry ($T \sim 10$ mK, $\delta E \sim 10^{-26}$ J)
\end{itemize}

\textbf{The Prediction:}\\
Differential energy balance:

\begin{equation}
  \Delta E_A + \Delta E_B \;=\; P_\sigma\,\Delta t \,,
  \tag{12.2}
\end{equation}

where the substrate power flow is:

$P_\sigma \sim \varepsilon \hbar\omega_0 \mathbb K \mathcal{Q} \Krate \tilde{\Delta\Phi}$

\textbf{Three scenarios for detectability:}

\begin{center}
\begin{tabular}{lccc}
\toprule
Scenario & $\mathcal{Q}$ & $\Delta E$ ($10^3$ s) & Detectability \\
\midrule
\textbf{Baseline} & $10^{-5}$ & $\sim 10^{-40}$ J & Not detectable ($\delta E \sim 10^{-26}$ J) \\
\textbf{Target} & $10^{-3}$ & $\sim 10^{-27}$ J & Below limit but approaching \\
\textbf{Ambitious} & $10^{-2}$ & $\sim 10^{-25}$ J & Marginally detectable at limit \\
\bottomrule
\end{tabular}
\end{center}

\textbf{Parameter example (``target'' scenario):}
\begin{itemize}
\item $P_{pump} = 1$ $\mu$W
\item $K \approx 0.5$ (50\% structural match)
\item $\tilde{\Delta\Phi} \approx 1$ (order unity potential difference)
\item $\varepsilon = 10^{-15}$
\item $\mathcal{Q} = 10^{-8}$
\end{itemize}

Result:\\
$J_\sigma \approx 5 \times 10^{-30}$ W

Over $\Delta t = 1000$ s:\\
$\Delta E \approx 5 \times 10^{-27}$ J

\textbf{This is just below the detection limit} ($\delta E \sim 10^{-26}$ J for mK calorimetry), indicating we're at the edge of feasibility.

\textbf{What we'd see:}
\begin{itemize}
\item Cavity A loses slightly more energy than it radiates locally
\item Cavity B gains a tiny amount of energy (heating above ambient)
\item The discrepancy $\Delta E_A + \Delta E_B \neq 0$ (to within measurement uncertainty)
\item Anti-correlation: When A loses excess, B gains; when A's excess decreases, B cools
\end{itemize}

\textbf{Key tests:}
\begin{enumerate}
\item \textbf{Matching test:} Vary internal geometry (0\%, 50\%, 100\% match) $\to$ correlation should track matching
\item \textbf{Latency scan:} Look for FTL arrival ($\delta t < -3$ $\mu$s negative lag vs. thermal leakage $\delta t > 0$ positive lag)
\item \textbf{Phase-locking on/off:} Coherence $\mathcal{Q}$ should matter - test by destroying/restoring phase coherence
\item \textbf{Sidereal modulation:} Mount on rotation platform, look for 23h 56m period (scan preferred frame)
\end{enumerate}

\textbf{Null bound:}\\
If $|\Delta E| < 10^{-26}$ J after $10^6$ s:

$\varepsilon \omega_0 \mathcal{Q} < 10^{-8}$ Hz

\begin{center}\rule{0.5\linewidth}{0.5pt}\end{center}

\subsubsection*{Prediction 3: Anisotropic Daily Modulation (E2 - Rotation Test; Matter Sector)}

\textbf{The Setup:}\\
Same as E2, but specifically looking for \textbf{modulation} as Earth rotates:

\textbf{The Prediction:}

\begin{equation}
  P_\sigma(t) \;=\; \bar P_\sigma\left[1+ \mathcal A\cos\!\big(\Omega_\oplus t + \phi\big)\right],
  \tag{12.3}
\end{equation}

with amplitude \textbf{in the matter sector}:

\begin{equation}
  A_{\rm sid}^{(\mathrm{mat}/\gamma)} \;\simeq\;
  \varepsilon \left(\frac{\lambda_\sigma}{L_{\rm exp}}\right)\mathcal Q \,\Xi \,,
  \tag{12.4}
\end{equation}

\textbf{Important distinction:}\\
This probes the \textbf{matter sector} ($\varepsilon_{mat}$, $\mathcal{Q}_{mat}$) as opposed to the \textbf{optical sector} ($\varepsilon_\gamma$, $\mathcal{Q}_\gamma$) tested by Michelson-Morley type experiments.

The bounds in \S11 (Eq. 11.4) constrain $\varepsilon_\gamma \mathcal{Q}_\gamma$ (optics). They don't directly bind (12.4) in the matter sector - see discussion of sector separation.

\textbf{Numerical target (3$\sigma$, $10^7$ s $\approx$ 115 days):}

$A_{sid} \gtrsim 10^{-20}$

\textbf{Stretch goal:} $A_{sid} \gtrsim 5 \times 10^{-21}$

\textbf{What we'd see:}
\begin{itemize}
\item Energy transfer power $P_\sigma$ varies sinusoidally
\item Period: 23h 56min 4.1s (sidereal day, not solar day)
\item Amplitude: $\sim 10^{-20}$ fractional variation
\item Phase $\phi$ related to apparatus orientation relative to preferred frame
\end{itemize}

\textbf{Why sidereal, not solar?}
\begin{itemize}
\item Solar day (24 hours): Earth's rotation relative to the Sun
\item Sidereal day (23h 56m): Earth's rotation relative to distant stars (inertial frame)
\item The preferred frame is fixed relative to the cosmos (like the CMB rest frame), not the Sun
\end{itemize}

\textbf{Experimental challenge:}\\
Distinguishing a $10^{-20}$ signal at 23h 56m from:
\begin{itemize}
\item Thermal fluctuations
\item Building vibrations (often 24h due to human activity)
\item Atmospheric pressure (solar-driven)
\end{itemize}

\textbf{Solution:}
\begin{itemize}
\item Long integration (months)
\item Multiple apparatuses at different latitudes/longitudes
\item Cross-correlation between sites
\item Blind analysis (seal sidereal phase prediction before unblinding)
\end{itemize}

\textbf{Null bound:}\\
If $\hat{A}_{sid} < 10^{-20}$:

$\varepsilon_{mat} \mathcal{Q}_{mat} < 10^{-20}$ (for given $L_{exp}$ and $\lambda_\sigma$ - report both)

\begin{center}\rule{0.5\linewidth}{0.5pt}\end{center}

\subsubsection*{Prediction 4: Complexity Optimum (E3)}

\textbf{The Setup:}\\
Two chaotic systems (turbulent flows or reaction-diffusion patterns):
\begin{itemize}
\item Identical geometry
\item Driven by modulated input (heat flux or chemical feed)
\item Vary drive complexity: pure sine $\to$ music $\to$ speech $\to$ white noise
\end{itemize}

\textbf{The Prediction:}\\
Sync-hop rate (simultaneous attractor transitions) vs. drive complexity:

\begin{equation}
r_{sync} = r_0 \, \Sigma_{drive} \, e^{-\Sigma_{drive} / \Sigma_{opt}},
\tag{12.5}
\end{equation}

\textbf{What this means:}

\textbf{$r_{sync}$:} Rate of simultaneous ``hops'' between attractors in both systems\\
\textbf{$\Sigma_{drive}$:} Algorithmic complexity of the driving signal (bits/sample)\\
\textbf{$\Sigma_{opt}$:} Optimal complexity where resonance is strongest

\textbf{Predicted behavior:}
\begin{itemize}
\item \textbf{White noise} ($\Sigma \to \infty$): $r_{sync} \to 0$ (no structure to match)
\item \textbf{Pure sine} ($\Sigma \to 0$): $r_{sync} \to 0$ (too simple, no diversity)
\item \textbf{Music/speech} ($\Sigma \sim \Sigma_{opt} \sim 5$ bits/sample): $r_{sync} \sim$ max (rich but compressible - ``interesting'')
\end{itemize}

\textbf{Visual analogy:}\\
Like tuning a radio. Too low frequency (pure tones) - no signal. Too high frequency (noise) - no signal. Just right (modulated carrier with information) - clear reception.

\textbf{What we'd measure:}
\begin{enumerate}
\item \textbf{Attractor topology:} Use persistent homology $\to$ Betti curves $\beta_0(r)$, $\beta_1(r)$
\item \textbf{Hop detector:} $|\Delta\beta_1| > \theta$ within 1 second
\item \textbf{Complexity scan:} Five levels from sine to noise
\item \textbf{Permutation test:} Shuffle timestamps $10^6$ times $\to$ p-value
\item \textbf{Mismatch control:} Change geometry (10\%, 20\%, 50\%) $\to$ expect $r_{sync} \propto \exp[-d_\sigma/\lambda_\sigma]$
\end{enumerate}

\textbf{Expected result:}\\
Unimodal curve: $r_{sync}$ peaks at intermediate $\Sigma$, falls off on both sides.

\textbf{Why this matters:}\\
It tests a unique prediction - that structural similarity prefers ``interesting'' patterns (compressible but rich), not simple or random ones.

\textbf{Null bound:}\\
If no excess synchronization above random baseline:

$\varepsilon \mathcal{Q} < 10^{-15}$

\begin{center}\rule{0.5\linewidth}{0.5pt}\end{center}

\subsection{Summary Table}

\begin{center}
\begin{tabular}{lcc}
\toprule
Experiment & Positive signal & Null bound \\
\midrule
E1 (ansible) & $\Delta BER \sim 10^{-3}$ (match vs. mismatch) & $\varepsilon \lambda_\sigma \mathcal{Q} < 10^{-12}$ m \\
E2 (energy) & $\Delta E > 10^{-25}$ J ($\mathcal{Q} \sim 10^{-2}$) & $\varepsilon \omega_0 \mathcal{Q} < 10^{-8}$ Hz \\
E2 (rotation, matter) & $A_{sid} \geq 10^{-20}$ (3$\sigma$, $10^7$ s) & $\varepsilon_{mat} \mathcal{Q}_{mat} < 10^{-20}$ \\
E3 (chaos) & $r_{sync}$ peak at $\Sigma_{opt}$ & $\varepsilon \mathcal{Q} < 10^{-15}$ \\
\bottomrule
\end{tabular}
\end{center}

\subsection{Parameter Mapping (End-to-End)}

\textbf{Which observables constrain which parameters:}

\begin{center}
\begin{tabular}{lll}
\toprule
Observable & Primary constraint & Notes \\
\midrule
E1 ($\Delta BER$, coherence) & $\varepsilon \lambda_\sigma \mathcal{Q}$ & Use distance ladder (\S7) \\
E2 ($\Delta E$ over time) & $\varepsilon \omega_0 \mathcal{Q}$ & Power form (6.1) \\
E2 (sidereal amplitude) & $\varepsilon \mathcal{Q} (\lambda_\sigma/L_{exp})$ & Geometry factor $\Xi$ (11.5) \\
E3 (hop rate vs. complexity) & $\lambda_\sigma$, $\mathcal{Q}$ & Log-linear fall with mismatch \\
\bottomrule
\end{tabular}
\end{center}

\textbf{The beauty:}\\
Different experiments constrain different parameter combinations. By combining all three, we can (in principle) separately determine $\varepsilon$, $\lambda_\sigma$, $\mathcal{Q}$, and $\omega_0$.

\begin{center}\rule{0.5\linewidth}{0.5pt}\end{center}

